\documentclass[12pt]{article}
\usepackage{lingmacros}
\usepackage{tree-dvips}

%\https://manualdelatex.com/tutoriales/listas-y-enumeraciones

\usepackage{geometry}
 \geometry{
 a4paper,
 total={170mm,257mm},
 left=20mm,
 top=20mm,
 }

\begin{document}

\begin{center}
CURRICULUM

V I TA E

González Santos Leopoldo
\end{center}

\centerline{\rule{7cm}{0.8pt}}
\bigskip

Masculino 1957-10-28

Laboratorio C-12, Instituto de Neurobiología - UNAM, Boulevard Juriquilla 3001, Juriquilla, Querétaro, México, C.P. 76230, Querétaro, 
México

52 (55) 5623 40 53, (442) 238 10 53

lgs@servidor.unam.mx GOSL571028H48 Técnico académico Tit. B T.C.

C Neurobiología Conductual y Cognitiva 803564 7 Esquivel Quiroz Bertha

\hfill

Perfil

Categorías

1. Técnico académico Tit. B T.C., Instituto de Neurobiología UNAM-Juriquilla, 1996-02-01.

\hfill

\textbf{Escolaridad}

\hfill

1. Maestría en Ciencias de la Computación, Maestría, Centro de Investigación en Computación, IPN, 1999.

\hfill

2. Licenciado en Física y Matemáticas, Licenciatura, Esc. Superior de Física y Matemáticas del I.P.N, 1980.

\hfill

\textbf{Cursos recibidos}

\hfill

\begin{enumerate}
\item AFNI Bootcamp, Duración: Semanal, Modalidad: Presencial, Horas por semana: 7, Total de horas: 35, UNAM, Campus Juriquilla, 
Querétaro, País: México, Fecha de inicio: noviembre de 2024, Fecha de conclusión: noviembre de 2024.

\item FSL, Duración: Semanal, Modalidad: Presencial, Horas por semana: 40, Total de horas: 40, University of British Columbia, 
Vancouver, 
País: Canadá, Fecha de inicio: junio de 2017, Fecha de conclusión: junio de 2017.

\item XNAT, Workshop, Duración: Semanal, Modalidad: Presencial, Horas por semana: 6, Total de horas: 30, St Louis MO, País: Estados 
Unidos 
de América, Fecha de inicio: junio de 2016, Fecha de conclusión: junio de 
2016.

\item Míneria de datos y Datawarehouse, Duración: Otro, Modalidad: Presencial, Horas por semana: 12, Total de horas: 12, Facultad de 
Contaduría y Administración, UNAM, País: México, Fecha de inicio: marzo de 2007, Fecha de conclusión: marzo de 2007.

\item Teórico - práctico de Microscopia, Duración: Otro, Modalidad: Presencial, Horas por semana: 10, Total de horas: 10, Instituto de 
Neurobiología, País: México, Fecha de inicio: abril de 2005, Fecha de conclusión: abril de 2005.

\item Resolución y Definición en el Procesamiento de Imagen Digital , Duración: Semanal, Modalidad: Presencial, Horas por semana: 16, 
Total de horas: 16, Alta Tecnología en laboratorios S.A., País: México, Fecha de inicio: octubre de 2001, Fecha de conclusión: octubre 
de 2001.

\item Administración de sistemas de Videoconferencia, Duración: Semanal, Modalidad: Presencial, Horas por semana: 24, Total de horas: 
24, 
DGSCA-VTEL Corporation, D.F. México., País: México, Fecha de inicio: septiembre de 1997, Fecha de conclusión: septiembre de 1997.

\item Operator Training II Avanzado, Duración: Semanal, Modalidad: Presencial, Horas por semana: 40, Total de horas: 40, Tandem 
Computers 
Software Education, D.F., México, País: México, Fecha de inicio: abril de 1992, Fecha de conclusión: abril de 1992.

\item Gerencia Moderna, Duración: Semanal, Modalidad: Presencial, Horas por semana: 40, Total de horas: 40, Centro Educacional para el 
Norte de América Latina, D.F., México, País: México, Fecha de inicio: febrero de 1992, Fecha de conclusión: febrero de 1992.

\item Planeación de la Capacidad en Informática , Duración: Semanal, Modalidad: Presencial, Horas por semana: 40, Total de horas: 40, 
IBM 
de México, S.A. Centro Educacional para el Norte de América Latina. D.F. México, País: México, Fecha de inicio: octubre de 1991, Fecha 
de conclusión: octubre de 1991.

\item Metodología de la Planeación Estratégica, Duración: Semanal, Modalidad: Presencial, Horas por semana: 40, Total de horas: 40, IBM 
de México, S.A. Centro Educacional para el Norte de América Latina, D.F., México, País: México, Fecha de inicio: septiembre de 1991, 
Fecha de conclusión: septiembre de 1991.

\item Lenguaje Ensamblador, Duración: Semanal, Modalidad: Presencial, Horas por semana: 40, Total de horas: 40, IBM de México, S.A. 
Centro Educacional para el Norte de América Latina, D.F., México, País: México, Fecha de inicio: marzo de 1990, Fecha de conclusión: 
marzo de 1990.

\item Migración a MVS/XA 2.20, Duración: Semanal, Modalidad: Presencial, Horas por semana: 40, Total de horas: 40, IBM de México, S.A. 
Centro Educacional para el Norte de América Latina, D.F. México., País: México, Fecha de inicio: diciembre de 1989, Fecha de 
conclusión: diciembre de 1989.

\item CICS Determinación de Problemas, Duración: Semanal, Modalidad: Presencial, Horas por semana: 40, Total de horas: 40, IBM de 
México, 
S.A. Centro Educacional para el Norte de América Latina. D.F. México., País: México, Fecha de inicio: noviembre de 1989, Fecha de 
conclusión: noviembre de 1989.

\item MVS Determinación de Problemas , Duración: Semanal, Modalidad: Presencial, Horas por semana: 40, Total de horas: 40, IBM de 
México, 
S.A. Centro Educacional para el Norte de América Latina. D.F. México., País: México, Fecha de inicio: noviembre de 1989, Fecha de 
conclusión: noviembre de 1989.

\item VTAM Determinación de Problemas, Duración: Semanal, Modalidad: Presencial, Horas por semana: 40, Total de horas: 40, IBM de 
México, 
S.A. Centro Educacional para el Norte de América Latina, D.F, México, País: México, Fecha de inicio: octubre de 1989, Fecha de 
conclusión: octubre de 1989.

\item JES2 Implantación, Duración: Semanal, Modalidad: Presencial, Horas por semana: 40, Total de horas: 40, IBM de México, S.A. Centro 
Educacional para el Norte de América Latina, D.F., México, País: México, Fecha de inicio: junio de 1988, Fecha de conclusión: junio de 
1988.

\item Ambiente de Comunicación de Datos, Duración: Semanal, Modalidad: Presencial, Horas por semana: 40, Total de horas: 40, , País: 
México, Fecha de inicio: marzo de 1988, Fecha de conclusión: marzo de 1988.

\item TSO Utilización, Duración: Semanal, Modalidad: Presencial, Horas por semana: 40, Total de horas: 40, , País: México, Fecha de 
inicio: marzo de 1988, Fecha de conclusión: marzo de 1988.
\end{enumerate}


\textbf{Asistencia a Congresos Internacionales}

\begin{enumerate}
\item Romero-Melendez, C., Castillo-Fernandez, D., and Gonzalez-Santos, L., “Asymptotic stability in a controlled stochastic 
Lotka-Volterra model with Lévy noise”, 13th International Conference on Pure and Applied Mathematics, ICPAM, País: Croacia, Tipo: 
Conferencia, Ámbito: Internacional, julio de 2024.

\item Romero-Melendez, C., Castillo-Fernandez, D., and Gonzalez-Santos, L., “On Stability Properties in a Stochastic Controlled 
Lotka-Volterra Model”, 7th International Conference on Mathematics and Statistics, ICoMS, País: Portugal, Tipo: Conferencia, Ámbito: 
Internacional, junio de 2024.

\item Vázquez, D., Martínez-Soto, J., Barrios, F., González-Santos, Pasaye, E., Alcauter, S., "Impact of a stress stimulus and images 
with 
restorative potential in the amygdala's resting state functional connectivity", , País: Estados Unidos de América, Tipo: Congreso, 
Ámbito: Internacional, noviembre de 2016.

\item Vazquez, D., Martinez-Soto, J., Barrios, F., Gonzalez-Santos, L., Pasaye, E., Alcauter, S., "Amygdala's functional connectivity 
in 
stress and after stimuli with low/high restorative potential", 22nd Annual Meeting of the OHBM. Organization for Human Brain Mapping, 
País: Suiza, Tipo: Congreso, Ámbito: Internacional, junio de 2016.
\end{enumerate}

\textbf{Participación institucional}

\begin{enumerate}
\item Asistencia técnica de análisis de imagen, Instituto de Neurobiología UNAM-Juriquilla, INB-UNAM, INB-UNAM, Fecha de inicio: agosto 
de 
2005, Fecha de conclusión: septiembre de 2005.

\item Asistencia técnica de análisis de imagen, Instituto de Neurobiología UNAM-Juriquilla, INB-UNAM, INB-UNAM, Fecha de inicio: julio 
de 
2004, Fecha de conclusión: agosto de 2004.

\item Asistencia técnica de análisis de imagen, Instituto de Neurobiología UNAM-Juriquilla, INB-UNAM, INB-UNAM, Fecha de inicio: agosto 
de 
2003, Fecha de conclusión: agosto de 2003.

\item Asistencia técnica de análisis de imagen, Instituto de Neurobiología UNAM-Juriquilla, INB-UNAM, INB-UNAM, Fecha de inicio: julio 
de 
2003, Fecha de conclusión: julio de 2003.
\end{enumerate}

\textbf{Agradecimientos, créditos, premios y estímulos}

\hfill

\textbf{Agradecimientos en tesis de doctorado}

\begin{enumerate}
\item Sánchez Moncada, I.C., “PARTICIPACIÓN DE LAS CORTEZAS PREMOTORAS EN EL FENÓMENO DE GENERALIZACIÓN TEMPORAL ENTRE TAREAS DE 
PERCEPCIÓN TEMPORAL Y TAREAS RÍTMICAS MOTORAS”, DOCTORADO EN CIENCIAS BIOMÉDICAS, Créditos en: Tesis Doctorado, noviembre de 2024.

\item RODRÍGUEZ-VIDAL, LL., “ESTUDIO DE LA CONECTIVIDAD FUNCIONAL EN ESTADO DE REPOSO DEL CLAUSTRUM”, DOCTORADO EN CIENCIAS BIOMÉDICAS, 
UNAM, 2023, Créditos en: Tesis Doctorado, diciembre de 2023.

\item Aguilar-Moreno, J. A., “Conducta sexual y actividad cerebral evaluada por resonancia magnética por incrementos de manganeso 
(MEMRI) 
en ratas hembra de la cepa Wistar”, Instituto de Neurobiología, 2023, Créditos en: Tesis Doctorado, noviembre de 2023.

\item Olalde Mathieu, V.E., "CONECTIVIDAD FUNCIONAL CEREBRAL RELACIONADA A COMPONENTES DE LA RESPUESTA EMPATICA EN PSICOTERAPEUTAS, 
Créditos en: Tesis Doctorado, diciembre de 2022.

\item Gracia-Tabuenca, Z., “Desarrollo de la conectividad funcional cerebral en la adolescencia y su correlato con el 
desempeño cognitivo”, Instituto de Neurobiología, UNAM, Créditos en: Tesis Doctorado, noviembre de 2020.

\item Ortiz Retana, J. J., “Redes cerebrales en estado de reposo de topillo de la pradera (microtus ochrogaster)”, Doctorado en 
Ciencias 
Biomedicas (Neurobiología), INB-UNAM,, Créditos en: Tesis Doctorado, mayo de 2019.

\item Rodríguez Cruces, R., “Anormalidades de la conectividad cerebral y deterioro cognitivo en la epilepsia del lóbulo temporal”, 
Doctorado en Ciencias Biomédicas (Neurobiología), INB-UNAM, Créditos en: Tesis Doctorado, mayo de 2019.

\item García Gomar, María G., “Tractografía del Subtálamo en Pacientes con Enfermedad deParkinson”, Créditos en: Tesis Doctorado, julio 
de 
2017.

\item Reyes-Aguilar, Azalea, "Correlatos Neuronales de la Inferencia de Estados Mentales de Otros Contextos de Cooperación VS No 
Cooperación", Créditos en: Tesis Doctorado, septiembre de 2016.

\item Alejandra A. Angulo Perkins, "Identificación mediante resonancia magnética funcional de regiones corticales que participan en la 
percepción de la música y el habla", Instituto de Neurobiología, UNAM, Créditos en: Tesis Doctorado, diciembre de 2015.

\item Sofía González-Salinas, "Eventos Moleculares Asociados a la Evocación de la Memoria y a la Extinción", Instituto de 
Neurobiología, 
UNAM, Créditos en: Tesis Doctorado, diciembre de 2015.

\item Martha B. Moreno-García, "Correlación entre área y la microestructura del Cuerpo Calloso y Pocesos Cognitivos en niños de 7 a 9 
años", Instituto de Neurobiología, UNAM, Créditos en: Tesis Doctorado, septiembre de 2015.

\item Mercadillo-Caballero R.E., “Correlatos Cerebrales de la Experiencia Emocional de Compasión y la Impulsividad”, Tesis Doctoral, 
Instituto de Neurobiología., Créditos en: Tesis Doctorado, noviembre de 2012.

\item Pasaye Alcaraz, E. H., "Resonancia Magnética Funcional a 3.0 T en sujetos amputados durante la sensación del Miembro Fantasma", 
Instituto de Neurobiología, Créditos en: Tesis Doctorado, noviembre de 2011.

\item Sandoval Minero, M.T, "Regulación de la proyección axonal de neuronas decusantes en el rombencéfalo caudal", Créditos en: Tesis 
Doctorado, 2008.

\item Martín González, Cecilia, "Regulación Gabáergica de la actividad de las adenilato ciclasas en las células GT1", Créditos en: 
Tesis 
Doctorado, 2007.

\item Frías Castañeda, María del Carmen,"Efecto de la desnutrición posnatal en la morfología celular del bulbo olfatorio", Créditos en: 
Tesis Doctorado, 2007.

\item Martínez-García, M. I., “Participación del hipocampo en la memoria de largo plazo”, Doctorado en ciencias biomédicas, UNAM-INB, 
Créditos en: Tesis Doctorado, 2003.
\end{enumerate}

\textbf{Agradecimientos en tesis de maestría}

\begin{enumerate}
\item Gaspar Martinez, E., “Diferencias volumétricas en estructuras cerebrales de pacientes con enfermedad de Parkinson, prodrómicos y 
controles, y su asociación con perfiles cognitivos”, Maestría en Ciencias (Neurobiología), Créditos en: Tesis Maestría, agosto de 2024.

\item Mares Román, C.P., “Caracterización de la sincronía auditivo-motora para distintos estímulos y efectores”, Maestría en Ciencias 
(Neurobiología), UNAM-Juriquilla, Créditos en: Tesis Maestría, diciembre de 2022.

\item Reyes González, I.D., “Estudio de los correlates neutrales de la percepción emocional por análisis de patrones en multitud de 
voxels”, Créditos en: Tesis Maestría, noviembre de 2021.

\item Brzezinski Rittner, A., “Estudio Longitudinal de la conectividad funcional en respuesta a una tarea de memoria de trabajo, 
n-back, 
por interacciones psico-fisiológicas”, Maestria en Ciencias (Neurobiología), INB-UNAM, Créditos en: Tesis Maestría, agosto de 2019.

\item Catillo López, G., “Relación de anormalidades en sustancia blanca y acumulaciones de alfasinucleína en biopsias de piel de 
pacientes 
con parkinson”, Maestria en Ciencias (Neurobiología), INB-UNAM, Créditos en: Tesis Maestría, mayo de 2019.

\item Rasgado Toledo, J., “La modulación de la intención comunicativa mediante la expresión facial y su correlato neural”, Maestria en 
Ciencias (Neurobiología), INB-UNAM, Créditos en: Tesis Maestría, octubre de 2019.

\item López Gutiérrez, M.F., “Efectos de la formación de vínculos de pareja sobre la conectividad funcional cerebral en microtus 
ochrogaster”, Maestria en Ciencias Físicas (Neurobiología), INBUNAM, Créditos en: Tesis Maestría, junio de 2019.

\item Camacho Herrera, C., “Caracterización de la variabilidad en la localización de la corteza motora suplementaria usando referencias 
anatómicas extracraneales”, Maestria en Ciencias Físicas (Neurobiología), INB-UNAM, Créditos en: Tesis Maestría, junio de 2019.

\item Rojas Vite, G., "Evaluación de la microestructura de la substancia blanca en regiones de cruce de fibras", Maestría en Ciencias 
(Neurobiología), UNAM, Créditos en: Tesis Maestría, noviembre de 2018.

\item Zamora Ursulo, M.A., "El correlato neural de la comprensión de la metáfora en adultos hispanohablantes", Maestría en Ciencias 
(Neurobiología), UNAM, Créditos en: Tesis Maestría, noviembre de 2018.

\item Martínez López, A. Y., "Correlatos Neuronales de la atención sostenida evaluada por estados de conectividad funcional dinámica", 
Maestría en Ciencias (Neurobiología), UNAM, Créditos en: Tesis Maestría, noviembre de 2018.

\item Carrillo-Peña, A.V., “Estudio de los Refranes Mexicanos y sus Correlatos Conductuales y Neurales”, Ciencias(Neurobiología), 
UNAM-Juriquilla, , Créditos en: Tesis Maestría, diciembre de 2017.

\item Rueda Zarazúa, Bertha G., “Análisis de la Carga Mutacional en el Genoma Mitocondrial enFibroblastos de Pulpa Dental Humana”, 
Créditos en: Tesis Maestría, julio de 2017.

\item Navarrete-Acevedo, Norma E., "Maduración de las Propiedades de Conectividad Funcional Cerebral en la Adolescencia", Créditos en: 
Tesis Maestría, julio de 2016.

\item Olalde-Mathieu, Víctor E., "Caracterización de la Conectividad Funcional Cerebral relacionada a Componentes de la Respuesta 
Empática", Créditos en: Tesis Maestría, agosto de 2016.

\item Circe A. Wilke-Quintero, "Participación de la Vía Dorsal auditiva en la Percepción del ritmo en la música, evaluada mediante 
IRMf",Instituto de Neurobiología, UNAM, Créditos en: Tesis Maestría, agosto de 2015.

\item Jiménez Valverde L.O. “Descripción de la Actividad Cortical asociada a la memoria de trabajo en pacientes con epilepsia del 
lóbulo 
temporal Un estudio de resonancia magnética funcional”, Tesis, Maestría en Ciencias (Neurobiología), UNAM, Créditos en: Tesis Maestría, 
septiembre de 2014.

\item Manni Zepeda M., “Correlatos Neurales de la Observación Natural de un Video con Carga Emocional”, Tesis, Maestría en Ciencias 
(Neurobiología), UNAM, Créditos en: Tesis Maestría, septiembre de 2014.

\item Cuaya Retana, L. V., "Correlatos Cerebrales de una Decisión Causal evaluados mediante Resonancia Magnética Funcional, Créditos 
en: 
Tesis Maestría, septiembre de 2013.

\item Rodríguez Nieto G., “Estudio de la Respuesta Hemodinámica en la Ínsula y Corteza Prefrontal en un Modelo de Compasión por 
Resonancia Magnética Funcional”, Tesis, Maestría en Ciencias (Neurobiología), UNAM, Créditos en: Tesis Maestría, septiembre de 2013.

\item Cruz García, M. G., "Efectos de la exposición al arsénico sobre los transportadores de glucosa y el receptor de insulina en el 
hipocampo de ratones macho de la cepa C57BL-6J", Instituto de Neurobiología, UNAM, Créditos en: Tesis Maestría, agosto de 2012.

\item Domínguez Vargas, A. U., "Análisis del neurodesarrollo de los movimientos oculares en niños con leucomalacia periventricular", 
Instituto de Neurobiología, UNAM, Créditos en: Tesis Maestría, octubre de 2011.

\item Cruz García M.G., “Efectos de la exposición al arsénico sobre los transportadores de glucosa y el receptor de insulina en el 
hipocampo de ratones macho de la cepa C57BL/6J”, Maestro en Ciencias (Neurobiología), Créditos en: Tesis Maestría, noviembre de 2010.

\item Rodríguez-Vidal, Lluviana, Diferencias en el metabolismo cerebral de acuerdo al temperamento en sujetos control, 
UNAM-INB, Créditos en: Tesis Maestría, 2009.

\item Cano Sotomayor, Yoana Daniela, "Efecto del estradiol sobre la neurogénesis en el estriado lesionado con ácido kaínico en la rata 
hembra adulta", Créditos en: Tesis Maestría, noviembre de 2007.

\item Cipriano, Margarita de Jesús, Diferencias entre imágenes anisotrópicas por difusión de sujetos control y pacientes con enfermedad 
de parkinson idiopática, UNAM-INB, Créditos en: Tesis Maestría, 2007.

\item Mercadillo Caballero, Roberto Emmanuele, "Correlatos Cerebrales de la Percepción del Sufrimiento en otro: Un estudio por 
Resonancia 
Magnética Funcional", Créditos en: Tesis Maestría, mayo de 2007.

\item Díaz Trujillo, A., “Identificación de genes asociados con la consolidación de la memoria en el neoestriado mediante microarreglos 
de cDNA”, Maestría en ciencias (neurobiología), UNAM-INB, Créditos en: Tesis Maestría, 2005.

\item Herrera-Alarcón, J., “Evaluación del desarrollo de las células de sertoli en ovinos durante el período postnatal”, Maestría en 
ciencias (neurobiología), UNAM-INB, Créditos en: Tesis Maestría, 2005.

\item Arroyo-Helguera, O. E., “Papel regulatorio de la región 3 no traducible de los RNA mensajeros de la enzima desyodasa tipo 1: 
Efecto 
sobre su traducción y estabilidad”, Maestría en ciencias (neurobiología), UNAM-INB, Créditos en: Tesis Maestría, 2004.

\item Galindo-Martínez, L. E., “Efecto del agotamiento de serotonina cerebral sobre los procesos de adquisición y retención en una 
tarea 
de evitación activa”, Maestría en ciencias (neurobiología), UNAM-INB, Créditos en: Tesis Maestría, 2003.

\item Morales-Vega, D. A., “Estudio de la activación medular por medio de la resonancia magnética funcional”, Maestría en ciencias 
(neurobiología), UNAM-INB, Créditos en: Tesis Maestría, 2003.

\item González-Torres, M., A., “Localización de los receptores s estriadol en el hipotálamo de la oveja y sus variaciones en relación 
con 
el parto y la conducta maternal”, Maestría en ciencias (neurobiología), UNAM-INB, Créditos en: Tesis Maestría, 2002.

\item Castillo-Martín del Campo, C. G., “Transplante de una línea celular inmortalizada productora de gaba en amígdala de la rata 
albina”, Maestría en ciencias (neurobiología), UNAM-INB, Créditos en: Tesis Maestría, 2002.

\item Hurtazo-Oliva, H. A., “Evaluación del funcionamiento del sistema vomeronasal por fos en ratas macho con lesiones del área 
preóptica 
media del hipotálamo anterior”, Maestría en ciencias (neurobiología), UNAM-INB, Créditos en: Tesis Maestría, 2002.

\item Huerta-Ocampo, I., “Dimorfismo sexual en la comisura anterior de la rata: Estudio morfológico y morfométrico”, Maestría en 
ciencias 
(neurobiología), UNAM-INB, Créditos en: Tesis Maestría, 2001.

\item Mena-Segovia, J., “Efectos de la lesión excitotoxica del estriado sobre la actividad electroencefalografica durante el ciclo 
vigilia sueño en la rata albina”, Maestría en ciencias (neurobiología), UNAM-INB, Créditos en: Tesis Maestría, 2001.

\item Hernández-Montiel, H. L., “Control del crecimiento axonal longitudinal en el sistema nervioso central de vertebrados”, Maestría 
en 
ciencias (neurobiología),UNAM-INB, Créditos en: Tesis Maestría, 2001.

\item Geovannini-Acuña, H., “Re-evaluación del papel de la actividad neuronal asociada con el uso y de la densidad de inervación 
periférica en la plasticidad sensoriomodal de la neocorteza de la rata”, Maestría en ciencias (neurobiología), UNAM-INB, Créditos en: 
Tesis Maestría, 2001.

\item Gutiérrez de la Barrera, A., “Efectos del factor de crecimiento I tipo insulina en la muerte celular normal durante el desarrollo 
retiniano en roedores”, Maestría en ciencias (neurobiología), UNAMINB, Créditos en: Tesis Maestría, 2001.

\item Uribe-Querol, E., “Efecto directo de esteroides ováricos sobre la secreción de GnRH”, Maestría en ciencias (neurobiología), 
UNAM-INB, Créditos en: Tesis Maestría, 2000.

\item Alonso-Onofre, F., “Diferenciación Funcional entre la región dorsal y ventral del estriado en el aprendizaje visuoespecial”, 
UNAM-INB, Maestría en ciencias (neurobiología), Créditos en: Tesis Maestría, 1999.
\end{enumerate}

\textbf{Agradecimientos en tesis de licenciatura}

\begin{enumerate}
\item Bravo-Coutiño, D.A., “Evaluación inmunohistoquímica del efecto de la lesión de las conexiones gabaérgicas o colinérgicas del 
septum 
medial y su relación con cambios histológicos a nivel de hipocampo dorsal y fimbria", Licenciatura en neurociencias, ENES Juriquilla, 
2023, Créditos en: Tesis Licenciatura, diciembre de 2023.

\item López-García, J.,A., “Un modelo estocástico para la ecuación controlada de Schrödinger”, Licenciatura en Ingeniería Física, 
Universidad Autónoma Metropolitana, Unidad Azcapotzalco, 2023, Créditos en: Tesis Licenciatura, diciembre de 2023.

\item Larriva Sánchez, F., “Análisis de la carga de deleciones en el genoma mitocondrial del cerebro embrionario de ratón”, Facultad de 
Química, UAQ, Créditos en: Tesis Licenciatura, noviembre de 2019.

\item Ahumada Solórzano, Santiaga Maricela, "La hormona de crecimiento (cGH) en el aparato reproductor de pollos hembras", Créditos en: 
Tesis Licenciatura, 2007.

\item Rodríguez-Franco, Y., “Efecto del bloqueo e los receptores a glucocorticoides en el estriado de la rata sobre memoria”, Facultad 
de 
Química, UAQ, Querétaro, México, Créditos en: Tesis Licenciatura, 2005.
\end{enumerate}

\textbf{Agradecimientos en artículos internacionales}

\begin{enumerate}
\item García-Vilchis, B, Román-López, T., , Ramírez-González, D., López-Camaño, X.J., Medina-Rivera, A., Alcauter, S. and 
Ruiz-Contreras, 
A., “TwinsMX: Exploring the Genetic and Environmental Influences on Health Traits in the Mexican Population”, Twin Research and Human 
Genetics, 27, 85–96, Créditos en: Artículo internacional, noviembre de 2024.

\item Rasgado-Toledo, J., Siddharth Duvvadab, S., Shahc, A., Ingalhalikarc, M., Allurib, GarzaVillarreal, E., “Structural and 
functional 
pathology in cocaine use disorder with polysubstance use: A multimodal fusion approach structural-functional pathology in cocaine use 
disorder”, Progress in Neuropsychopharmacology and Biological Psychiatry, 1-7, Créditos en: Artículo internacional, noviembre de 2024.

\item Angeles-Valdez, D., Rasgado-toledo, J., Villicaña, V., Davalos Guzman, A., Almanza, C., FajardoValdez, A., Alcala-Lozano, R. and
Garza-Villarreal, E., “The Mexican dataset of a repetitive transcranial magnetic stimulation clinical trial on cocaine use disorder 
patients: SUDMEX tMS”, Scientific Data, Créditos en: Artículo internacional, noviembre de 2024.

\item Luna-Munguia, H., Gasca-Martinez, Garay-Cortes, Coutiño, D., Regalado, M. De los Rios, E., VillaseñorP., Hidalgo-Flores, 
Flores-Guapo, K., Brandon-Yair, B., Concha, L., “Selective Medial Septum Lesions in Healthy Rats Induce Longitudinal Changes in 
Microstructure of Limbic Regions, Behavioral Alterations, and Increased Susceptibility to Status Epilepticus”, Molecular Neurobiology 
(2024) 61:7898–7918, Créditos en: Artículo internacional, noviembre de 2024.

\item Vázquez, P. G. , Whitfield-Gabrieli, S., Bauer, C.C. and Barriosa, F. A., “Brain functional connectivity of hypnosis without 
target 
suggestion. An intrinsic hypnosis rs-fMRI study”, THE WORLD JOURNAL OF BIOLOGICAL PSYCHIATRY, VOL. 25, NO. 2, 95–105, Créditos en: 
Artículo internacional, noviembre de 2024.

\item Rodríguez-Vidal, LL., Alcauter, S., Barrios, F.A., “The functional connectivity of the human claustrum, according to the Human 
Connectome Project database”, PLOS ONE, Créditos en: Artículo internacional, abril de 2024.

\item Reyes-Pérez, P., Hernández-Ledesma, A.L., Román-López, T.V., García-Vilchis, B., Alcauter, S., “Building national patient 
registries 
in Mexico: insights from the MexOMICS Consortium”, frontiers, 1-14, Créditos en: Artículo internacional, junio de 2024.

\item Trujillo-Villarreal, Cruz-Carrillo, G., Angeles-Valdez, D., Garza-Villarreal, E.A., Camacho-Morales, A., “Paternal Prenatal and 
Lactation Exposure to a High-Calorie Diet Shapes Transgenerational Brain Macro- and Microstructure Defects, Impacting Anxiety-Like 
Behavior in Male Offspring Rats”, eNeuro, 1-22, Créditos en: Artículo internacional, febrero de 2024.

\item Garcia-Saldivar, P. , De León, C., Mendez Salcido, F.A., Concha, L., Hugo Merchant, H., “White matter structural bases for phase 
accuracy during tapping synchronization”, eLife, 1-27, Créditos en: Artículo internacional, noviembre de 2024.

\item Fajardo-Valdez, A., Camacho-Téllez. V., Rodríguez-Cruces, García-Gomar, Pasaye, E.H., Concha, L., Functional correlates of 
cognitive performance and working memory in temporal lobe epilepsy: Insights from task-based and resting-state fMRI”, PLOS ONE, 1-20, 
Créditos en: Artículo internacional, marzo de 2024.

\item Olalde-Mathieu, V.E., Licea-Haquet, G.,L., Alcauter, S., Atilano-Barbosa, D., Dominquez-Frausto, C., A., Angulo-Perkins, A., 
Barrios, F.A., “Empathy-related differences in the anterior cingulate functional connectivity of regular cannabis users when compared 
to controls”, Neuroscience Research, 1-9, 2023, Créditos en: Artículo internacional, diciembre de 2023.

\item Rasgado-Toledo, J., Siddharth-Duvvadab, S., Shahc, A. , Ingalhalikarc, M. , Allurib, V., GarzaVillarreal, E.A., “Structural and 
functional pathology in cocaine use disorder with polysubstance use: A multimodal fusion approach structural-functional pathology in 
cocaine use disorder”, Progress in Neuropsychopharmacology and Biological Psychiatry, 1-8, 2023, Créditos en: Artículo internacional, 
noviembre de 2023.

\item Cruz-Carrillo, G., Trujillo-Villarreal, L.A., Angeles-Valdez, D., Concha, L.,Eduardo Garza-Villarreal, E.A., Camacho-Morales, A., 
“Prenatal Cafeteria Diet Primes Anxiety-like Behavior Associated to Defects in Volume and Difusion in the Fimbria-fornix of Mice 
Ospring”, NEUROSCIENCE, 70-85, 2023, Créditos en: Artículo internacional, noviembre de 2023.

\item Román-López, T. V., García-Vilchis, B. , Murillo-Lechuga, V. , Chiu-Han, E., López-Camaño, X. , Aldana-Assad, O. , Diaz-Torres, 
S., 
5 , Caballero-Sánchez, U., Ivett Ortega-Mora, Ramírez-González, D., Zenteno, D. , Espinosa-Valdés, Z., Tapia-Atilano, A. , 
Pradel-Jiménez, S., Rentería, M.E., MedinaRivera, A., Ruiz-Contreras, A.E., Alcauter, S., “Estimating the Genetic Contribution to 
Astigmatism and Myopia in the Mexican Population”, CAMBRIDGE, University Press, 1-9, 2023, Créditos en: Artículo internacional, 
noviembre de 2023.

\item Villaseñor, P. J., Cortés-Servín, D., Pérez-Moriel, A., Aquiles, A., Luna-Munguía, H., RamirezManzanares, A., Coronado-Leija, R., 
Larriva-Sahd, J., 1 and Concha, L., “Multi-tensor diffusion abnormalities of gray matter in an animal model of cortical dysplasia”, 
frontiers, 1-10, 2023, Créditos en: Artículo internacional, noviembre de 2023.

\item Vázquez, P.G., Whitfield-Gabrieli, S., Bauer, C.C.C. and Barrios, F.A., “Brain functional connectivity of hypnosis without target 
suggestion. An intrinsic hypnosis rs-fMRI study”, The World Journal of Biological Psychiatry, 1-12, 2023, Créditos en: Artículo 
internacional, noviembre de 2023.

\item Gracia-Tabuenca, Z., Díaz-Patiño, J.C., Arelio-Ríos,I., Moreno-García, M.B., Barrios, F.A. and Alcauter, S., “Development of the 
Functional Connectome Topology in Adolescence: Evidence from Topological Data Analysis”, eNeuro, 1-10, 2023, Créditos en: Artículo 
internacional, noviembre de 2023.

\item Atilano-Barbosa, D. and Barrios, F.A., “Brain morphological variability between Whites and African Americans: the importance of 
racial identity in brain imaging research”, frontiers, 1-15, 2023, Créditos en: Artículo internacional, diciembre de 2023.

\item Llamas-Alonso, L.A., Barrios, F.A., González-Garrido A., Ramos-Loyo J., “ Emotional faces interfere with saccadic inhibition and 
attention re-orientation: An fMRI study”, Neuropsychologia, 1-9, Créditos en: Artículo internacional, diciembre de 2022.

\item Rodríguez-Nieto, G., Mercadillo, R.E., Pasaye, E.H., and Fernando A. Barrios, F.A., “Affective and cognitive brain-networks are 
differently integrated in women and men while experiencing compassion”, Frontiers in Psychology, 1-12, Créditos en: Artículo 
internacional, diciembre de 2022.

\item Olalde-Mathieu, V.E., Sassi, F., Reyes-Aguilar, A., Mercadillo, R.E., Alcauter, S., Barrios., A.F., “Greater 
Empathic Abilities and Resting State Brain Connectivity Differences in Psychotherapists Compared to Non-psychotherapists”, 
Neuroscience, 1-10, Créditos en: Artículo internacional, diciembre de 2022.

\item Olalde-Mathieu, V.E., Licea-Haquet, G., Reyes-Aguilar, A. and Barrios, F.A., “Psychometric properties of the Emotion Regulation 
Questionnaire in a Mexican sample and their correlation with empathy and alexithymia”, Cogent Psychology, 1-11, Créditos en: Artículo 
internacional, diciembre de 2022.

\item Aguilar-Moreno, A., Ortiz, J., Concha, L., Alcauter, S., Paredes, R.G., “Brain circuits activated by female sexual behavior 
evaluated by manganese enhanced magnetic resonance imaging”, PLOS ONE, 1-24, Créditos en: Artículo internacional, agosto de 2022.

\item Domínguez-Arriola, M. E., Olalde-Mathieu, V. E., Garza-Villarreal, E.A., Barrios, F.A, “The Dorsolateral Prefrontal Cortex 
Presents 
Structural Variations Associated with Empathy and Emotion Regulation in Psychotherapists”, Brain Topography, 1-14, 21, Créditos en: 
Artículo internacional, agosto de 2022.

\item Hevia-Orozco, J., Azalea Reyes-Aguilar, A., Pasaye, E.H., Barrios, F.A, “Participation of visual association areas in social 
processing emerges when rTPJ is inhibited”, eNeurologicalSci 27, 1-6, Créditos en: Artículo internacional, diciembre de 2022.

\item Lizcano-Cortés, F., Rasgado-Toledo, J., Giudicessi, A. and Giordano, M., “Theory of Mind and Its Elusive Structural Substrate”, 
frontiers in Human Neuroscience, p1-9, Vol. 15, Créditos en: Artículo internacional, noviembre de 2021.

\item Rasgado-Toledo, J., Lizcano-Cortés, F., Víctor Olalde-Mathieu, V.E., Licea-Haquet, G., Zamora-Ursulo, M.A., Giordano, M. and 
Reyes-Aguilar, A., “A Dataset to Study Pragmatic Language and Its Underlying Cognitive Processes”, frontiers in Human Neuroscience, 
p1-7, Vol. 15, Créditos en: Artículo internacional, noviembre de 2021.

\item Martínez, A.Y., Demertzi, A., Bauer, C.C.C., Gracia-Tabuenca, Z., Alcauter, S., and Barrios, F.A., “The Time Varying Networks of 
the Interoceptive Attention and Rest”, eNeuro, p1-13, Créditos en: Artículo internacional, noviembre de 2021.

\item Domínguez-Arriola, M.E., Víctor E. Olalde-Mathieu, V.E., Garza-Villareal, E.A., Barrios, F.A., “The Dorsolateral Prefrontal 
Cortex Presents Structural Variations Associated with Empathic Capacity in Psychotherapists“, bioRxiv, p1-19, Créditos en: Artículo 
internacional, noviembre de 2021.

\item Lazcano, I., Cisneros-Mejorado, A., Concha, L., José Ortiz-Retana, J.J., Garza-Villarreal, E.A. and Orozco, A., “MRI and 
histologically derived neuroanatomical atlas of the Ambystoma mexicanum (axolotl)”, Scientific Reports, p1-13, Créditos en: Artículo 
internacional, noviembre de 2021.

\item Luna-Munguia, H., Marquez-Bravo, L., Concha, L., “Longitudinal changes in gray and white matter microstructure during 
epileptogenesis in pilocarpine-induced epileptic rats”, Seizure: European Journal of Epilepsy, 90, p130-140., Créditos en: Artículo 
internacional, noviembre de 2021.

\item Garcia-Saldivar, P., Garimella, A., Garza-Villarreal, E.A., Mendez, F.A., Concha, L., Merchant, H., “PREEMACS: Pipeline for 
preprocessing and extraction of the macaque brain surface”, NeuroImage, 227, p1-16., Créditos en: Artículo internacional, noviembre de 
2021.

\item López-Gutiérrez, M.F., Gracia-Tabuenca, Z., Ortiz, J.J., Camacho, F.J., Young, L.J., Paredes, R.G., Díaz, N.F., Portillo, W., 
Alcauter, S., “Brain functional networks associated with social bonding in monogamous voles”, eLife, p1-25, Créditos en: Artículo 
internacional, noviembre de 2021.

\item Gracia-Tabuenca, Z., Moreno, M.B., Barrios, F.A., Alcauter, S., “Development of the brain functional connectome follows 
puberty-dependent nonlinear trajectories”, NeuroImage, 229, p1-9, Créditos en: Artículo internacional, noviembre de 2021.

\item Rodríguez-Cruces, R., Bernhardt, B. C., Concha, L., “Multidimensional associations between cognition and connectome organization 
in 
temporal lobe epilepsy”, NeuroImage (213)1-11, Créditos en: Artículo internacional, marzo de 2020.

\item Niño, S. A., Erika Chi-Ahumada, E., Ortíz, J., Zarazua, S., Concha, L., Jiménez-Capdeville, M.E., “Demyelination associated with 
chronic arsenic exposure in Wistar rats”, Toxicology and Applied Pharmacology, 
(393)1-9, 2020., Créditos en: Artículo internacional, marzo de 2020.

\item García-Gomar, M.G., Concha, L., Soto-Abraham, J., Tournier, J. D., Aguado-Carrillo, G., VelascoCampos, F., “Long-Term Improvement 
of Parkinson Disease Motor Symptoms Derived From Lesions of Prelemniscal Fiber Tract Components”, OPERATIVE NEUROSURGERY, Vol. 0, Num. 
0, 1-12, Créditos en: Artículo internacional, abril de 2020.

\item Gracia-Tabuenca, Z., Juan Carlos Díaz-Patiño, J.C., Arelio, I.,2 and Alcauter, S., “Topological Data Analysis Reveals Robust 
Alterations in the Whole-Brain and Frontal Lobe Functional Connectomes in Attention-Deficit/Hyperactivity Disorder”, March-April, 7(2) 
ENEURO, 1–10, Créditos en: Artículo internacional, marzo de 2020.

\item Hatton, S. N., Huynh, K. H., Bonilha, L., Abela, E., Alhusaini, S., Altmann, A., Alvim, M. K. M., Akshara R. Balachandra, A. R., 
Concha, L., “White matter abnormalities across different epilepsy syndromes in adults: an ENIGMA-Epilepsy study”, BRAIN, A Journal of 
Neurology, 1-20, Créditos en: Artículo internacional, abril de 2020.

\item Rojas-Vite, G., Coronado-Leija, R., Narvaez-Delgado, O., Ramírez-Manzanares, A., Marroquín, J.L., Noguez-Imm, R., Aranda, M.L, 
Sherrer, B., Larriva-Sahd, J., Concha, L., “Histological validation of per-bundle water diffusion metrics within a region of fiber 
crossing following axonal degeneration”, NeuroImage, 201, Créditos en: Artículo internacional, noviembre de 2019.

\item Angulo-Perkins, A., Concha, L., “Discerning the functional networks behind processing of music and speech through human 
vocalizations”, PLOS-ONE, Créditos en: Artículo internacional, octubre de 2019.

\item Narvaez-Delgado, O., Rojas-Vite, G., Coronado-Leija, R., Ramírez-Manzanares, A., Marroquín, J.L., Noguez-Imm, R., Aranda, M.L, 
Sherrer, B., Larriva-Sahd, J., Concha, L., “Histological and diffusion-weighted magnetic resonance imaging data from normal and 
degenerated optic nerve and chiasm of the rat”, Data in brief, 26, 1-19, Créditos en: Artículo internacional, noviembre de 2019.

\item Manno, F.A.M, Raul R. Cruces,R.R., Lau, C. and Barrios, F.A., “Uncertain Emotion Discrimination Differences Between Musicians and 
Non-musicians Is Determined by Fine Structure Association: Hilbert Transform Psychophysics”, frontiers in Neuroscience,1-13, Créditos 
en: Artículo internacional, noviembre de 2019.

\item Rodríguez-Cruces, R., Bernhardt, B.C., Luis Concha, L., “MULTIDIMENSIONAL ASSOCIATIONS BETWEEN COGNITION AND CONNECTOME 
ORGANIZATION IN TEMPORAL LOBE EPILEPSY”, COGNITIVE CONNECTOMICS IN TLE, bioRxiv, June, 1-25, Créditos en: Artículo internacional, 
noviembre de 2019.

\item CCC, Bauer, Whitfield-Gabrieli S, Díaz JL, Pasaye EH, Barrios FA, “From state-to-trait meditation: Reconfiguration of central 
executive and default mode networks”, eNeuro, 1-20, Créditos en: Artículo internacional, noviembre de 2019.

\item Hernández-Abrego, A., Vázquez-Gómez, E., García-Colunga, J., "Effects of the antidepressant mirtazapine and zinc on nicotinic 
acetylcholinereceptors", Neuroscience Letters 665, 1-6, 2018. , Créditos en: Artículo internacional, noviembre de 2018.

\item Mercadillo, R.E., Alcauterc, S., Barrios, F.A., "Effects of primatological training on anthropomorphic valuations of emotions", 
IBRO Reports 5, 54–59, 2018., Créditos en: Artículo internacional, noviembre de 2018.

\item Ortiz, J.J., Portillo, W., Paredes, R.G., Young, L.J, Alcauter, S., "Resting state brain networks in theprairie vole", SCiENTifiC 
REPOrTS, 8:1-11, 2018., Créditos en: Artículo internacional, noviembre de 2018.

\item Gracia-Tabuenca, Z., Moreno, M.B., Barrios, F.A., Alcauter, S., "Hemispheric asymmetry and homotopy of resting state 
functionalconnectivity correlate with visuospatial abilities in school-age children", NeuroImage 174, 441–448, 2018. , Créditos en: 
Artículo internacional, noviembre de 2018.

\item Rodríguez-Cruces, R, Velázquez-Pérez, L., Rodríguez-Leyva, I., Velasco, A.L.,Trejo-Martínez, D., Barragán-Campos, H.M., 
Camacho-Téllez, V., Concha, L., "Association of white matter diffusion characteristics and cognitivedeficits in temporal lobe 
epilepsy", Epilepsy and Behavior 79, 138–145, 2018., Créditos en: Artículo internacional, noviembre de 2018.

\item Sarael Alcauter, Liliana García-Mondragóna, Zeus Gracia-Tabuenca, Martha B. Moreno, Juan J.Ortiza, Fernando A. Barrios, “Resting 
state functional connectivity of the anterior striatum andprefrontal cortex predicts reading performance in school-age children”, Brain 
and Language, 174(94-102), Créditos en: Artículo internacional, junio de 2017.

\item Reyes-Aguilar, A., Fernández-Ruiz, J., Pasaye, E.H., Barrios, F. A., “Executive Mechanisms forThinking about Negative Situations 
in 
both cooperative and non-cooperative contexts”, Frontiersin Human Neuroscience, 11(1-5), Créditos en: Artículo internacional, mayo de 
2017.

\item Angulo-Perkins A., William Aubé W., Peretz I., Barrios F.A. , Armony J.L. and Concha L., “Music listening engages specific 
cortical 
regions within the temporal lobes: Differences between musicians and non-musicians”, CORTEX, 59, 126-137., Créditos en: Artículo 
internacional, septiembre de 2014.

\item Bauer C.C.C., Díaz J.L, Concha L., Barrios F.A., “Sustained attention to spontaneous thumb sensations activates brain 
somatosensory 
and other proprioceptive areas”, Brain and Cognition 87, 86–96., Créditos en: Artículo internacional, septiembre de 2014.

\item Martínez-Soto J., Montero M.E., Córdova A., “Restauración psicológica y naturaleza urbana: algunas implicaciones para la salud 
mental”, Salud Mental, 37, 217-224, Créditos en: Artículo internacional, mayo de 2014.

\item Donnet S., Bartolo R., Fernández J.M., Silva Cunha J.P., Prado L., and Merchant H., “Monkeys time their pauses of movement and 
not 
their movement-kinematics during a synchronizationcontinuation rhythmic task", J Neurophysiol 111: 2138–2149., Créditos en: Artículo 
internacional, septiembre de 2014.

\item Sánchez-Resendis O., Medina A.C., Serafín N., Prado-Alcalá R.A., Roozendaal B., Quirarte G.L., “Glucocorticoid-cholinergic 
interactions in the dorsal striatum in memory consolidation of inhibitory avoidance traning”, frontiers in Behavioral Neuroscience, 
Vol. 6, 1-6., Créditos en: Artículo internacional, noviembre de 2012.

\item Mercadillo R.E., Trujillo C., Sánchez-Cortazar J., Barrios F.A., “Brain activity in adhd patients performing the counting stroop 
task: A social neuroscience approach”, Phychological Reports: Disability and Trauma, 111,2,652-668., Créditos en: Artículo 
internacional, septiembre de 2012.

\item Alcauter, S, Barrios, F. A., Díaz, R, Fernández-Ruiz, J., "Gray and white matter alterations in spinocerebellar ataxia type 7: An 
in vivo DTI andVBM study", NeuroImage 55 1–7, Créditos en: Artículo internacional, 2011.

\item Mercadillo, R.E., Díaz, J. L., Pasaye, E. H., Barrios, F. A., "Perception of suffering and compassion experience: Brain gender 
disparities", Brain and Cognition 76, 5–14., Créditos en: Artículo internacional, noviembre de 2011.

\item Castillo C.G, Mendoza S., Saavedra J., Giordano M., “Lack of effect of intranigral transplants of a GABAergic cell line on 
absence 
seizures”, Epilepsy and Behavior., Créditos en: Artículo internacional, noviembre de 2010.

\item PASAYE E.H, GUTIÉRREZ R.A., S. ALCAUTER S., MERCADILLO R.E., AGUILAR-CASTAÑEDA E, DE ITURBE M., ROMERO-ROMO J., BARRIOS F.A., 
“Event-Related Functional Magnetic Resonance Images during the Perception of Phantom Limb. A Brushing Task”, The Neuroradiology Journal 
23: 665-670., Créditos en: Artículo internacional, noviembre de 2010.

\item ROMERO-ROMO J.I, BAUER C.C.C, PASAYE E.H., GUTIÉRREZ R.A., FAVILA R., BARRIOS F.A., “Abnormal Functioning of the Thalamocortical 
System Underlies the Conscious Awareness of the Phantom Limb Phenomenon”, The Neuroradiology Journal 23: 671-679., Créditos en: 
Artículo internacional, noviembre de 2010.

\item Mercadillo, RE, Díaz, JL, Pasaye, EH, Barrios, FA, "Gender differences in brain activation during induced compassion 
experiences", 
Neuropsychologia, manuscript number: NSY-D-10-00001, Créditos en: Artículo internacional, enero de 2010.

\item Morales, T., Ahuilar, L., Ramos, E., Mena, F., Morgan, C., "Fos Expression induced by milk ingestion in the caudal brainstem of 
neonatal rats", Brain Reseach, 1241, 76-83 , Créditos en: Artículo internacional, 2008.

\item Anguiano, B, García-Solís, P, Delgado, G, and, Aceves, CA, "Uptake and Gene Expression with Antitumoral Doses of Iodine in 
Thyroid 
and Mammary Gland: Evidence That Chronic Administration Has No Harmful Effects", THYROID, 17(9): 851-859, Créditos en: Artículo 
internacional, 2007.

\item Herrera-Alarcón, J, Villagómez-Amezcua, E, González-Padilla E, Jiménez-Severiano, H, "Stereological study of postnatal testicular 
development in Blackbelly sheep", Theriogenology, 582591, Créditos en: Artículo internacional, 2007.

\item Medina, AC, Charles, JR, Espinoza-González, V, Sánchez-Resendis, O, Prado-Alcalá, RA, Roozendaal, B, and Quirarte, GL, 
"Glucocorticoid administration into the dorsal stratium facilitates memory consolidation of inhibitory avoidance training but not of 
the context or footshock components", Learning and Memory, 673-677, Créditos en: Artículo internacional, mayo de 2007.

\item Castillo, C.G., Mendoza, S., Freed., W.J., Giordano, M., "Intranigral transplants of immortalized GABAergic cells decrease the 
expression of kainic acid-induced seizures in the rat", Behavioural Brain Research, 171, 109-115, Créditos en: Artículo internacional, 
2006.

\item Zugasti-Cruz, A., Maillo, M., López-Vera, E., Falcón, A., Heimer de la Cotera, E.P., Olivera, B., M., Aguilar, M.B., Amino acid 
sequence and biological activity of a gama-conotoxin-like peptide from the worm-hunting snail, Peptides 27, 506-611, Créditos en: 
Artículo internacional, 2006.

\item Frias, C., Torrero, C., Regalado, M., and Salas., M., Organization of olfactory glomeruli in neonatally undernourished rats, 
Nutritional Neuroscience, 1-7, Créditos en: Artículo internacional, 2006.

\item Aguilar, M. B. , Lopez-Vera, E., Imperial, J. S., Falcon, A., Olivera, B. M., De la Cotera, E. P., “Putative gamma-conotoxins in 
vermivorous cone snails: the case of Conus delessertii”, Peptides. 26:23-7, Créditos en: Artículo internacional, mayo de 2005.

\item Luna, M., Barraza, N., Berumen, L., Carranza, M., Pedernera, E., Harvey, S., Arámburo, C., “Heterogeneity of growth hormone 
immunoreactivity in lymphoid tissues and changes during ontogeny in domestic fowl”, Gen. Comp. Endocrinol., 144:28-37, Créditos en: 
Artículo internacional, mayo de 2005.

\item Aguilar, M. B. , Lopez-Vera, E., Ortiz, E., Becerril, B., Possani, L., D., Olivera, B. M., De la Cotera, E. P., “A novel 
conotoxin from conos delessertii with posttranslationally modified lisien residues”, Biochemistry, 44:11130-11136, 
Créditos en: Artículo internacional, mayo de 2005.

\item Cepeda-Nieto, A. C., Pfaff, S. L., Varela-Echavarría, A., “Homeodomain transcription factors in the development of subsets of 
hindbrain reticulospinal neurons”, Mol. Cell Neurosci., 28:30-41, Créditos en: Artículo internacional, mayo de 2005.

\item García-Solís, P, Alfaro, Y, Anguiano, B, Delgado G, Guzman, R. C. , Nandi, S., Díaz-Muñoz, M., Vazquez-Martinez, O., Aceves, C., 
“Inhibition of N-methyl-N-nitrosourea-induced mammary carcinogenesis by molecular iodine (I2) but not by iodide (I-) treatment Evidence 
that I2 prevents cancer promotion”, Mol. Cell Endocrinol., 236:49-57, Créditos en: Artículo internacional, mayo de 2005.

\item Arroyo-Helguera, O., Mejia-Viggiano, C., Varela-Echavarria, A., Cajero-Juárez, M., Aceves, C., “Regulatory role of the 3I 
untranslated region (3’UTR) of rat 5I deiodinase (D1). Effects on Messenger RNA translation and stability”, Endocrine, 27:219-227, 
Créditos en: Artículo internacional, mayo de 2005.

\item Granados-Rojas, L., Aguilar, A., Díaz-Cintra, S., “The mossy fiber system of the hippocampal formation is decreased by chronic 
and 
postnatal but not by prenatal protein malnutrition in rats”, Nutr. Neurosci., 7:301-8, Créditos en: Artículo internacional, mayo de 
2004.

\item Peréz-León, J.A., Sarabia, G., Miledi, R, Garcia-Alcocer, G., “Distribution of 5hydroxytriptamine2C receptor RNA in rat retina”, 
Brain Res Mol Brain Res. 125:140-2, Créditos en: Artículo internacional, mayo de 2004.

\item Anguiano, B., Rojas-Huidobro, R., Delgado, G., Aceves, C., “Has the mammary gland a protective mechanism against overexposure to 
triiodothyronine during the peripartum period? The prolactin pulse down-regulates mammary type I deiodinase responsiveness to 
norepinephrine”, J. Endocrinol., 183:267-77, Créditos en: Artículo internacional, mayo de 2004.

\item Díaz del Guante, M.A., Rivas, M., Prado-Alcalá, R. A., Quirarte, G. L., Amnesia produced by pretraining infusion of serotonin 
into 
the substantia nigra, SYNAPTIC TRANSMISSION NEUROREPORT, Vol. 15, No. 15, 2527-2529, Créditos en: Artículo internacional, 2004.

\item Luna, M., Huerta, L, Berumen, L, Martínez-Coria, H., Harvey, S., Arámburo, C., “Growth hormone in the male reproductive tract of 
the chicken: heterogeneity and changes during ontogeny and maturation”, Gen. Comp. Endocrinol. 137:37-49, Créditos en: Artículo 
internacional, mayo de 2004.

\item Díaz-Cintra, S, Yong A, Aguilar A, Bi, X., Lynch, G., Ribak, C.E., “Ultrastructural analysis of hippocampal piramidal neurons 
from 
apolipoprotein E-deficient mice treated with a cathepsin inhibitor”, J. Neurocytol. 33:37-48, Créditos en: Artículo internacional, mayo 
de 2004.

\item Mena-Segovia, J., Favila, R., Giordano, M., “Long-term effects of striatal lesions on c-Fos immunoreactivity in the 
pedunculopontine nucleus”, Eur. J. Neurosci., 20:2367-76, Créditos en: Artículo internacional, mayo de 2004.

\item López-Vera, E., de la Cotera, E. P., Maillo, M., Riesgo-Escovar, J., Olivera, B. M., Aguilar, M. B., “A novel structural class of 
toxins: the methionine-rich peptides from the venoms of turrid marine snails (Mollusca, Conoidea)”, Toxicon., 43:365-74, Créditos en: 
Artículo internacional, mayo de 2004.

\item Berumen, L. C., Luna, M., Carranza, M., Martinez-Coria, H., Reyes, M., Carabez, A., Arámburo, C., “Chicken growth hormone: 
further 
characterization and ontogenic changes of an N-glycosylated isoform in the anterior pituitary gland”, Gen. Comp. Endocrinol., 
139:113-23, Créditos en: Artículo internacional, mayo de 2004.

\item Mena-Segovia, J., Giordano, M., “Striatal dopaminergic stimulation produces c-Fos expression in the PPT and an increase in 
wakefulness”, Brain Res., 986:30-8, Créditos en: Artículo internacional, mayo de 2003.

\item Prado-Alcalá, R. A., Solana-Figueroa, R., Galindo, L. E., Medina, AC, Quitarte, G. L., “Blockade of striatal 5-HT2 receptors 
produces retrograde amnesia in rats”, Life Sci., 74:481-8, Créditos en: Artículo internacional, mayo de 2003.

\item Vázquez-Martínez, O., Cañedo-Merino, R., Díaz-Muñoz, M., Riesgo-Escovar, “Biochemical characterization, distribution and 
phylogenetic analysis of Drosophila melanogaster ryanodine and IP3 receptors, and thapsigargin-sensitive Ca2+ ATPase”, J. Cell Sci., 
116:2483-94, Créditos en: Artículo internacional, mayo de 2003.

\item Prado-Alcalá, R. A., Ruiloba, M. I., Rubio, L., Solana-Figueroa, R., Medina, C., Salado-Castillo, R., Quitarte, G. L., “Regional 
infusions of serotonin into the striatum and memory consolidation”, Synapse, 47:169-75, Créditos en: Artículo internacional, mayo de 
2003.

\item Hernádez-Montiel, H. L., Meléndez-Herrera, E., Cepeda-Nieto, A. C., Mejía-Viggiano, C., LarrivaSahd, J., Guthrie, S., 
Varela-Echavarría, A.”Diffusible signals and fasciculated growth in reticulospinal axon pathfinding in the hindbrain”, Dev. Biol., 
255:99-112, Créditos en: Artículo internacional, mayo de 2003.

\item Mena-Segovia, J., Cintra, L., Prospéro-Garcia, O., Giordano, M. “Changes in sleep-waking cycle after striatal excitotoxic 
lesions”, 
Behav. Brain Res., 136:475-81, Créditos en: Artículo internacional, mayo de 2002.

\item Marina, N., Morales, T., Díaz, N., Mena, F., “Suckling-induced activation of neural c-fos expression at lower thoracic rat spinal 
cord segments”, Brain Res., 954:100-14, Créditos en: Artículo internacional, mayo de 2002.

\item Larriva-Sahd, J., Condés Lara, M., Martínez-Cabrera, G., Varela-Echavarría, A., “Histological and ultrastructural 
characterization 
of interfascicular neurons in the rat anterior commissure”, Brain Res., 931:81-91, Créditos en: Artículo internacional, mayo de 2002.

\item Martínez-Coria, H., López-Rosales, L. J., Carranza, M., Berumen, L., Luna, M., Arámburo, C., “Differential secretion of chicken 
growth hormone variants after growth hormone-releasing hormone stimulation in Vitro”, Endocrine, 17:91-102, Créditos en: Artículo 
internacional, mayo de 2002.

\item Solana-Figueroa, R., Salado-Castillo, R., Galindo, L. E., Quitarte, G. L., Prado-Alcalá, R. A., “Effects of pretraining 
intrastriatal administration of p-chloroamphetamine on inhibitory avoidance”, Neurobiol Learn Mem., 78:178-85, Créditos en: Artículo 
internacional, mayo de 2002.

\item Díaz, N., Huerta, I., Marina, N., Navarro, N., Mena, F., “Regional mechanisms within anterior pituitary of lactating rats may 
regulate prolactin secretion”, Endocrine, 18:41-6, Créditos en: Artículo internacional, mayo de 2002.

\item Martinez, I., Quitarte, G. L., Díaz-Cintra, S., Quiroz, C., Prado-Alcala, R. A., “Effects of lesions of hippocampal fields CA1 
and 
CA3 on acquisition of inhibitory avoidance”, Neuropsychobiology, 46:97103, Créditos en: Artículo internacional, mayo de 2002.

\item Giordano, M., Mejía-Viggiano, M. C., “Gender differences in spontaneous and MK-801-induced activity alter striatal lesions”, 
Brain 
Res. Bull., 56:553-61, Créditos en: Artículo internacional, mayo de 2001.

\item Fernández-Bouzas, A., Harmony, T., Fernández, T., Silva-Pereyra, J., Valdés, P., Bosch, J., Aubert, E., Casian, G., Otero Ojeda, 
G., Ricardo, J., Hernandez-Ballesteros, A., Santiago, E., “Sources of abnormal EEG activity in brain infarctions”, Clin. 
Electroencephalogr., 31:165-9, Créditos en: Artículo internacional, mayo de 2000.

\item Arámburo, C., Luna, M., Carranza, M., Reyes, M., Martínez-Coria, H., Scanes, C. G., “Growth hormone size variants: changes in the 
pituitary during development of the chicken”, Proc. Soc. Exp. Biol. Med., 223:67-74, Créditos en: Artículo internacional, mayo de 2000.

\item Aguilar, M.B., Lezama-Monfil, L., Maillo, M., Pedraza-Lara, H., López-Vera E., Heimer de la Cotera, E.P., "A biologically active 
hydrophobic T-1-conotoxin from the venom of Conus spurius", Peptides, 27, 500-505, Créditos en: Artículo internacional, 2000.

\item García-Colunga, J., Valdiosera, R., García, U., “P-type Ca2+ current in crayfish peptidergic neurones”, J. Exp, Biol., 
202:429-440, Créditos en: Artículo internacional, mayo de 1999.

\item Dueñas, Z., Torner, L., Corbacho, A. M., Ochoa, A., Gutiérrez-Ospina, G., López-Barrera, F., Barrios, F. A., Berger, P., Martínez 
de la Escalera, G., Clapp, C., “Inhibition of rat corneal angiogenesis by 16-kDa prolactin and by Endogenous prolactin-like molecules”, 
Invest. Ophthalmol. Vis. Sci., 40:2498-505, Créditos en: Artículo internacional, mayo de 1999.

\item Gutiérrez-Ospina, G., Jiménez-Trejo, F. J., Favila, R., Moreno-Mendoza, N. A., Granados Rojas, Barrios, F. A., Diaz-Cintra, S., 
Merchant-Larios, H., “Acetylcholinesterase-positive innervation is present at undifferentiated stages of the sea turtle Lepidochelis 
olivacea embryo gonads: implications for temperature-dependent sex determination”, J. Comp. Neurol., 410:90-8, Créditos en: Artículo 
internacional, mayo de 1999.

\item Gutiérrez-Ospina, G., Díaz-Cintra, S., Aguirre-Portilla, A., Aguilar-Vázquez, A., López, S. R., Barrios, F.A., “Comparable 
activity levels in developmentally deprived and non-deprived layer IV cortical columns of the adult rat primary somatosensory cortex”, 
Neurosci. Lett., 247:5-8, Créditos en: Artículo internacional, mayo de 1998.

\item Giordano, M., Salado-Castillo, R., Sánchez-Alvarez, M., Prado-Alcalá, R. A., “Striatal transplants prevent AF64A-induced 
retention 
deficits”, Life Sci., 63:1953-61, Créditos en: Artículo internacional, mayo de 1998.
\end{enumerate}

\textbf{Agradecimientos en capítulos en libros}

\hfill

1. Merchant, H, Zarco, W, Prado, L, Pérez, O, "Behavioral and Neurophysiological Aspects of Target Interception", Editor: Dagmar 
Sternad, Springer, Créditos en: Capítulo de libro nacional, 2009.

\hfill

\textbf{Agradecimientos en congresos nacionales}

\hfill

\begin{enumerate}
\item Cisneros-Mejorado, A., Aguilera, P., Peralta-Arrieta, I., Gega-Miranda, A., Hernández-Cruz, A., Alquisiras-Burgos, I., “Klotho 
Anti-Aging Protein Deficiency Impact on Cerebral Cytoarchitecture”, V Congreso Nacional de Neurobiología y de la Sociedad Mexicana de 
Bioquímica, Morelia, Michoacán, del 13 al 17, Créditos en: Congreso nacional, marzo de 2024.

\item Cisneros-Mejorado, A., Vétez Uriza, F., Pablo Ordaz, R., Cárdenas-Pérez, G., Garay, E., Arellano, R. O., “B-carboline B-CCB, a 
potent and specific potentiator of GABA, receptor expressed in oligodendrocytes, promote remyelination in 
an in vivo demyelination model”, V Congreso Nacional de Neurobiología y de la Sociedad Mexicana de Bioquímica, Morelia, Michoacán, del 
13 al 17, Créditos en: Congreso nacional, abril de 2024.

\item Cisneros-Mejorado, A., Garay, E., Arellano, R.O., "beta-carbolines in focal and systemic demyelination-remyelination models", 4TH 
Symposium on Physiology and Pathology of Neuroglia, Créditos en: Congreso nacional, octubre de 2022.

\item Angulo Perkins A. A., López Carrera E., Barrios Álvarez F. A., Concha Loyola L., “EL LADO OCULTO DE LA IMAGEN POR RESONANCIA 
MAGNÉTICA: TÉCNICAS DE ANÁLISIS EN UN ESTUDIO DE MÚSICA Y MÚSICOS”, XIX Jornadas Académicas del Instituto de Neurobiología, UNAM, 
Septiembre 17-21. , Créditos en: Congreso nacional, septiembre de 2012.

\item Concha L., Angulo-Perkins A., Peretz A., Armony J., Barrios F.A., "Estímulos musicales provocan activaciones preferentes en el 
lóbulo temporal: Identificación mediante Resonancia Magnética Funcional", XVIII Jornadas Académicas del Instituto de 
Neurobiología,UNAM, Septiembre 19-23., Créditos en: Congreso nacional, septiembre de 2011.

\item Léon-Jacinto U., Quirarte, G.L., Aguilar-Vázquez A.R., Serafín N., Beltrán-Campos, V., PradoAlcalá, R.A., Díaz-Miranda, 
S.Y.,"Cambios en el tipo de espinas dendrítis basales en el CA3 del hipocampo asociados al aprendizaje espacial incrementado",XLIX 
Congreso Nacional de Ciencias Fisiologicas; Querétaro, Qro., Créditos en: Congreso nacional, 2006.

\item Beltrán-Campos, V., Quirarte, G.L., Ramírez-Amaya, V.,Aguilar-Vázquez, A.R., Serafín, N., PradoAlcalá, R.A., Díaz-Miranda, S.Y., 
Densidad de las Espinas dendríticas en las células piramidales del CA1 del hipocampo después del aprendizaje espacial en ratas 
ovariectomizadas, XLIX Congreso Nacional de Ciencias Fisiológicas; Querétaro, Qro. , Créditos en: Congreso nacional, 2006.

\item Aguilar Vazquez, A.R., Granados Rojas, L., Díaz Cintra, S.Y., Efecto de la malnutrición crónica sobre la densidad de las celulas 
gabaérgicas del giro dentado, XLVII Congreso Nacional de Ciencias Fisiológicas, Boca del Río, Veracruz, Créditos en: Congreso nacional, 
2004.

\item C.R., Díaz del Guante, M.A., Garín-Aguilar, M.E., Quirare, G.L., Prado-Alcalá, E.A., Efecto de la inactivación del hipocampo 
dorsal 
(HD) con TTX sobre la retención de la memoria evocada de dos diferentes niveles de reforzamiento, XLVII Congreso Nacional de Ciencias 
Fisiológicas, Boca del Rio, Veracruz, Créditos en: Congreso nacional, 2004.

\item Martínez, M.I., Quirarte, G.L., y Prado-Alcala, R.A.,Efecto de la lesión bilateral del hipocampo ventral sobre el aprendizaje, 
XLVI 
Congreso Nacional de Ciencias Fisiólogicas, Universdad Autónoma de Aguascalentes, Créditos en: Congreso nacional, 2003.
\end{enumerate}

\textbf{Agradecimientos en congresos internacionales}

\begin{enumerate}
\item Ocampo-Ruiz, A.L., Vazquez-Carrillo, D.,I., Dena-Beltrán, J., L., Cárdenas, A.G., Arellano, R., Cisneros-Mejorado, A., 
Dimas-Rufino, 
MA., Castillo, X., Garay, Martínez de la Escalera, E., G., Clapp, C., and Macotela, Y., “PROLACTIN RECEPTOR DEFICIENCY PROMOTES 
HYPOMYELINATION DURING CENTRAL NERVOUS SYSTEM MATURATION OF SUCKLING AND PREPUBERTAL MICE”, Seventh Biennial Meeting of the North 
American Society for Comparative Endocrinology, MAY.28-June.1, 2023, Créditos en: Congreso internacional, diciembre de 2023.

\item Gracia-Tabuenca, Z., Días-Patiño, J.C., Arelio, I., Alcauter, S., “Topological abnormal load in ADHD functional brain network”, 
Chicago, IL, NEUROSCIENCE, 19-23, Créditos en: Congreso internacional, octubre de 2019.

\item Valles-Capetillo, E. and Giordano, M., “Comprehension of language in social contexts”, The Eleventh Annual Meeting of the Society 
for the Neurobiology of Language, Helsinki Finlandia, 20-22, Créditos en: Congreso internacional, agosto de 2019.

\item Navarrete, E., Valles-Capetillo, E., Giordano, M., “Metaphors comprehension throughout development and its neural correlate”, 
Chicago, IL, NEUROSCIENCE, 19-23, Créditos en: Congreso

\item Martínez, A., Y., Demertzi, A., Bauer, C.C.C., Gracia, Z., Alcauter, S., Barrios, F.A., “Body awareness results in 
characteristics 
dynamic functional connectivity states”, 25th Annual Meeting of the Organization for Human Brain Mapping, ROMA, 9-13, Créditos en: 
Congreso internacional, junio de 2019.

\item Gracia-Tabuenca, Z., Moreno B., Barrios, F., Alcauter, S., “Functional Segregation of the Sensorymotor and Associative Networks 
in 
Adolescence”, 25th Annual Meeting of the Organization for Human Brain Mapping, ROMA, 9-13, Créditos en: Congreso internacional, junio 
de 2019.

\item Gracia-Tabuenca, Z., Moreno, M.B., Barrios, F.A., Alcauter, S., "Inverse relationship homotopic and intra-hemispheric asymmetry 
of 
intrinsic brain functional connectivity in school-age children, Society for Neuroscience, San Diego, CA, November 3-7, Créditos en: 
Congreso internacional, noviembre de 2018.

\item Gracia-Tabuenca, Z., Moreno, M.B., Barrios, F.A., Alcauter, S., "Hemispheric Asymmetries in FunctionalConnectivity in Children 
correlatewith Visual Attention", OHBM, Annual Meeting, Singapore, June 17-21, Créditos en: Congreso internacional, junio de 2018.

\item Martínez, A.Y., Bauer, C.C.C., Gracia, Z., Alcauter, S., Barrios, F.A., "From Attention to Perception: A Dynamic Functional 
Connectivity Study in Experienced Meditators", OHBM, Annual Meeting, Singapure, June 17-21, Créditos en: Congreso internacional, junio 
de 2018.

\item Garcia-Hernandez, A., Ladron de Guevara Cervantes, D., Vasquez-Hernandez, A., AnguloPerkins, A., Nava-Gomez, L., Alcauter, S., 
“Cerebellar gray matter volume is differentiallyaffected by age in heavy marijuana users”, 23 Annual Meeting of the OHBM, 
Vancouver,Canada, Créditos en: Congreso internacional, junio de 2017.

\item Barrios, AF., Rodríguez, Ll., Alcauter, S, "Functional connectivity of the human claustrum",Society for Neuroscience, Washington, 
DC, EUA, Créditos en: Congreso internacional, noviembre de 2017.

\item Martínez, A., Bauer, C., Gracia, Z., Alcauter, S., Barrios, F. “Dynamical Functional ConnectivityStates may Detect Changes in 
Brain 
Patterns of Proprioception”, 23 Annual Meeting of theOHBM, Vancouver, British Columbia, Canada, Créditos en: Congreso internacional, 
junio de 2017.

\item Navarrete, E., Gracia, Z., Moreno, B., García, L., Ortiz, J.J., Barrios, F., Alcauter, S., "Executive functions performance 
correlates with fron to-parietal functional connectivity in children", 22 Annual Meeting of the OHBM, Geneva, Switzerland. , Créditos 
en: Congreso internacional, junio de 2016.

\item Reyes-Aguilar, A., Morales-Ramirez, E., Fernandez-Ruiz, J., Barrios, F., "Thinking About Mental States of Cooperators and 
Non-Cooperators Resulting from Personal Interactions", 22nd Annual Meeting of the OHBM, Geneva, Switzerland., Créditos en: Congreso 
internacional, junio de 2016.

\item Gracia, Z., Navarrete, E., Moreno, B., Barrios, F., Alcauter S., "Local efficiency of brain's functional connectivity reflects 
longitudinal changes in temporal cortex lateralization in school-age children and adolescents", Neuroscience, San Diego, EUA., Créditos 
en: Congreso internacional, noviembre de 2016.

\item Navarrete, E., Gracia, Z., Barrios, F., Alcauter, S., "Functional connectivity changes throughout childhood to adolescence", 
Neuroscience, San Diego, EUA., Créditos en: Congreso internacional, noviembre de 2016.

\item Gracia, Z., Moreno, B., Navarrete, E., Barrios, F., Alcauter, S., "Development of Functional Connectivity Asymmetries: 
Longitudinal 
effects in Children and Adolescents", 22nd Annual Meeting of the OHBM. Geneva, Switzerland., Créditos en: Congreso internacional, junio 
de 2016.

\item Bauer C., Díaz, J.L., Pasaye E., Barrios F.A., “Conscious perception as the inner construct of a priori structures: 
neurophenomenology through fMRI”, OHBM Annual Meeting, Hamburg, Germany, Jun 8-12, Créditos en: Congreso internacional, junio de 2014.

\item Vázquez P., Whitfield-Gabrieli S., Barrios F.A., “Elucidating dissociation in brain functional connectivity during hypnotic 
state. 
A rs-fMRI study”, OHBM Annual Meeting, Hamburg, Germany, Jun 8-12., Créditos en: Congreso internacional, septiembre de 2014.

\item Pasaye E., León-Vázquez M., Rodríguez-Chávez E., Molina-Carrión E., Barrios F.A. “Resting state cortico-cerebellar networks are 
absent in MS Patients and pseudobulbar affect”, OHBM Annual Meeting, Hamburg, 
Germany, Jun 8-12, 2014., Créditos en: Congreso internacional, septiembre de 2014.

\item León-Vázquez, M., Molina-Carrion, E., Barrios F., Pasaye, E., "Pseudobulbar Affect is an Affective Dysmetria in multiple 
Sclerosis 
Patients, A Connectivity Study", Créditos en: Congreso internacional, junio de 2013.

\item Bauer, C., Díaz, J., Concha L., Barrios, F., "Qualia in the Brain: an fMRI Approach to Neurophenomenology", Créditos en: Congreso 
internacional, junio de 2013.

\item Cavazos-Rodríguez R., Moreno B., Alcauter S., Concha L., Barrios F.A., “Hemispheric asymmetries of the Default Mode Network in 
Children”, 3 rd. Bienal Conference on Resting State Brain Connectivity, Magdeburg-Germany, September 5-7., Créditos en: Congreso 
internacional, septiembre de 2012.

\item Pasaye, E.H., Alcauter, S., Mercadillo, R.E., Bauer, C.C.C., Paz, J., Taboada, J. and Barrios, F.A., "Mirror System Involved in 
Tactile Stimuli", 18th Annual Meeting OHBM, Beijing-China, June 10-14., Créditos en: Congreso internacional, junio de 2012.

\item Bauer, C.C.C, Díaz, J.L., Concha, L., Barrios, F.A, "My is the Brain: Distinct Cortical Structures for offline or online body 
representations", 18th Annual Meeting OHBM, Beijing-China, June 10-14., Créditos en: Congreso internacional, junio de 2012.

\item Alcauter, S, Pasaye, E., Barrios, F.A., "Brain's white matter alterations in lower limb amputees: a DTI study", 18th Annual 
Meeting 
OHBM, Beijing-China, June 10-14., Créditos en: Congreso internacional, junio de 2012.

\item Rodríguez-Nieto G., Mercadillo R.E., Martínez-Soto J., Barrios F.A., “Gender differences of hemodynamic responses in brain 
regions 
activated while experiencing compassion”, Society for Neuroscience, New Orleans-USA, October 13-17., Créditos en: Congreso 
internacional, octubre de 2012.

\item Bauer, C.C.C., Díaz, J., Pasaye, E.H., Vázquez, P.G., Concha, L., Barrios, F.A., "Sustained Attention in Absence of External 
Stimuli Evokes Somesthetic Activity in the Primary Somatosensory Cortex: A Functional MRI Study", 17th Annual Meeting HBM, 
Quebec-Canada, June 26-30., Créditos en: Congreso internacional, junio de 2011.

\item Mercadillo, R.E., Diaz, J.L., Alcauter, S., Barrios, F.A., "Police culture inhibit the empathy related brain activity in women: A 
neuro-ethnographical proposal", 17th Annual Meeting of the Organization on Human Brain Mapping, Quebec-Canada, June 26-30., Créditos 
en: Congreso internacional, junio de 2011.

\item Angulo-Perkins, A., Aubé, W., Peretz, I., Barrios, F, Armony, J., Concha, L., "Music-specific responses within the temporal 
lobe", 
17th Annual Meeting of the Organization on Human Brain Mapping, Québec-Canada, June 26-30., Créditos en: Congreso internacional, 
noviembre de 2011.

\item Torres, C.A., Pasaye, E.H., Concha, L., Barrios, F.A., "Neural correlates of attitudinal evaluation: A fMRI study of attitudes 
towards animals", 17th Annual Meeting HBM, Quebec-Canada, June 26-30., Créditos en: Congreso internacional, junio de 2011.

\item Pasaye, E.H., Alcauter, S., Mercadillo, R.E., Bauer, C.C.C., Ortiz, J.J., Barrios, F.A., "Neural Correlates of Spatial Encoding 
of 
Sensory Stimuli in Healthy Subjects, an fMRI Study", 17th Annual Meeting HBM, Quebec-Canada, June 26-30., Créditos en: Congreso 
internacional, junio de 2011.

\item Romero-Romo, J. I., Romero-Cano, D., Graff-Guerrero, A., “Memantine: uncompetitive NMDA receptor ntagonist offers relief in 
patients with neuropathic pain”, International Association for the Study of Pain (IASP), 11th World Congress on Pain, August 21-26, 
Sydney, Australia, Créditos en: Congreso internacional, 2005.

\item Ollervides, M. A., Ortiz, J. J., Favila, G. R., Barrios, F. A., “Brain mapping of motor and imaginary motor activity with 
functional mri at 1.0 T”, SACNAS, Nacional Conference, Austin, Texas, EUA, Créditos en: Congreso internacional, 2004.

\item Prado-Alcalá, R., Ruiloba, M. I., Rubio, L., Solana-Figueroa, R., Galindo, L. E., Quitarte, G. L., “Interference with memory 
consolidation produced by serotonin infusions into different regions of the striatum”, Society for Neuroscience, Orlando, Florida, EUA, 
Créditos en: Congreso internacional, 2002.

\item Romero-Romo, J., Favila, R., Salgado., P., Barrios, F.A., Brushing stimuli in lower limb amputees evoke phantom limb perception; 
a 
fMRI study, Créditos en: Congreso internacional, 2000.
\end{enumerate}

\textbf{Otros agradecimientos}

\begin{enumerate}
\item Cisneros Mejorado, A. J* ., Hernández Cortés, A., Vélez Uriza, F. and Arellano, R. O., “DEMYELINATION AND REMYELINATION IN A 
PRECLINICAL MULTIPLE SCLEROSIS ANIMAL MODEL, APPROACHED LONGITUDINALLY, HISTOLOGICALLY AND FUNCTIONALLY”, XXXI Jornadas Académicas del 
instituto de Neurobiología, UNAM, del 23 al 27, Créditos en: XXXI Jornadas Académicas, septiembre de 2024.

\item Guerrero-Morales, J.R., García Miranda, L., Sánchez-Yépez, J., Mendoza-Trejo,Ma. S., GarcíaGomar, Ma. G., Giordano, M., 
Rodríguez-Córdova, V.M., “Cambios en el sistema nervioso central medidos por tensor de difusión en un modelo de exposición repetida al 
herbicida atrazina en roedores”, XXVI Reunión conjunta de Procesamiento de Neuroimágenes y Visión Computacional, CIMAT-Guanajuato, 9 al 
11, Créditos en: Jornadas académicas, octubre de 2024.

\item Ocampo Lunam P., Carbajal-Valenzuela, C.C., Fair, M., García-Gomar, M.G., “Estudio de la microestructura de los núcleos de la vía 
auditiva en adultos jóvenes mediante Imágenes de Resonancia Magnética”, XXVI Reunión conjunta de Procesamiento de Neuroimágenes y 
Visión Computacional, CIMAT-Guanajuato, 9 al 11 de octubre, Créditos en: Jornadas académicas, octubre de 2024.

\item Badillo Sanjuanero, S., Mendoza Medina, V., Ramírez González, D., Licea Haquet, G. L., Espinosa Méndez, I. M., Robles Rodríguez, 
G. 
D., Piña Hernández, A., Zaldivar, E., Medina Rivera, A., Ruiz Contreras, A. E., Rentería, M., Alcauter, S. and Domínguez Frausto, C. 
A., “ASOCIACIÓN DE LA EDAD CON EL VOLUMEN CORTICAL EN UNA MUESTRA DE ADULTOS MEXICANOS”, XXXI Jornadas Académicas del instituto de 
Neurobiología, UNAM, del 23 al 27, Créditos en: XXXI Jornadas Académicas del Instituto de Neurobiología., septiembre de 2024.

\item Espinosa Méndez, I. M., Díaz Patiño, J. C., Ramírez González, D., Román López, T. V., Sánchez Moncada, C. I., Robles Rodríguez, 
G. 
D., Licea Haquet, G. L., Domínguez Frausto, C. A., Piña Hernández, A., Encarnación Fernández, K. V., Zamora Suárez, A., Dorantes 
Larrauri, M., Medina Rivera, A., Ruiz Contreras, A., Rentería, M. and Alcauter, S., “GENETIC CONTRIBUTIONS OF THE FUNCTIONAL CONNECTOME 
TOPOLOGY IN THE MEXICAN POPULATION”, XXXI Jornadas Académicas del instituto de Neurobiología, UNAM, del 23 al 27, Créditos en: XXXI 
Jornadas Académicas del Instituto de Neurobiología., septiembre de 2024.

\item Alfaro Moreno, J. I., González Pérez, E. G., Carranza Aguilar, C. J. and Alcauter, S., “MORPHOLOGICAL CHANGES IN MOTOR CORTICES 
AND 
THEIR RELATIONSHIP WITH MOTOR PERFORMANCE OF RATS EXPOSED TO AN AEROBIC EXERCISE REGIMEN”, XXXI Jornadas Académicas del instituto de 
Neurobiología, UNAM, del 23 al 27, Créditos en: XXXI Jornadas Académicas del Instituto de Neurobiología., septiembre de 2024.

\item Encarnación Fernández, K. V., Zamora Suárez, A., Ramírez González, D., Licea Haquet, G. L., Espinosa Méndez, I. M., Robles 
Rodríguez, G. D., Piña Hernández, A., Zaldivar, E., Medina Rivera, A., Ruiz Contreras, A. E., Rentería, M., Alcauter, S. and Domínguez 
Frausto, C. A., “PREVALENCIA DE HALLAZGOS RADIOLÓGICOS EN IMÁGENES POR RESONANCIA MAGNÉTICA DEL REGISTRO MEXICANO DE GEMELOS”, XXXI 
Jornadas Académicas del instituto de Neurobiología, UNAM, del 23 al 27, Créditos en: XXXI Jornadas Académicas del Instituto de 
Neurobiología., septiembre de 2024.

\item Varela Correa B* ., Garay, E. and Arellano, R. O1 ., “MOLECULAR CHARACTERIZATION OF BINDING SITE OF CCB IN OLIGODENDROGLIAL GABAA 
RECEPTOR 3-2-1”, XXXI Jornadas Académicas del instituto de Neurobiología, UNAM, del 23 al 27, Créditos en: XXXI Jornadas Académicas 
del Instituto de Neurobiología., septiembre de 2024.

\item Rasgado-Toledo, J., Angeles-Valdez, D., Maya-Arteaga, J. P., Carranza-Aguilar, C., Lopez-Castro, A., 
Trujillo-Villarreal, L., Serrano, M. S., Medina-Sánchez, D., Elizarrarás-Herrera, A. D. and GarzaVillarreal, E., “STRESS INCREASES 
ETHANOL CONSUMPTION IN THE ACUTE PHASE BUT NOT IN THE CHRONIC OF AN IA2BC MODEL”, Jornadas Académicas del Instituto de Neurobiología, 
UNAM, 2023, Créditos en: 30 Jornadas Académicas del Instituto de Neurobiología UNAM, noviembre de 2023.

\item Vélez-Uriza, F., Cisneros-Mejorado, A., Garay, E. and Arellano, R. O., “LA ADMINISTRACIÓN SISTÉMICA DE LA CARBOLINA 
N-BUTILCARBOLINA3- CARBOXILATO (CCB) PROMUEVE LA REMIELINIZACIÓN EN EL MODELO DE DESMIELINIZACIÓN POR CUPRIZONA”, Jornadas 
Académicas del Instituto de Neurobiología, UNAM, 2023, Créditos en: 30 Jornadas Académicas del Instituto de Neurobiología UNAM, 
noviembre de 2023.

\item Cisneros-Mejorado, A., Ordaz, R. P., Garay, E., and Arellano-Ostoa, R, “CARBOLINES THAT ENHANCE GABAAR RESPONSE EXPRESSED IN 
OLIGODENDROCYTES PROMOTE REMYELINATION IN AN IN VIVO RAT MODEL OF FOCAL DEMYELINATION”, Jornadas Académicas del Instituto de 
Neurobiología, UNAM, 2023, Créditos en: 30 Jornadas Académicas del Instituto de Neurobiología UNAM, noviembre de 2023.

\item Hernández, A., Cisneros-Mejorado, A., Sierra-Camacho, J. J., Arellano, R. O. and MartínezTorres, A., “NEURAL PROGENITOR CELL 
TRANSPLANTATION FOR THE REGENERATION THERAPY IN A HEMI-PARKINSON DISEASE MODEL”, Jornadas Académicas del Instituto de Neurobiología, 
UNAM, 2023, Créditos en: 30 Jornadas Académicas del Instituto de Neurobiología UNAM, noviembre de 2023.

\item Trujillo-Villarreal, L.A., Cruz-Carrillo, G., Angeles-Valdez, D., Garza-Villarreal, E. and CamachoMorales, A., “EFECTO 
TRANSGENERACIONAL DE LA PROGRAMACIÓN FETAL POR DIETA MATERNA SOBRE LA ESTRUCTURA CEREBRAL Y SU ASOCIACIÓN A CONDUCTAS SIMILARES A LA 
ANSIEDAD EN RATAS MACHO”, Académicas del Instituto de Neurobiología, UNAM, 2023, Créditos en: 30 Jornadas Académicas del Instituto de 
Neurobiología UNAM, noviembre de 2023.

\item Moreno, J. A., González-Pérez, E. G., Rocha-García, M., Carranza-Aguilar, C. J. and AlcauterSolórzano, S., “CAMBIOS MORFOLÓGICOS 
EN 
LAS CORTEZAS CEREBRALES MOTORA PRIMARIA Y MOTORA SECUNDARIA Y SU RELACIÓN CON EL DESEMPEÑO MOTRIZ DE RATAS WISTAR EXPUESTAS A UN 
RÉGIMEN DE EJERCICIO AERÓBICO”, Jornadas Académicas del Instituto de Neurobiología, UNAM, 2023, Créditos en: 30 Jornadas Académicas del 
Instituto de Neurobiología UNAM, noviembre de 2023.

\item Espinosa-Méndez, I.M., Román-López, T. V., Ramírez-González, D., Sánchez-Moncada, C. I., Díaz-Téllez, X., Domínguez-Frausto, C. 
A., 
Murillo-Lechuga, V., López-Camaño, X. J., GuzmánTenorio, G. E., Robles-Rodríguez, G. D., Ortiz-Tapia, E. B., Piña-Hernández, A., 
Aldana-Assad, O.., Medina-Rivera, A., Ruiz-Contreras, A. E., Rentería, M.. and Alcauter, S., “HERITABILITY AND GENETIC RELATIONSHIPS OF 
CORTICAL SURFACE AREA AND THICKNESS IN THE MEXICAN POPULATION”, Jornadas Académicas del Instituto de Neurobiología, UNAM, 2023, 
Créditos en: 30 Jornadas Académicas del Instituto de Neurobiología UNAM, noviembre de 2023.

\item Rocha-García, M., González-Pérez, E. G., Alfaro-Moreno, J., Carranza-Aguilar, C. J. and Alacuter, S, “CAMBIOS EN LA PLASTICIDAD 
CEREBRAL EN EL HIPOCAMPO DE ROEDORES EXPUESTOS A EJERCICIO AERÓBICO EXPLORADOS A TRAVÉS DE INMUNOHISTOQUÍMICA”, Jornadas Académicas del 
Instituto de Neurobiología, UNAM, 2023, Créditos en: 30 Jornadas Académicas del Instituto de Neurobiología UNAM, noviembre de 2023.

\item Serrano-Ramírez, M. S., Rasgado-Toledo, J., Medina-Sánchez, D., Elizarrarás-Herrera, A. D., Ángeles-Valdez, D., Carranza-Aguilar, 
C. J. and Garza-Villarreal, E. A, “ALTERACIONES CEREBELOSAS Y COGNITIVAS INDUCIDAS POR LA AUTOADMINISTRACIÓN CRÓNICA DE MORFINA EN 
RATAS WISTAR MACHO”, Jornadas Académicas del Instituto de Neurobiología, UNAM, 2023, Créditos en: 30 Jornadas Académicas del Instituto 
de Neurobiología UNAM, noviembre de 2023.

\item Vázquez, A., Garay, E., Arellano O., R. and Cisneros-Mejorado, A., “SYSTEMIC TREATMENT WITH THE STEROID GANAXOLONE 
INDUCES HETEROGENEOUS EFFECTS IN BRAIN FOLLOWING CUPRIZONE INDUCED DEMYELINATION”, Jornadas Académicas del Instituto de Neurobiología, 
UNAM, 2023, Créditos en: 30 Jornadas Académicas del Instituto de Neurobiología UNAM, noviembre de 2023.

\item González-Pérez, E., Ortíz-Retana, J., Alfaro-Moreno, J., Rocha-García, M., Espinosa-Mendez, I., Gasca-Martinez, D., 
Carranza-Aguilar, C. J. and Alcauter-Solórzano, S., “EFECTOS DEL AMBIENTE ENRIQUECIDO EN EL COMPORTAMIENTO, LA CONECTIVIDAD FUNCIONAL Y 
LA MORFOLOGÍA CEREBRAL DE ROEDORES DESDE LA INFANCIA HASTA LA ADULTEZ TEMPRANA”, Jornadas Académicas del Instituto de Neurobiología, 
UNAM, 2023, Créditos en: 30 Jornadas Académicas del Instituto de Neurobiología UNAM, noviembre de 2023.

\item Ramírez-González, D., Román-López, T.V., Sánchez-Moncada, C.I., Espinosa-Méndez, I. M., García-Vilchis, B., Díaz-Téllez, X., 
Domínguez-Frausto, C. A., Murillo-Lechuga, V., López-Camaño, X. J., Guzmán-Tenorio, G. E., Robles-Rodríguez, G. D., Ortiz-Tapia, E. B., 
Piña-Hernández, A., AldanaAssad, O., Medina-Rivera, A., Ruiz-Contreras, A. E., Rentería, M. and Alcauter, S., “AVANCES DEL REGISTRO 
MEXICANO DE GEMELOS (TWINSMX) PARA LA CARACTERIZACIÓN DE LA HEREDABILIDAD DE FENOTIPOS CONDUCTUALES Y DE FUNCIONAMIENTO CEREBRAL”, 
Jornadas Académicas del Instituto de Neurobiología, UNAM, 2023, Créditos en: 30 Jornadas Académicas del Instituto de Neurobiología 
UNAM, noviembre de 2023.

\item López Gutiérrez, M.F., Gracia Tabuenca, Z., Ortiz, J., Camacho, F., Paredes, R.G., Young, L. J., Diaz, N.F., Alcauter, S. and 
Portillo W., "CAMBIOS EN LA CONECTIVIDAD FUNCIONAL CEREBRAL POR LA CRIANZA MONOPARENTAL EN EL TOPILLO DE LA PRADERA", XXIX Jornadas 
Académicas del Instituto de Neurobiología, Créditos en: Congreso local, septiembre de 2022.

\item Vázquez, A., Vélez Uriza, F., Garay, E., Arellano, Ostoa, R. and Cisneros Mejorado, A., "IMPROVEMENT OF REMYELINATION IN 
DEMYELINATED CENTRAL NERVOUS SYSTEM USING GANAXOLONE IN THE CUPRIZONE MICE MODEL OF MULTIPLE SCLEROSIS", XXIX Jornadas Académicas del 
Instituto de Neurobiología, Créditos en: Congreso local, septiembre de 2022.

\item Villaseñor, P.J., Aquiles, A., Luna Munguia, H., Larriva Sahd, J. and Concha, L., "ANÁLISIS LONGITUDINAL DE LA CORTEZA CEREBRAL 
EN 
UN MODELO ANIMAL DE DISPLASIA CORTICAL", XXIX Jornadas Académicas del Instituto de Neurobiología, Créditos en: Congreso local, 
septiembre de 2022.

\item Coutiño, D., Hidalgo Flores, F.J., Gasca Martínez, D., Concha, L., Luna Munguia, H., "ANÁLISIS LONGITUDINAL DE LOS CAMBIOS 
MICROESTRUCTURALES EN FIMBRIA E HIPOCAMPO TRAS LA LESIÓN DEL SEPTUM MEDIAL MEDIANTE IMÁGENES DE DIFUSIÓN", XXIX Jornadas Académicas del 
Instituto de Neurobiología, Créditos en: Congreso local, septiembre de 2022.

\item Rasgado Toledo, J., Ángeles Valdez, D., Carranza Aguilar, C., Maya Arteaga, J. P., Ortuzar, D., López Castro, A. and Garza 
Villarreal, E. A., "EL ESTRÉS AUMENTA EL CONSUMO DE ETANOL EN LA FASE AGUDA PERO NO EN LA FASE CRÓNICA DE UN MODELO DE ELECCIÓN DE 2 
BOTELLAS DE ACCESO INTERMITENTE EN RATAS WISTAR", XXIX Jornadas Académicas del Instituto de Neurobiología, Créditos en: Congreso local, 
septiembre de 2022.

\item Ocampo Ruiz, A. L., Dimas Rufino, M. A., Castillo, X., Vázquez Carrillo, D., Dena Beltrán, J. L., Garay, E., Martínez de la 
Escalera, G., Clapp, C., Arellano, R., Cisneros Mejorado, A. and Macotela, Yazmín", PROLACTIN RECEPTOR DEFICIENCY PROMOTES 
HYPOMYELINATION IN THE CORPUS CALLOSUM DURING POSTNATAL DEVELOPMENT IN MICE", XXIX Jornadas Académicas del Instituto de Neurobiología, 
Créditos en: Congreso local, septiembre de 2022.

\item Cisneros Mejorado, A., Garay, E., Moctezuma, J. P. and Arellano, R, “B-CARBOLINES IMPROVE REPAIR OF WHITE MATTER INJURY IN A 
FOCAL 
DEMYELINATION MURINE MODEL”, Créditos en: Jornadas Académicas INB, noviembre de 2021.

\item Lazcano, I., Cisneros Mejorado, A., Hernández, Y., Ortiz Retana, J., Arellano, R., Concha, L. and Orozco, A, “DIFERENCIAS EN EL 
SISTEMA NERVIOSO CENTRAL ENTRE UN AJOLOTE PRE- Y POST- METAMORFICO INDUCIDO POR 
HORMONAS TIROIDEAS”, Créditos en: Jornadas Académicas INB, noviembre de 2021.

\item García Saldivar, P., De León Andrez, C., Ayala, Y.A., Prado, L., Concha, L. and Merchant, H, “THE ROLE OF SUPERFICIAL WHITE 
MATTER IN THE SENSORIMOTOR SYNCHRONIZATION”, Créditos en: Jornadas Académicas INB, noviembre de 2021.

\item Vélez Uriza, F. Z., Cisneros Mejorado, A., Garay, E. and Arellano, R. O, “DEMYELINATION–REMYELINATION OF CEREBELLAR PEDUNCLES OF 
MICE EVALUATED WITH MRI AND BLACKGOLDII STAIN “, Créditos en: Jornadas Académicas INB, noviembre de 2021.

\item Valles Capetillo, E. and Giordano, M, “THE NEUROCOGNITION OF SOCIAL COMMUNICATION”, Créditos en: Jornadas Académicas INB, 
noviembre 
de 2021.

\item Munguía-Villanueva Deyanira, Cisneros-Mejorado Abraham, Arellano O. Rogelio”, MOUSE CORPUS CALLOSUM DEMYELINATION EVALUATED BY 
DIFFUSION WEIGHTED MAGNETIC RESONANCE IMAGING”, Créditos en: Jornadas Académicas INB, noviembre de 2021.

\item Ocampo Ruiz, A. L., Arellano O, R., Garay, E., Martínez de la Escalera, G., Clapp C., Cisneros Mejorado, A.. and Macotela, Y., 
“PROLACTIN RECEPTOR DEFICIENCY PROMOTES HYPOMYELINATION IN THE DEVELOPING CENTRAL NERVOUS SYSTEM OF MICE”, Créditos en: Jornadas 
Académicas INB, noviembre de 2021.

\item Cortes, G., B. A., Regalado, M., Concha, L. and Luna Munguía, H, “ANALISIS LONGITUDINAL DE LOS CAMBIOS DEL SISTEMA LIMBICO TRAS 
LA MODULACION DE LA VIA SEPTOHIPOCAMPAL EN UN MODELO DE EPILEPSIA”, , Créditos en: Jornadas Académicas INB, noviembre de 2021.

\item Romero Santiago, S., Cisneros Mejorado, A. and Arellano, R, O, “CURSO TEMPORAL DE LA MIELINIZACION EN LA ETAPA POSNATAL DE RATON 
EVALUADO CON RESONANCIA MAGNETICA E HISTOLOGIA”, Créditos en: Jornadas Académicas INB, noviembre de 2021.

\item Olalde-Mathieu, V.E, Sassi, F., Reyes-Aguilar, A., Mercadillo, R.E., Alcauter, S., Barrios, F.A., “Psychotherapists present 
differences related to empathy sub-processes when compared to nonpsychotherapists”, Jornadas Académicas del INB-UNAM, 23-27 , Créditos 
en: Otro, septiembre de 2019.

\item Delgado-Herrera, M. and Giordano M., “How to study the brain while lying? A systematic review”, 23-27, Créditos en: Jornadas 
Académicas INB, septiembre de 2019.

\item Atilano-Barbosa, D., Passaye Alcaraz, E.H. and Mercadillo-Caballero, R. E., “Cognitive function and brain morphometry in Mexican 
workers occupational exposed to solvents”, 23-27, Créditos en: Jornadas Académicas INB, septiembre de 2019.

\item Garcia-Saldivar, P., De León-Anfrez, C., Ayala Y., A., Prado, L., Concha L. and Merchant, H., “The role of the superficial white 
matter in the sensoriomotor synchronization”, 23-27, Créditos en: Jornadas Académicas INB, noviembre de 2019.

\item Barbosa-Luna, M., Ricardo-Garcell, J., Alcauter-Solorzano, S., Solis-Vivanco, R., GarcíaHernández, S. and Pasaye-Alcaraz, E.H., 
"MULTIDIMENSIONAL CHARACTERIZATION OF OBSESSIVE-COMPULSIVEDISORDER. A CLINICAL, NEUROPSYCHOLOGICAL AND NEUROIMAGEN PERSPECTIVE", 25 
Jornadas Académicas, Instituto de Neurobiología, UNAM, Septiembre 24-28, Créditos en: Jornadas Académicas INB, septiembre de 2018.

\item Gracia-Tabuenca, Z., Moreno, B., Barrios, F., Alcauter, S., "CHARACTERIZATION OF THE FUNCTIONAL BRAIN NETWORK IN ADOLESCENCE:A 
FUNCTIONAL SEGREGATION PROCESS THAT PREDICTS MEMORY PERFORMANCE", 25 Jornadas Académicas, Instituto de Neurobiología, UNAM, Septiembre 
24-28, Créditos en: Jornadas Académicas INB, noviembre de 2018.

\item García-Saldivar, P., De León, C., Prado, L., Concha, L., Merchant H., "LONGITUDINAL STRUCTURAL CHANGES IN GRAY MATTER OF THE 
RHESUSMONKEY ASSOCIATED TO TRAINING IN A SENSORIMOTOR SYNCHRONIZATION TASK", 25 Jornadas Académicas, Instituto de Neurobiología, UNAM, 
Septiembre 24-28, Créditos en: Jornadas Académicas INB, noviembre de 2018.

\item Gallego, R. J., Corsi Cabrera, M., Garcell Ricardo, J., Alcauter Solorzano, S., Pasaye Alcaraz, E., "ADRESSING MRI DATA 
RELIABILITY 
DURING SIMULTANEOUS EEG-FMRIRECORDING", 25 Jornadas Académicas, Instituto de Neurobiología, UNAM, Septiembre 24-28, Créditos en: 
Jornadas Académicas INB, noviembre de 2018.

\item Fajardo-Valdez, A., Rodríguez-Cruces, R., Rosas-Carrera, A.E., Concha L, Pasaye-Alcaraz, E., "ALTERATIONS IN RESTING STATE FMRI 
CONNECTIVITY AND COGNITIVEPERFORMANCE IN TEMPORAL LOBE EPILEPSY PATIENTS", 25 Jornadas Académicas, Instituto de Neurobiología, UNAM, 
Septiembre 24-28, Créditos en: Jornadas Académicas INB, noviembre de 2018.

\item Carrillo-Peña, AV, Valles-Capetillo, DE, Licea-Haquet, GL, and Giordano, M., "Effect of familiarity in the interpretation of 
pragmatic language in healty mexican subjects", Créditos en: Jornadas Académicas INB, septiembre de 2016.

\item Bauer, C.C.C., "The after effect of meditation: Increased resting-state functional connectivity after a single 20 min meditation 
epoch", 21 Annual Meeting of the Organization for Human Brain Mapping, Honolulu Hawaii, Créditos en: Otro, junio de 2015.

\item Liliana García, "Association between functional connectivity and language abilities in school-age children", 21 Annual Meeting of 
the Organization for Human Brain Mapping, Honolulu Hawaii, Créditos en: Otro, diciembre de 2015.

\item Gracia, Z., "Functional Connectivity asymmetries in School-Age Children: Sex and Cognitive Performance Effects", 21 Annual 
Meeting 
of the Organization for Human Brain Mapping, Honolulu Hawaii, Créditos en: Otro, junio de 2015.

\item Gracia Z, "Asimetrías en conectividad funcional cerebral en niños de edad escolar", 22 Jornadas Académicas del Instituto de 
Neurobiología, UNAM, Créditos en: Apoyo técnico, octubre de 2015.

\item Hernandez-Rios, E.N, Barrios, A.F., Utilización de programas para el analisís de imágenes como apoyo a la investigación en el 
INB, 
Créditos en: Otro, 2009.

\item García-Solís, P., Anguiano, B., Delgado, G., Aceves, C., “Diferencias de señalización y captura del yodo molecular (I2) en la 
glándula mamaria lactante, virgen y neoplástica”, Jornadas del Instituto de Neurobiología, Sep. 19-23, Querétaro, México, Créditos en: 
Otro, 2005.

\item Ortiz, J. J., harmony, T., Fernández-Bouzas, A., Barrios, F., A., “Espectroscopia por resonancia magnética en 1.0T en infantes”, 
Jornadas del Instituto de Neurobiología, Sep. 19-23, Querétaro, México, Créditos en: Otro, 2005.

\item Castillo, C. G., Mendoza, M. S., Falcón, A., Aguilar, M. B., Giordano, M., “Cuantificación de gaba liberado por una línea celular 
estriatal inmortalizada y trasnfectada con el cdna del gad67h”, Jornadas del Instituto de Neurobiología, Sep. 19-23, Querétaro, México, 
Créditos en: Otro, 2005.

\item León-Jacinto, U., Quitarte, G., Serafín-López, N., Aguilar-Vázquez, A., Beltrán-Campos, V., Prado-Alcalá, Díaz-Miranda, S., Y., 
“Cambios en densidad y tipo de espinas dendríticas en el hipocampo asociados a una tarea de sobreentrenamiento”, Jornadas del Instituto 
de Neurobiología, Sep. 19-23, Querétaro, México, Créditos en: Otro, 2005.

\item Aranda-López, N., Delgado, G., Aceves, C., Anguiano-Serrando, B. “Actividad desyodativa tipo I (dio 1) durante la ontogenia del 
epidídimo”, Jornadas del Instituto de Neurobiología, Sep. 19-23, Querétaro, México, Créditos en: Otro, 2005.

\item Arroyo-Helguera, O., Aceves, C., “Captura y efecto atiproliferativo del yodo molecular (I2) en cultivos celulares de cancer 
mamario”, Jornadas del Instituto de Neurobiología, Sep. 19-23, Querétaro, México, Créditos en: Otro, 2005.

\item Sandoval-Minero, M. T., Varela-Echavarría, A., “Interacciones axonales entre neuronas decusantes como un mecanismo para el cruce 
de 
la línea media en vertebrados”, Jornadas del Instituto de Neurobiología, Sep. 19-23, Querétaro, México, Créditos en: Otro, 2005.

\item López-Juárez, A., Aceves, C., Delgado, G., Anguiano, B., “La actividad sexual incrementa la producción local triyodotironina (T3) 
en el lóbulo ventral de la próstata”, Jornadas del Instituto de Neurobiología, Sep. 19-23, Querétaro, México, Créditos en: Otro, 2005.

\item Beltrán-Campos, V., Rodríguez-Santillán, E., Quitarte, G., Serafín-López, N., Aguilar-Vázquez, A., Díaz’Miranda, S. 
Y., “Influencia estrogémica con la densidad de espinas dendríticas de las células piramidales del CA1 del hipocampo”, Jornadas del 
Instituto de Neurobiología, Sep. 19-23, Querétaro, México, Créditos en: Otro, 2005.

\item Frías, C., Torrero, C., Regalado, M., Salas, M., “Desarrrollo de las células mitrales en ratas desnutridas posnatalmente: 
posibles 
alteraciones funcionales”, Jornadas del Instituto de Neurobiología, Sep. 19-23, Querétaro, México, Créditos en: Otro, 2005.

\item Tinajero, A., Ramos., M., E., Morales, T., “Respuesta diferencial de los núcleos hipotalámicos, paraventricular y supraóptico, al 
estrés osmótico durantes el ciclo estral de la rata”, Jornadas del Instituto de Neurobiología, Sep. 19-23, Querétaro, México, Créditos 
en: Otro, 2005.

\item Aguilar-Vázquez, A., Martínez-Jaramillo, J., Beltrán-Campos, V., Díaz-Miranda, S. Y., “Influencia de los estrógenos en la 
plasticidad del hipocampo: espinogénesis”, Jornadas del Instituto de Neurobiología, Sep. 19-23, Querétaro, México, Créditos en: Otro, 
2005.

\item Díaz del Guante, M.A., Quiroz, C., Garín-Aguilar, M.E., Quirarte, G.L. y Prado-Alcalá R. “Bloqueo temporal de la amígdala: 
¿Reconsolidación o evocación?”, Jornadas del Instituto de Neurobiología, Sep. 20-24, Querétaro, México, Créditos en: Otro, 2004.

\item Delgado, G., Anguiano, B., Aceves, C. “La interacción hormonas tiroideas-acido retinoico es necesaria para antener la 
diferenciación del epitelio en carcinomas mamarios”, Jornadas del Instituto de Neurobiología, Sep. 20-24, Querétaro, México, Créditos 
en: Otro, 2004.

\item Martínez, Y., Díaz-Cintra, S., Díaz del Guante, M., Aguilar A., Cárabez-Trejo, A., Quirarte G. L., Prado-Alcalá R., “Hipocampo 
Senil: Un estudio conductal y molfológico”, Jornadas del Instituto de Neurobiología, Sep. 20-24, Querétaro, México., Créditos en: Otro, 
2004.

\item Quiroz, C., Díaz del Guante, M. A., Garín-Aguilar, M. E., Quirarte, G. L., Prado-Alcalá R., “Inactivación del Hipocampo Dorsal: 
Efectos sobre la retención de la memoria evocada de dos diferentes niveles de reforzamiento”, Jornadas del Instituto de Neurobiología, 
Sep. 20-24, Querétaro, México, Créditos en: Otro, 2004.

\item Cepeda Nieto, A. C., Pfaff, S. , Varela-Echavarría, A., “Diferenciación y proyección de las neuronas ticuloespinales 
romboencefálicas”, Jornadas del Instituto de Neurobiología, Sep. 20-24, Querétaro, México, Créditos en: Otro, 2004.

\item Soriano, O., Anguiano, B., Aceves, C., “Inhibición del efecto protector del lugol por progesterona en el cáncer mamario inducido 
por 7,12-dimetilbenzeno[a]antraceno (DMBA)”, Jornadas del Instituto de Neurobiología, Sep. 20-24, Querétaro, México, Créditos en: Otro, 
2004.

\item Sandoval-Minero, M.T., Varela-Echavarría, A., “Regulación de la proyección axonal de neuronas decusantes en el rombencéfalo 
caudal”, Jornadas del Instituto de Neurobiología, 20 al 24 de Sep, Créditos en: Otro, 2004.

\item Castillo, R. A., Villalobos, M. S., Galindo, L. E., Quirarte G. L., Prado-Alcalá R. “Efectos de la inactivación reversible del 
estriado sobre la evocación de la memoria de un aprendizaje incrementado”, Jornadas del Instituto de Neurobiología, Sep. 22-26, 
Querétaro, México, Créditos en: Otro, 2003.

\item Rojas-Huidobro, R., Aceves C., “Efecto Protector de T4 y KI en la inducción del cáncer mamario por 7, 12-dimetilbenzo[a] 
antraceno 
(DMBA) en ratas púberes”, Jornadas del Instituto de Neurobiología, Sep. 22-26, Querétaro, México, Créditos en: Otro, 2003.

\item Quiroz C., Quirarte G. L., Prado-Alcalá R., “El sobrerreforzamiento impide la deficiencia en la retención inducida por la 
aplicación pre-entrenamiento de TTX en el hipocampo”, Jornadas del Instituto de Neurobiología, Sep. 22-26, Querétaro, México, Créditos 
en: Otro, 2003.

\item Granados-Rojas, L., Sánchez, A., Aguilar, A., Quirarte G. L., Prado-Alcalá R., Díaz-Cintra, S., “Crecimiento de fibras musgosas 
después del sobreentrenamiento en el laberinto acuático de morris en ratas con malnutrición hipoproteínica prenatal, crónica y 
postnatal”, Jornadas del Instituto de Neurobiología, Sep. 22-26, Querétaro, México, Créditos en: Otro, 2003.

\item Aguilar Vázquez, A., Granados-Rojas, L. y Díaz-Cintra S., “Estudio inmunocitoquimico de las células gabaergicas en el hipocampo 
de 
la rata con malnutrición hipoproteínica crónica de 30 y 60 días de edad”, Jornadas del Instituto de Neurobiología, Sep. 22-26, 
Querétaro, México, Créditos en: Otro, 2003.

\item Quiroz, C., Garín-Aguilar, M. E., Morales T., Quirarte G. L., Prado-Alcalá R. “Efecto de la tetrodotoxina (TTX) en el hipocampo 
de 
ratas: expresión de C-FOS inducida por ácido kaínico (AK) y pentilenetetrazol (PTZ) como indicador de inactivación”, Jornadas del 
Instituto de Neurobiología, Sep. 22-26, Querétaro, México, Créditos en: Otro, 2003.

\item Teja, I. S., Prado-Alcalá R. Quirarte, G. L., “Participación de los receptores a corticosterona estriatales en la memoria de una 
tarea de evitación inhibitoria”, Jornadas del Instituto de Neurobiología, Sep. 22-26, Querétaro, México, Créditos en: Otro, 2003.

\item Moreno-Ocaña, G., Martínez-Cabrera, G., Varela-Echavarría, A., A., Condés-Lara, M., LarrivaSahd, J., “Neuronas interfasciculares 
de 
la comisura anterior: Caracterización citológica e interacciones locales”, Jornadas del Instituto de Neurobiología, Sep. 23-27, 
Querétaro, México, Créditos en: Otro, 2002.

\item Sánchez, A., Granados-Rojas, L., Pineda, T., Aguilar, A., Prado-Alcalá, R. Quirarte, G. L., DíazCintra, S., “Efecto del sobre 
entrenamiento en el laberinto acuático de morris sobre la plasticidad sináptica de las fibras musgosas en ratas malnutridas”, Jornadas 
del Instituto de Neurobiología, Sep. 23-27, Querétaro, México, Créditos en: Otro, 2002.

\item Reyes-Haro, D., Miledi, R., García-Colunga, J., “Transmisión serotonérgica en el cuerpo calloso de la rata”, Jornadas del 
Instituto 
de Neurobiología, Sep. 23-27, Querétaro, México, Créditos en: Otro, 2002.

\item Aguilar-Vázquez, A., Granados-Rojas, L., Díaz-Cintra S., “Marcaje in vitro de vesículas sinápticas activas con FMI-43 en el 
hipocampo de ratas de 220 días con malnutrición prenatal y crónica”, Jornadas del Instituto de Neurobiología, Sep. 23-27, Querétaro, 
México, Créditos en: Otro, 2002.
\end{enumerate}


\textbf{Premios y distinciones}

\hfill

1. Premio: Primer Lugar, "Segmentación semi-automática por crecimiento de regiones con estimación robusta: aplicación a imágenes por 
resonancia magnetica", Otorgado por: Federación Mexicana de Radiologia e Imagen, A. C, septiembre de 1998.

\textbf{Estímulos académicos}

1. PRIDE C Dirección General de Asuntos del Personal Académico, INB-UNAM, Fecha de inicio: marzo de 2003.

\hfill

\textbf{Proyectos de Colaboración}

\hfill

1. “Propiedades Turnpike en Problemas de Control Óptimo Estocástico”, Responsable: Dr. Cutberto Romero Melèndez, Tipo: Investigación, 
Status: Inicio, Monto 0.00, Fecha de inicio: abril de 2023, Fecha de conclusión: abril de 2026.

\hfill

\textbf{Artículos publicados Internacionales}

\begin{enumerate}
\item Hevia-Orozco, J.C., Reyes-Aguilar, A., Hernández-Pérez, R., González-Santos, L., Pasay, E H., Barrios, F.A., Personality Traits 
Induce Different Brain Patterns When Processing Social and Valence Information, Frontiers in Physiology, enero de 2022, 1-9.

\item Romero-Meléndez, C. , Castillo-Fernández, D. and González-Santos, L., “On the Boundedness of the Numerical Solutions’ Mean Value 
in 
a Stochastic Lotka–Volterra Model and the Turnpike Property“, Hindaw Complexity, noviembre de 2021, 1-14.

\item Romero-Meléndez, C., González-Santos, L., Castillo-Fernández, “A NUMERICAL APPROACH FOR THE STOCHASTIC CONTROL OF A TWO-LEVEL 
QUANTUM SYSTEM”, CYBERNETICS AND PHYSICS, agosto de 2020; 9(2), 107-116.

\item Velásquez-Upegui, E.P., Tovar-González, J., González, L., “Hacia una caracterización prosódica de los actos del habla directivos: 
producción y percepción de mandatos”, Lengua y Habla, Revista del Centro de Investigación y Atención Lingüística C.I.A.L, agosto de 
2020; 24, 1-5.

\item Muley, V.Y., López-Victorio, C.J., Ayala-Sumuano, J.T., González-Gallardo, A., González-Santos, L., Lozano-Flores, C., Wray, G., 
Hernández-Rosales, M., Varela-Echavarría, A., “Conserved and divergent expression dynamics during early patterning of the T 
telencephalon in mouse and chick embryos”, Progress In Neurobiology, enero de 2020; 186, 1-15.

\item Martínez-Soto, J., De la Fuente Suárez, L.A., González-Santos, L., Barrios, F.A., “Observation of environments with different 
restorative potential results in differences in eye patron movements and pupillary size”, IBRO Reports, noviembre de 2019; 7, 52-58.

\item Nanni, M., Martínez-Soto, J., González-Santos, L., Barrios, F.A., "Neural correlates of the natural observationof an emotionally 
loaded video", Plos One, noviembre de 2018, 1-19.

\item Martinez J, Gonzalez-Santos L, Barrios F.A., AFFECTIVE AND RESTORATIVE VALENCES FOR THREE ENVIRONMENTAL CATEGORIES, Perceptual 
And 
Motor Skills, octubre de 2014, 1-10.

\item Beatriz Moreno M., Concha L., González-Santos L., Ortiz J.J., Barrios F.A., , Correlation between Corpus Callosum Sub-Segmental 
Area 
and Cognitive Processes in School-Age Children, Plos One, agosto de 2014; 9(8), 1-10.

\item Bauer C.C., Moreno B., Gonzalez-Santos L., Concha L., Barquera S., Barrios F.A., Child overweight and obesity are associated with 
reduced executive cognitive performance and brain alterations: a magnetic resonance imaging study in Mexican children, Pediatric 
Obesity, julio de 2014.

\item Martínez-Soto J., González-Santos L., Pasaye E. and Barrios F.A., Exploration of neural Correlates of restorative environment 
exposure through functional magnetic resonance, Intelligent Buildings International, Taylor Francis Online, septiembre de 2013; 5(S1), 
10-28.

\item Gonzalez-Santos L, Mercadillo RE, Graff A, Barrios FA, "Versión computarizada para la aplicación del Listado de Síntomas 90 (SCL 
90) y del Inventario de Temperamento y Carácter (ITC)", Salud Mental, julio de 2007; 30(4), 31-40.

\item Barríos F.A., González L., Fávila R., Alonso M.E., Salgado P., Díaz R., Fenández, J., OLFACTION AND NEURODEGENERATION IN HD, 
Neuroreport, enero de 2007; 18(1), 73-76.

\item Huerta-Ocampo I., Mena F., Barríos F.A., Martínez G., González L., Larriva-Sahd J.L., Perinatal Exposure to Androgen Suppresses 
Sexual Dimorphism in Nerve Trunk Diameter, Axon Number, and Fiber Size Spectrum: a Quantitative Ultrastructural Study of the Adult Rat 
Mammary Nerve., Brain Research, octubre de 2005; 1060(1-2), 179-183.

\item George-Téllez R, Segura-Valdez ML, González-Santos L, Jiménez-García LF, "Cellular organization of pre-mRNA splicing factors in 
several tissues. Changes in the uterus by hormone action", Biology of the cell, mayo de 2002; 94(2), 99-108.

\item González-Santos, L., Rojas-Jasso, R., Salgado, P., Ponte-Romero, R., Sánchez-Cortázar, J., Barrios, F.A., "Segmentación 
semiautomática de imágenes por resonancia magnética cerebral por crecimiento de regiones con estimación robusta: una validación 
radiológica", Revista Mexicana De Radiologia, abril de 1999; 53(2), 51-54.
\end{enumerate}

\textbf{Artículos enviados o aceptados}

\begin{enumerate}
\item Martínez-Soto, J., González-Santos, L., Pasaye, E.H., “Adityavarman, R., Barrios, F.A, “Restorative Influences of Nature vs. 
Urban 
Settings on Resting State Networks of FunctionalBrain Connectivity”, Environmental Psychology, diciembre de 2017, Status: Enviado.

\item Romero-Meléndez, C., González-Santos, L., An Iterative Algorithm for Optimal Control of TwoLevel Quantum Systems, CYBERNETICS AND 
PHYSICS, diciembre de 2017; 6(4), 231-238, Status: Aceptado.
\end{enumerate}

\textbf{Artículos en memorias arbitradas}

\hfill

1. Romero-Meléndez, C, González-Santos, L.

Numerical Approximation in Optimal Control ofTwo-Level Quantum System"; PHYSCON, Florence, T, 2017.

\hfill

\textbf{Artículos en memorias In Extenso}

\begin{enumerate}
\item Barrios, F. A., González-Santos, L., Favila, R., Rojas, R., Sánchez-Cortazar, J. Adaptive robust filters in MRI"; Proceedings of 
SPIE, SPIE, SPIE's International Symposium on Medical, 2002, 1028-1033.

\item Garcia, J. U., González-Santos, L., Favila, G. R., Rojas, R., Barrios, F.Three-dimensional MRI segmentation based on back-propagation neural network with robust supervised training"; Proceedings of SPIE, 
SPIE, SPIE's International Symposium on Medical, 2000, 817-824.

\item González-Santos, L., Correa, F. L-Sistemas y Gráficas de Fractales"; 7o Coloquio de Investigación ESFM-IPN, 1998, 247-254.

\item González-Santos, L., Reyes, M. A., Sossa-Azuela, J.H., Rojas, R., Salgado, P., Sánchez-Cortazar, J., Barrios, F. A. Segmentación 
Semiautomática por Crecimiento de Regiones con Estimación Robusta: Aplicación a Imágenes por Resonancia Magnética"; 
Congreso Latinoamericano de Ingeniaría, 1998, 9-12.

\item González-Santos, L., Rojas, R., Sossa-Azuela, J.H., Barrios, F. A. MRI Segmentation Based on Region Growing with Robust 
Estimation"; Proceedings of SPIE, SPIE, SPIE's International Symposium on Medical 
Poster, 1999, 880-885.

\item Ortiz, JJ, González-Santos, Rodríguez, Ll, Márquez, J., Barrios FA Brain atlas of children six to eight years old""; NeuroImage, 
Elsevier, OHBM 15th Annual Meeting, 2009, S152.

\item Romero-Meléndez, C, González-Santos, L. Planificación de Trayectorias y Complejidad Métrica en el Problema de los Cuerpos 
Rodantes"; Libro de Resúmenes, ISBN: 
976-956-319-926-0, CAIP, 9o Congreso Interamericano de computación Aplicada a la Industria de Procesos, 2009, 767-772.
\end{enumerate}

\textbf{Congresos, seminarios y conferencias}

\hfill

\textbf{Internacionales}

\begin{enumerate}
\item Reyes-Aguilar, A., Pasaye, E.H., González-Santos, L., Barrios, F.A., Poster: "Sex differences in Empathy for Positive and 
Negative 
Situations in Social Context", Modalidad: Presencial, 21 Annual Meeting of the Organization for Human Brain Mapping, OHBM, OHBM, País: 
Estados Unidos de América, Tipo: Congreso, Ámbito: Internacional, junio de 2015.

\item Vázquez Carrillo,D., Martínez-Soto, J., Alcauter, S., Pasaye E., González-Santos, L., Barrios, F.A., Poster: "Resting 
State 
Conectivity of the amygdala after the exposure to a stressful video", Modalidad: Presencial, Society for Neuroscience, Society for 
Neurocience, País: Estados Unidos de América, Tipo: Congreso, Ámbito: Internacional, octubre de 2015.

\item Martinez-Soto J., Barrios F.A., Gonzalez L., Pasaye E., Poster: Attentional patterns during the view of environments with low vs. 
high restorative potential, Modalidad: Presencial, OHBM Annual Meeting, País: Alemania, 
Tipo: Congreso, Ámbito: Internacional, junio de 2014.

\item Reyes-Aguilar A., Pasaye E., González-Santos L., Barrios F.A., Poster: Empathy in Cooperative and Non-Cooperative Context, 
Modalidad: Presencial, OHBM Annual Meeting, País: Alemania, Tipo: Congreso, Ámbito: Internacional, junio de 2014.

\item Rodríguez-Nieto, G., Mercadillo, R., Martínez-Soto, J., González-Santos, L., Barrios, F., Poster: Hemodynamic response and 
functional connectivity in women and men while experiencing compassion, Modalidad: Presencial, 19th Annual Meeting of the Organization 
for Human Brain Mapping, OHBM, País: Estados Unidos de América, Tipo: Congreso, Ámbito: Internacional, junio de 2013.

\item Martínez-Soto, González-Santos L., Barrios F.A., Poster: “Psychological Environmental Restoration: An Evaluation of some Neural 
Correlates”, Modalidad: Presencial, 22nd International Association People Environment Studies (IAPS), País: Reino Unido, Tipo: 
Congreso, Ámbito: Internacional, 2012.

\item Martínez-Soto J., González-Santos L., Pasaye E.H. and Barrios F.A., Poster: “Exploration of neural correlates of restorative 
environment exposure through fMRI”, Modalidad: Presencial, Academy of Neuroscience for Architecture, País: Estados Unidos de América, 
Tipo: Congreso, Ámbito: Internacional, 2012.

\item Moreno B., Concha L., González-Santos L., Ortiz J., Barrios F.A., Poster: “Cognitive and Executive Capacities Correlate with 
Corpus 
Callosum Area in Healthy Children”, Modalidad: Presencial, OHBM, País: China, Tipo: Congreso, Ámbito: Internacional, 2012.

\item Moreno M.B., Concha L., González-Santos L., Ortiz J.J., Barrios F.A., Ponencia: “Correlación entre regiones del Cuerpo Calloso 
(CC) 
y habilidades cognitivas en infantes sanos”, Modalidad: Presencial, VII Congreso Internacional Cerebro y Mente, I Congreso Antioqueño 
de Neurología y Neuropediatría, País: Colombia, Tipo: Congreso, Ámbito: Internacional, 2012.

\item Gonzalez, L., Ortiz, J., Pasaye, E., Barrios, F.A., Poster: "Etiquetando estructuras cerebrales en el espaciode Talairach, 
utilizando Imágenes de ResonanciaMagnética", Modalidad: Presencial, XIII Reunión de Neuroimagen, País: México, Tipo: Encuentro, Ámbito: 
Internacional, octubre de 2011.

\item Moreno, B., Concha, L., Gonzalez-Santos, L., Ortiz, J., Barrios, F., Poster: "Phonological Awareness correlate with Caudate 
Nucleus 
and Amygdala Volumes in Healthy Children", Modalidad: Presencial, 17th, Annual Meeting, País: Canadá, Tipo: Congreso, Ámbito: 
Internacional, 2011.

\item Beatriz Moreno, Luis Concha, Leopoldo González-Santos, Juan Ortiz, Fernando Barrios, Poster: "Intellect, Apparent Diffusion 
Coefficient, and width of Corpus Callosum in Healthy Children", Modalidad: Presencial, HBM2010, 16th Annual Meeting of the Organization 
for Human Brain Mapping, País: España, Tipo: Congreso, Ámbito: Internacional, junio de 2010.

\item Romero M.C., González-Santos L., Poster: “Curvas Elásticas en la Esfera y el Plano Hiperbólico”, Modalidad: Presencial, 1 er. 
Congreso Multidisciplinario de Ciencias Aplicadas en Latinoamerica, País: Nicaragua, Tipo: Congreso, Ámbito: Internacional, 2010.

\item Ortiz, JJ, González-Santos, L, Rodriguez, LI, Márquez, J., Barrios, FA, Poster: "Brain atlas of children six to eight years old", 
Modalidad: Presencial, OHBM 15th Annual Meeting, País: Estados Unidos de América, Tipo: Congreso, Ámbito: Internacional, junio de 2009.

\item Romero-Meléndez, C, González-Santos, L, Poster: "Curvas elásticas en superficies de cuvatura seccional constante", Modalidad: 
Presencial, III CLAM Congreso Latino Americano de Matemáticos, País: Chile, Tipo: Congreso, Ámbito: Internacional, septiembre de 2009.

\item Romero-Meléndez, C, González-Santos, L, Poster: "Planificación de Trayectorias y Complejidad Métrica en el Problema de los 
Cuerpos 
Rodantes", Modalidad: Presencial, 9o Congreso Interamericano de CAIP, País: Uruguay, Tipo: Congreso, Ámbito: Internacional, agosto de 
2009.

\item Ortiz, JJ, González-Santos, L, Mercadillo, RE, Barrios, FA, Poster: "MRI Children Atlas Based on a Robust Estimator", Modalidad: 
Presencial, Annual 16th Scientific Meeting, País: Estados Unidos de América, Tipo: Congreso, Ámbito: Internacional, mayo de 2008.

\item Barrios, F.A., Rodgers, C., Ortiz, J.J., González-Santos, L., Romero, A., Romero-Romo, J.J., Poster: "A VBM study in lower limb 
amputees", Modalidad: Presencial, 12th Anual Meeting Human Brain Mapping, País: 
Italia, Tipo: Congreso, Ámbito: Internacional, junio de 2006.

\item Romero-Romo, J.I., Rodgers, C., Ortiz, J.J., Gonzalez, L., Romero, A., Barrios, F. A., Poster: "Plasticity changes as 
reorganization of pre-motor cortex in lower limb amputees: a VBM study", Modalidad: Presencial, ISMRM 14th Scientific Meeting, País: 
Estados Unidos de América, Tipo: Congreso, Ámbito: Internacional, mayo de 2006.

\item Barrios, F.A., Ollervides, Marina A., Ortiz, Juan J., González L., and Romero, Jose A., Poster: "Simple activation tasks in fMRI 
at 
1.0 T, fail to produce good cross subject co-localization", Modalidad: Presencial, Thirteenth Scientific Meeting and Exhibition, País: 
Estados Unidos de América, Tipo: Congreso, Ámbito: Internacional, mayo de 2005.

\item Rodgers, C., Barrios, F., Ortiz, J. J., González-Santos, L., Romero, A., Romero-Romo, J., Poster: "Brain structural differences 
associated with phantom limb sensation: a voxel-based morphometry study", Modalidad: Presencial, ABRCMS, País: Estados Unidos de 
América, Tipo: Congreso, Ámbito: Internacional, noviembre de 2005.

\item Barrios, F. A., González-Santos, L., Favila, R., Fernández, J., Alonso, M., Salgado, P., Poster: "Behavioral deficits in 
Huntington’s disease correlate with tissue differences measured with MRI", Modalidad: Presencial, ISMRM, País: Japón, Tipo: Congreso, 
Ámbito: Internacional, mayo de 2004.

\item Morales, F.M., Paredes, González-Santos, L., Mejia, C., Varela-Echavarría, A., Larriva-Sahd, J., Poster: "Sexual dimorphism gene 
expression in neonate hypothalamus", Modalidad: Presencial, 13th of Neuroscience: From molecules to behavior, País: Estados Unidos de 
América, Tipo: Congreso, Ámbito: Internacional, noviembre de 2004.

\item Torrero, C., Medina, I., Gutiérrez, G., Regalado, M., González-Santos, L., Salas, M, Poster: "Interaction between undernutrition 
and unilateral olfactory occlusion upon synaptic vesicles (Sv) of olfactory bulb (Ob) in the rat", Modalidad: Presencial, 1st 
International Meeting of Latin American Society of Developmental Biologists, País: Chile, Tipo: Congreso, Ámbito: Internacional, junio 
de 2003.

\item Barrios, F., González-Santos, L., Favila, R., Rojas, R., Sánchez-Cortazar, J., Poster: "Adaptive robust filters in MRI", 
Modalidad: 
Presencial, SPIE's International Symposium on Medical Imaging, País: Estados Unidos de América, Tipo: Congreso, Ámbito: Internacional, 
febrero de 2002.

\item García, J. U., González-Santos, L., Favila, R., Rojas, R., Barrios, F. A., Poster: "3D MRI segmentation based on backpropagation 
neural network with robust supervised training", Modalidad: Presencial, SPIE's International Symposium on Medical, País: Estados Unidos 
de América, Tipo: Congreso, Ámbito: Internacional, febrero de 2000.

\item González-Santos, L., Rojas, R., Sossa-Azuela, J.H., Barrios, F., Poster: "MRI Segmentation Based on Region Growing with Robust 
Estimation", Modalidad: Presencial, SPIE's International Symposium on Medical Imaging, País: México, Tipo: Congreso, Ámbito: 
Internacional, 1999.

\item Barrios, F., González-Santos, L., Rojas, R., Salgado, P., Sánchez-Cortazar, J., Reynoso-Padilla, G., Poster: "Automatic Image 
Segmentation of 3D MRI Images, Based on 3D Robust Statistical Filters", Modalidad: Presencial, RSNA, País: Estados Unidos de América, 
Tipo: Congreso, Ámbito: Internacional, 1998.

\item González-Santos, L., Reyes, M.A., Sossa, J.H., Rojas, R., Salgado, P., Sánchez-Cortazar J., Barrios, F.A., Ponencia: 
"Segmentación 
semiautomática por crecimient de regiones con estimación robusta. Aplicación a imágenes por resonancia magnética", Modalidad: 
Presencial, 1er. Congreso Latinoamericano de Ingeniería Biomédica, País: México, Tipo: Congreso, Ámbito: Internacional, 1998.
\end{enumerate}

Nacionales

1. Martínez-Soto, J., González-Santos, L., Ponencia: Influencias Ambientales en la Restauración Psicológica, Modalidad: Presencial, 
XXXVI Congreso Interamericano de Psicología, País: México, Tipo: Congreso, Ámbito: Nacional, julio de 2017.

2. Martínez Soto, J., Gonzalez--Santos, L. y Barrios, F.A., Poster: CORRELATOS NEURALES Y PSICOLOGICOS DE LA PERCEPCION 
DE AMBIENTES RESTAURADORES, Modalidad: Presencial, “Corteza Prefrontal: Cognición y Conducta”, País: México, Tipo: Otro, Ámbito: 
Nacional, noviembre de 2013.

3. Martínez Soto, J., Gonzalez-Santos, L. y Barrios, F.A., Poster: Paradigma experimental de restauración psicológica ambiental con 
resonancia magnética funcional, Modalidad: Presencial, XV REUNION DE NEUROIMAGENES, País: México, Tipo: Encuentro, Ámbito: Nacional, 
noviembre de 2013.

4. Nanni-Zepeda. M., Martinez- Soto, J., Barrios-Alvarez, F., Pasaye-Alcaraz E., Gonzalez-Santos L., Ponencia: Correlatos neurales de 
la observación natural de un video, Modalidad: Presencial, XV REUNION DE NEUROIMAGENES, Centro de Investigación en Matemáticas AC, 
CIMAT, País: México, Tipo: Congreso, Ámbito: Nacional, octubre de 2013.

5. Reyes-Aguilar, A., Barrios Álvarez, A. F., González-Santos, L., Pasaye Alcaraz E., Ponencia: CORRELATOS NEURONALES DE LA EMPATÍA EN 
CONTEXTOS DE JUSTICIA E INJUSTICA, Modalidad: Presencial, XV REUNION DE NEUROIMAGENES, País: México, Tipo: Congreso, Ámbito: Nacional, 
octubre de 2013.

6. González-Santos L., Méndez O., Angulo-Perkins A., Ortiz-Retana J., López-Sánchez L., Barrios F.A., Concha L., Poster: “Examinando la 
Microestructura Cerebral del Conejo utilizando Análisis Tensorial de cortes histológicos”, Modalidad: Presencial, XIX Jornadas 
Académicas del Instituto de Neurobiología, UNAM, Instituto de Neurobiología UNAM-Juriquilla, INB-UNAM, INB-UNAM, País: México, Tipo: 
Jornadas, Ámbito: Nacional, 2012.

7. Martínez-Soto J., González-Santos L. Y Barrios F.A., Poster: “Bioetica y Neurosciencias para la Promoción de la Salud Humana”, 
Modalidad: Presencial, XIX Jornadas Académicas del Instituto de Neurobiología, UNAM, Instituto de Neurobiología UNAM-Juriquilla, 
INB-UNAM, INB-UNAM, País: México, Tipo: Jornadas, Ámbito: Nacional, 2012.

8. Romero Meléndez C., González-Santos L., Poster: “El problema de Dido”, Modalidad: Presencial, 45 Congreso Nacional de la Sociedad 
Matemática Mexicana, País: México, Tipo: Congreso, Ámbito: Nacional, 2012.

9. Romero Meléndez C., González-Santos L., Ponencia: “Generación de trayectorias para sistemas diferencialmente planos”, Modalidad: 
Presencial, 45 Congreso Nacional de la Sociedad Matemática Mexicana, País: México, Tipo: Congreso, Ámbito: Nacional, 2012.

10. Rodgers, CM, Barrios, FA, Ortiz JJ, Gonzalez-Santos, L, Romero A, Romero-Romo, J, Poster: "Brain Structural Differences Associated 
with Phantom Limb Sensation: A Voxel-Based Morphometry", Modalidad: Presencial, Annual Biomedical Research Conference for Minority 
Studens, País: Estados Unidos de América, Tipo: Congreso, Ámbito: Nacional, 2005.

11. Caicedo, R.E., Quintanar, A., Gutierrez-Ospina, G., González, S.L., Martínez de la Escalera, G., y Clapp, C. , Poster: "Efecto de 
los estrogenos sobre la vascularización del utero en la rata", Modalidad: Presencial, Congreso Nacional de Ciencias Fisiológicas, País: 
México, Tipo: Congreso, Ámbito: Nacional, 1999.

Locales

1. González Santos, L., González Pérez, E. and Alcauter, S, Poster: IMPROVING INTENSITYBIASED VIDEOS TO EVALUATE THE MORRIS WATER MAZE 
TEST, Modalidad: Presencial, XXIX Jornadas Académicas del Instituto de Neurobiología, Instituto de Neurobiología UNAM-Juriquilla, 
INB-UNAM, INB-UNAM, País: México, Tipo: Jornadas, Ámbito: Local, septiembre de 2022.

2. Ayala-Sumuano, J.T., López-Victorio, C.J., Lozano-Flores, C., González-Gallardo, A., GonzálezSantos, L., Wray, G., 
Varela-Echavarría, A., Poster: "Alta carga de mutaciones en el ADN mitocondrial: Ejemplo de un genoma dinámicamente estable", 
Modalidad: Presencial, 22 Jornadas Académicas, Instituto de Neurobiología UNAM-Juriquilla, INB-UNAM, INB-UNAM, País: México, Tipo: 
Jornadas, Ámbito: Local, octubre de 2015.

3. López-Victorio, C.J., Ayala-Sumuano, J., Lozano-Flores, C., González-Santos, L., Linares-Belman, P.B., Varela 
Echavarría, Poster: Analísis comparativo del transcriptoma del talencéfalo de ratón y pollo durante el desarrollo embrionario", 
Modalidad: Presencial, 22 Jornadas Académicas, Instituto de Neurobiología UNAM-Juriquilla, INB-UNAM, INB-UNAM, País: México, Tipo: 
Jornadas, Ámbito: Local, octubre de 2015.

4. Vázques, D, Martínez-Soto, J, Barrios, F.A., González-Santos, L., Pasaye, E., Alcauter, S., Poster: "Conectividad Funcional de la 
amígdala de reposo después de un evento estresante", Modalidad: Presencial, 22 Jornadas Académicas, Instituto de Neurobiología 
UNAM-Juriquilla, INB-UNAM, INBUNAM, País: México, Tipo: Jornadas, Ámbito: Local, octubre de 2015.

5. López-Victorio C.J., Ayala-Sumuano J., Lozano-Flores C., González-Santos L., Echavarría V., Poster: ANÁLISIS COMPARATIVO DEL 
TRANSCRITÓMA DEL TELENCÉFALO DE RATÓN Y POLLO DURANTE EL DESARROLLO EMBRIONARIO, Modalidad: Presencial, Jornada Académicas, País: 
México, Tipo: Jornadas, Ámbito: Local, septiembre de 2014.

6. Reyes-Aguilar A., González-Santos P., Pasaye E. y Barrios F.A.,, Poster: Empatía En Contextos De Cooperación Y No-Cooperación, 
Modalidad: Presencial, Jornada Académicas, Instituto de Neurobiología UNAM-Juriquilla, INB-UNAM, INB-UNAM, País: México, Tipo: 
Jornadas, Ámbito: Local, septiembre de 2014.

7. González-Santos L., Ortiz, J. J., Pasaye E. H., Concha L., Barrios F.A., Ponencia: "LOCALIZADOR DIGITAL DE ESTRUCTURAS ANATÓMICAS EN 
IMÁGENES PORRESONANCIA MAGNÉTICA REFERIDAS A UN ESPACIO ESTEROTAXICO", Modalidad: Presencial, Jordanas Académicas, País: México, Tipo: 
Jornadas, Ámbito: Local, septiembre de 2011.

8. Martínez-Soto, Joel; Gonzales-Santos, L. y Barrios, F. A., Poster: "OBTENCIÓN DE LAS CUALIDADES AFECTIVAS DE TRES TIPOS DE 
AMBIENTES", Modalidad: Presencial, Jornadas Académicas, País: México, Tipo: Jornadas, Ámbito: Local, septiembre de 2011.

9. Ortiz-Retana J.J., González-Santos L., Rodríguez Ll., Márquez J., Ricardo J., Mercadillo R.E., Barrios F.A., Poster: ATLAS CEREBRAL 
PROMEDIO DE ONCE NIÑOS Y ONCE NIÑAS DE SEIS A OCHO AÑOS DE EDAD, Modalidad: Presencial, xvi Jornadas Conmemorativas, País: México, 
Tipo: Congreso, Ámbito: Local, septiembre de 2009.

10. González-Santos, L., Ortiz, J.J., Barrios, F.A., Poster: "Resonancia Funcional por relación a evento a 1.0 T", Modalidad: 
Presencial, Jornadas , País: México, Tipo: Jornadas, Ámbito: Local, octubre de 2006.

11. Silis, J.J., González-Santos L., Larriva-Sahd, Barrios, F.A., Poster: "Segmentación y cantificación automatica de estructuras de 
interés a partir de imágenes digtales del microscopio electrónico", Modalidad: Presencial, Jornadas, País: México, Tipo: Congreso, 
Ámbito: Local, septiembre de 2006.

12. González-Santos L., Barrios A.,F., Poster: "Segmentación de Imagen Digital", Modalidad: Presencial, Jornadas del Instituto de 
Neurobiología, País: México, Tipo: Jornadas, Ámbito: Local, 2003.

13. González-Santos, L., Correa, F., Taller: "L-Sistemas y Gráficas de Fractales", Modalidad: Presencial, Reunión Académica de Física y 
Matemáticas, País: México, Tipo: Encuentro, Ámbito: Local, agosto de 1998.

Comités de congresos

1. V Mexican Symposium on Medical Physics, País: México, Tipo: Simposio, Ámbito: Internacional, marzo de 2001.

Docencia

Cursos regulares

1. “Introducción al lenguaje de Programación R”, Modalidad: Presencial, Semestre: 1, Maestría en Ciencias 
Neurobiología, Maestría, Instituto de Neurobiología, Instituto de Neurobiologia, Tipo de programa académico: Escolarizado, Año del 
programa: 2023, Periodos: 2024-2, 2024-1.

2. “Introducción al lenguaje de Programación R”, Modalidad: Presencial, Semestre: 2, Maestía en Ciencias neurobiología, Maestría, 
Instituto de Neurobilogía, Instituto de Neurobiologia, Tipo de programa académico: Escolarizado, Año del programa: 2023, Periodos: 
2024-1.

3. “Introducción a la programación y computación con Phyton”, Modalidad: Presencial, Semestre: 2, Maestria w¡en Ciencias Neurobiología, 
Maestría, Instituto Neurobiología, Instituto de Neurobiologia, Tipo de programa académico: Escolarizado, Año del programa: 2023, 
Periodos: 2023-2.

4. “Introducción al lenguaje de Programación R”, Modalidad: Presencial, Semestre: 2, Maestría en Ciencias, Neurobiología, Maestría, 
Instituto de Neurobiologia, Instituto de Neurobiologia, Tipo de programa académico: Escolarizado, Año del programa: 2023, Periodos: 
2023-2.

5. Introducción a la Computación y Lenguaje de Programación R, Modalidad: A Distancia, Semestre: 1, Maestría en Ciencias 
(Neurobiología), Maestría, Instituto de Neurobiología, Instituo de Neurobiologia, Tipo de programa académico: Escolarizado, Año del 
programa: 2022, Periodos: 20222, 2022-1.

6. “Introducción a la Computación y Lenguaje de Programación PYTHON”, Modalidad: A Distancia, Semestre: 1, Maestría en Neurobiología, 
Maestría, Neurobiología, Instituo de Neurobiologia, Tipo de programa académico: Escolarizado, Año del programa: 2021, Periodos: 2021-2.

7. “Introducción a la Computación y Lenguaje de Programación R”, Modalidad: A Distancia, Semestre: 1, Maestría en Neurobiología, 
Maestría, Neurobiología, Instituo de Neurobiologia, Tipo de programa académico: Escolarizado, Año del programa: 2021, Periodos: 2021-2.

8. ‘Introducción a la Computación y Lenguaje de Programación “R”’, Modalidad: Presencial, Semestre: 2, Maestría en Ciencias 
(Neurobiología), Maestría, Instituto de Neurobiología UNAMJuriquilla, Instituto de Neurobiología, Tipo de programa académico: 
Escolarizado, Año del programa: 2019, Periodos: 2021-1, 2020-2, 2020-1, 2019-2.

9. Introducción al Lenguaje de Programación "R", Modalidad: Presencial, Semestre: 2, Programa Maestría en Ciencias (Neurobiología), 
Maestría, Instituto de Neurobiología UNAM-Juriquilla, INBUNAM, Instituto de Neurobiología UNAM-Juriquilla, Tipo de programa académico: 
Escolarizado, Año del programa: 2018, Periodos: 2019-1.

10. Introducción al Lenguaje de Programación "R", Modalidad: Presencial, Semestre: 1, Programa Maestría en Ciencias (Neurobiología), 
Maestría, Instituto de Neurobiología UNAM-Juriquilla, INBUNAM, Instituto de Neurobiología UNAM-Juriquilla, Tipo de programa académico: 
Escolarizado, Año del programa: 2018, Periodos: 2018-2.

11. “Introducción al Lenguaje de Programación Python”, Modalidad: Presencial, Semestre: 1, Maestría en Ciencias (Neurobiología), UNAM, 
Doctorado, Instituto de Neurobiología UNAMJuriquilla, INB-UNAM, Instituto de Neurobiología UNAM-Juriquilla, Tipo de programa académico: 
Escolarizado, Año del programa: 2018, Periodos: 2018-1.

12. “Introducción al Lenguaje de Programación R”, Modalidad: Presencial, Semestre: 2, Maestría en Ciencias (Neurobiología), Doctorado, 
Instituto de Neurobiología UNAM-Juriquilla, INB-UNAM, Instituto de Neurobiología UNAM-Juriquilla, Tipo de programa académico: 
Escolarizado, Año del programa: 2017, Periodos: 2017-2.

13. “Métodos Numéricos”, Modalidad: Presencial, Semestre: 2, Maestría en Ciencias (Neurobiología), Doctorado, Instituto de 
Neurobiología UNAM-Juriquilla, INB-UNAM, Instituto de Neurobiología UNAM-Juriquilla, Tipo de programa académico: Escolarizado, Año del 
programa: 2017, Periodos: 2017-1.

14. Métodos Numéricos, Modalidad: Presencial, Semestre: 1, Créditos: 4, Maestría en Ciencias (Neurobiología), Maestría, Instituto de 
Neurobiología UNAM-Juriquilla, INB-UNAM, Instituto de Neurobiología UNAM-Juriquilla, Tipo de programa académico: Escolarizado, Año del 
programa: 2016, Periodos: 2017-1.

15. Estadística, Modalidad: Presencial, Semestre: 2, Créditos: 8, Maestría en Ciencias (Neurobiología), Maestría, Instituto de 
Neurobiología UNAM-Juriquilla, INB-UNAM, Instituto de Neurobiología 
UNAM-Juriquilla, Tipo de programa académico: Escolarizado, Año del programa: 2016, Periodos: 2016-2.

16. Estadística, Modalidad: Presencial, Semestre: 2, Maestría en Ciencias (Neurobiología), Maestría, Instituto de Neurobiología 
UNAM-Juriquilla, INB-UNAM, Instituto de Neurobiología UNAM-Juriquilla, Tipo de programa académico: Escolarizado, Año del programa: 
2015, Periodos: 2015-2.

17. Métodos Numéricos, Modalidad: Presencial, Semestre: 2, Maestría en Ciencias (Neurobiología), Maestría, Instituto de Neurobiología 
UNAM-Juriquilla, INB-UNAM, Instituto de Neurobiología UNAMJuriquilla, Tipo de programa académico: Escolarizado, Año del programa: 2015, 
Periodos: 2015-2.

18. Estadística, Modalidad: Presencial, Semestre: 2, Programa de Maestría en Ciencias (Neurobiología), Maestría, Instituto de 
Neurobiología UNAM-Juriquilla, INB-UNAM, Instituto de Neurobiología UNAM-Juriquilla, Tipo de programa académico: Escolarizado, Año del 
programa: 2014, Periodos: 2015-1.

19. Lenguaje de Programación MatLab, Modalidad: Presencial, Semestre: 1, Maestría en Ciencias (Neurobiología), Maestría, Instituto de 
Neurobiología UNAM-Juriquilla, INB-UNAM, Instituto de Neurobiología UNAM-Juriquilla, Tipo de programa académico: Escolarizado, Año del 
programa: 2012, Periodos: 2012-1.

20. Lenguaje de Programación Matlab, Modalidad: Presencial, Semestre: , Maestría en Ciencias (Neurobiología), Maestría, Instituto de 
Neurobiología UNAM-Juriquilla, INB-UNAM, Instituto de Neurobiología UNAM-Juriquilla, Tipo de programa académico: Escolarizado, Año del 
programa: 2009, Periodos: 2008-1.

21. Lenguaje Pascal y “C”, Modalidad: Presencial, Semestre: 2, Ingeniero en Computación, Licenciatura, Tecnológico de Estudios 
Superiores de Ecatepec, Tipo de programa académico: Escolarizado, Año del programa: 1996, Periodos: Indefinido.

22. Mecánica, Modalidad: Presencial, Semestre: 2, Ingeniero Químico, Licenciatura, Facultad de Química de la UAQ. Querétaro, México, 
Tipo de programa académico: Escolarizado, Año del programa: 2001, Periodos: Indefinido.

23. Mecánica, Modalidad: Presencial, Semestre: 3, Fisíco, Licenciatura, Facultad de Química de la UAQ. Querétaro, México, Tipo de 
programa académico: Escolarizado, Año del programa: 2001, Periodos: Indefinido.

24. Taller de Matemáticas, Modalidad: Presencial, Semestre: , Maestría en Ciencias (Neurobiología), Maestría, Instituto de 
Neurobiología UNAM-Juriquilla, INB-UNAM, Instituto de Neurobiología UNAMJuriquilla, Tipo de programa académico: Escolarizado, Año del 
programa: 1997, Periodos: Indefinido.

25. Lenguaje de Programación MatLab, Modalidad: Presencial, Semestre: 1, Maestría en Ciencias (Neurobiología), UNAM, Maestría, 
Instituto de Neurobiología UNAM-Juriquilla, INB-UNAM, Instituto de Neurobiología UNAM-Juriquilla, Tipo de programa académico: 
Escolarizado, Año del programa: 2010, Periodos: 2010-1.

Cursos especiales

1. “Análisis Estadístico Multivariado”, Duración: Semanal, Modalidad: Presencial, Horas por semana: 24, Total de horas: 24, Instituto 
de Neurobiología, UNAM, País: México, Fecha de inicio: enero de 2020, Fecha de conclusión: enero de 2020.

2. “Introducción a la Computación y Programación del Lenguaje R con ESTADÍSTICA”, Duración: Semanal, Modalidad: Presencial, Horas por 
semana: 24, Total de horas: 24, Instituto de Neurobiología, UNAM, País: México, Fecha de inicio: enero de 2020, Fecha de conclusión: 
enero de 2020.

3. “Introducción a la Computación y Programación en Lenguaje PYTHON”, Duración: Semanal, Modalidad: Presencial, Horas por semana: 24, 
Total de horas: 24, Instituto de Neurobiología, UNAM, País: México, Fecha de inicio: enero de 2020, Fecha de conclusión: enero de 2020.

4. “Curso teórico práctico de Resonancia Magnética funcional”, Duración: Semanal, Modalidad: Presencial, Horas por semana: 16, Total de 
horas: 16, Instituto de Neurobiología-UNAM, País: México, Fecha de inicio: 
diciembre de 2019, Fecha de conclusión: diciembre de 2019.

5. “Introducción al Lenguaje de Programación ‘R’ con Estadística”, Duración: Semanal, Modalidad: Presencial, Horas por semana: 20, 
Total de horas: 20, Instituto de Neurobiología-UNAM, País: México, Fecha de inicio: junio de 2019, Fecha de conclusión: junio de 2019.

6. “Introducción al Lenguaje de Programación R con Estadística”, Duración: Semanal, Modalidad: Presencial, Horas por semana: 30, Total 
de horas: 30, Instituto de Neurobiología, UNAM, País: México, Fecha de inicio: junio de 2017, Fecha de conclusión: junio de 2017.

7. “Análisis Estadístico Multivariado con R”, Duración: Semanal, Modalidad: Presencial, Horas por semana: 30, Total de horas: 30, 
Instituto de Neurobiología, UNAM, País: México, Fecha de inicio: enero de 2017, Fecha de conclusión: enero de 2017.

8. “Introducción al Lenguaje de Programación R con Estadística”, Duración: Semanal, Modalidad: Presencial, Horas por semana: 30, Total 
de horas: 30, Instituto de Neurobiología, UNAM, País: México, Fecha de inicio: enero de 2017, Fecha de conclusión: enero de 2017.

9. “Manejo de Grandes Volúmenes de Datos Utilizando Herramientas LINUX”, Duración: Semanal, Modalidad: Presencial, Horas por semana: 
30, Total de horas: 30, Instituto de Neurobiología, UNAM, País: México, Fecha de inicio: enero de 2017, Fecha de conclusión: enero de 
2017.

10. Introducción al Lenguaje de Programación R con Estadística, Duración: Semanal, Modalidad: Presencial, Horas por semana: 30, Total 
de horas: 30, Instituto de Neurobiología - UNAM, País: México, Fecha de inicio: junio de 2016, Fecha de conclusión: julio de 2016.

11. Manejo de Grandes Volúmenes de Datos Utilizando Herramientas LINUX (Instr), Duración: Semanal, Modalidad: Presencial, Horas por 
semana: 30, Total de horas: 12, Instituto de Neurobiología - UNAM, País: México, Fecha de inicio: junio de 2016, Fecha de conclusión: 
junio de 2016.

12. Introducción al Lenguaje de Programación R con Estadística, Duración: Semanal, Modalidad: Presencial, Horas por semana: 30, Total 
de horas: 30, Instituto de Neurobiología - UNAM, País: México, Fecha de inicio: enero de 2016, Fecha de conclusión: enero de 2016.

13. Manejo de Grandes Volúmenes de Datos Utilizando Herramientas LINUX (Instr), Duración: Semanal, Modalidad: Presencial, Horas por 
semana: 30, Total de horas: 12, Instituto de Neurobiología - UNAM, País: México, Fecha de inicio: enero de 2016, Fecha de conclusión: 
enero de 2016.

14. R CON ESTADÍSTICA", Duración: Semanal, Modalidad: Presencial, Horas por semana: 30, Total de horas: 30, Instituto de Neurobiología, 
UNAM, País: México, Fecha de inicio: junio de 2015, Fecha de conclusión: julio de 2015.

15. R con ESTADISTICA, Duración: Semanal, Modalidad: Presencial, Horas por semana: 15, Total de horas: 30, Instituto de Neurobiología, 
País: México, Fecha de inicio: julio de 2014, Fecha de conclusión: agosto de 2014.

16. R con ESTADISTICA, Duración: Semanal, Modalidad: Presencial, Horas por semana: 15, Total de horas: 30, Instituto de Neurobiología, 
UNAM, País: México, Fecha de inicio: enero de 2014, Fecha de conclusión: enero de 2014.

17. SHELL – UNIX, Duración: Semanal, Modalidad: Presencial, Horas por semana: 15, Total de horas: 30, Instituto de Neurobiología, UNAM, 
País: México, Fecha de inicio: enero de 2014, Fecha de conclusión: enero de 2014.

18. “Introducción al lenguaje R con Estadística”, Duración: Semanal, Modalidad: Presencial, Horas por semana: 20, Total de horas: 20, 
Instituto de Neurobiología, País: México, Fecha de inicio: julio de 2012, Fecha de conclusión: julio de 2012.

19. Matemáticas para el Análisis de Imagen Digital, Duración: Semanal, Modalidad: Presencial, Horas por semana: 12, Total de horas: 12, 
, País: México, Fecha de inicio: octubre de 2008, Fecha de conclusión: octubre de 2008.

20. SPSS, Análisis Estadístico", Duración: Semanal, Modalidad: Presencial, Horas por semana: 20, Total de horas: 20, Juriquilla, QRO, 
País: México, Fecha de inicio: noviembre de 2006, Fecha de conclusión: noviembre de 2006.

21. Análisis multivariado", Duración: Semanal, Modalidad: Presencial, Horas por semana: 16, Total de horas: 16, Juriquilla, Qro, País: 
México, Fecha de inicio: septiembre de 2006, Fecha de conclusión: septiembre de 2006.

Tesis de licenciatura

1. Solis Silva, U.I. (estudiante), “Control aplicado a un modelo Lotka-Volterra con variaciones aleatorias”, Modalidad: Tesina, 
Ciencias Básica, Licenciatura, Universidad Autónoma Metropolitana, UAM, Fecha de presentación de examen: 20/11/2024, González Santos 
Leopoldo (Director).

2. González Espinosa, V. (estudiante), “Control óptimo en un modelo de poblaciones de tipo presadepredador”, Modalidad: Tesina, 
Ciencias Básica, Licenciatura, Universidad Autónoma Metropolitana, UAM, Fecha de presentación de examen: 20/08/2024, González Santos 
Leopoldo (Director).

3. Jiménez Hernández Noemi (estudiante), "Segmentación y cuantificación del cuerpo calloso en imagen de resonancia magnética en 
Matlab", Modalidad: Tesina, Técnico Superior Universitario en Tecnologías de la Información y Comunicación, Licenciatura, Universidad 
Tecnológica de Querétaro, UTEQ, Fecha estimada de presentación y obtención de grado: 01/11/2009, González Santos Leopoldo (Director).

4. Silis García, C. José (estudiante), "Segmentación y cuantificación automática de estructuras de interés a partir de imágenes 
digitales del microscopio digital", Modalidad: Tesis, Ingeniero en Sistemas Computacionales, Licenciatura, Instituto Tecnológico de 
Querétaro, Querétaro, Fecha estimada de presentación y obtención de grado: 01/10/2006, González Santos Leopoldo (Director).

Actividades técnico-académicas

Productos técnicos

1. González-Santos, L, Mercadillo, RE, Barrios, FA, Versíon Computarizada para la aplicación del SCL90 e ITC, Tipo de trabajo: 
Software, Status: Entregado, 2008.

Otras actividades o productos

Actividades de difusión y extensión

1. Caravana del Cerebro: 25 de abril, Tipo: Semana del cerebro, abril de 2024.

2. Neuro-Emociónate", CAC-UNAM, del 19 al 23: , Tipo: Semana del Cerebro, marzo de 2024.

3. Semana del Cerebro: Centro Educativo y Cultural ‘Manuel Gómez Morin’, Querétaro, Qro., 19-23 marzo, Tipo: Semana del cerebro, marzo 
de 2019.

4. ¡Muévete ... Mantén activo tu Cerebro!: Semana del Cerebro, Tipo: En la semana del cerebro, marzo de 2016.

5. Feria de las Ciencias y las Humanidades: El cerebro del buen comer, Tipo: Conferencia, octubre de 2013.

6. “¿Qué tengo en mi cabeza?: Participación de Carteles, Tipo: Semana del cerebro, marzo de 2012.

7. Semana del Cerebro, CerebrAte: , Tipo: En la semana del cerebro, marzo de 2011.

8. Juego de Rompecabezas del Cerebro": Participación en las actividades de la Semana del Cerebro, Marzo, 16-20, Tipo: En la semana del 
cerebro, 2010.

9. Puedes ver los que pasa en tu cerebro...Sí!: La Sociedad Mexicana de Ciencias Fisiológicas y El Capítulo de la Ciudad de México de 
la "Society for Neuroscience",Marzo, 15-21, Tipo: Semana del Cerebro, 2010.

10. Organización de la "Celebración Internacional del Cerebro": , Tipo: Semana del Cerebro, abril de 2006.

11. Participación en la "Celebración Internacional del Cerebro": , Tipo: Semana del Cerebro, marzo de 2005.

\end{document}

