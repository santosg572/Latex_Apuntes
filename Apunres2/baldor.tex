\documentclass[a4paper,10pt]{article}

%%%%%%%%%%%% PREÁMBULO %%%%%%%%%%%%%%%%%%%%%

% Paquetes

\usepackage[utf8]{inputenc}
\usepackage[spanish,es-tabla]{babel}
\usepackage[T1]{fontenc}

\usepackage{listings}

% Comandos

\renewcommand{\lstlistingname}{Código}
\renewcommand{\lstlistlistingname}{Índice de fragmentos de código fuente}

% Opciones

\title{Python 2.*}
\author{Ondiz Zarraga}

%%%%%%%%%%%%%%%%%%%%%%%%%%%%%%%%%%%%%%%%%%%%%%

\begin{document}
\maketitle

\begin{abstract}
Este documento es una pequeña guía de Python 
\end{abstract}

\tableofcontents

\section{Sobre el lenguaje}

\textbf{baldor E12}

\hfill

Hallar el valor numérico de las expresiones siguientes para

$$
a = 3, b=4, c=\frac{1}{3}, d=\frac{1}{2}, m = 6, n=\frac{1}{4}
$$

\begin{enumerate}
\item $\frac{a^2}{3} - \frac{b^2}{2} + \frac{m^2}{6}$

\item $\frac{ab}{n} + \frac{ac}{d} - \frac{bd}{m}$

\item $c \sqrt{3a} - d \sqrt{16b^2} + n \sqrt{8d}$

\item  $\frac{12c-a}{2b} - \frac{16n-a}{m} + \frac{1}{d}$

\item  $\frac{\sqrt{b} + \sqrt{2d}}{2} - \frac{\sqrt{3c}+\sqrt{8d}}{4}$
\end{enumerate}

\textbf{ E13}

\hfill

Har el valor numérico de las expresiones siguientes para

$$
a = 1, b=2, c=3, d=4, m = \frac{1}{2}, n=\frac{2}{5}, p=\frac{1}{4}, x=0
$$

\begin{enumerate}
\item $(2m+3n)(4p+b^2)$

\item  $2mx + 6(b^2+c^2)-4d^2$

\item  $(\frac{8m}{9n} + \frac{16p}{b})a$

\item  $\frac{4(m+p)}{a} + \frac{a^2 + b^2}{c^2}$
\end{enumerate}





\end{document}

