4
Graphical Analysis

4.1
Graphical Analysis

4.2
Orbit Analysis

4.3
The Phase Portrait

Exercises
1. Use graphical analysis to describe the fate of all orbits for each of the
following functions. Use different colors for orbits that behave differently.

a. F (x) = 2x
b. F (x) = 1/3x
c. F (x) = -2x + 1
d. $F(x) = x^2$
e. $F(x) = -x^3$
f. $F(x) = x - x^2$
g. $S(x) = \sin x$

2. Use graphical analysis to find $\{x0 | F^n(x_0 ) →  ±∞}$ for each of the following
functions.

a. F (x) = 2x(1 − x)
b. $F(x) = x^2 + 1$
c. T (x) =
2x
2 − 2x
x ≤ 1/2
x > 1/2

3. Sketch the phase portraits for each of the functions in exercise 1.
4. Perform a complete orbit analysis for each of the following functions.

a. $F(x) = \frac{1}{2} x - 2$
b. A(x) = |x|
c. $F(x) = -x^2$
d. $F(x) = -x^5$
e. F (x) = 1/x
f. $E(x) = e^x$

5. Let F (x) = |x - 2|. Use graphical analysis to display a variety of orbits of
F . Use red to display cycles of period 2, blue for eventually fixed orbits, and
green for orbits that are eventually periodic.

6. Consider $F(x) = x^2 - 1.1$. First find the fixed points of F . Then use the
fact that these points are also solutions of $F^2(x) = x$ to find the cycle of prime
period 2 for F .

7. All of the following exercises deal with dynamics of linear functions of the
form F (x) = ax + b where a and b are constants.

a. Find the fixed points of F (x) = ax + b.
b. For which values of a and b does F have no fixed points?
c. For which values of a and b does F have infinitely many fixed points?
d. For which values of a and b does F have exactly one fixed point?
e. Suppose F has just one fixed point and 0 < |a| < 1. Using graphical
analysis, what can you say about the behavior of all other orbits of
F ? We will call these fixed points attracting fixed points later. Why
do we use this terminology?
f. What is the behavior of all orbits when a = 0?
g. Suppose F has just one fixed point and |a| > 1. Using graphical
analysis, what can you say about the behavior of all other orbits of
F in this case? We call such fixed points repelling. Can you explain
why?
h. Perform a complete orbit analysis for F (x) = x + b in cases $b > 0$,
b = 0, and $b < 0$.
i Perform a complete orbit analysis for F (x) = -x + b.


