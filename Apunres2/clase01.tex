\documentclass[12pt, letterpaper, twoside]{article}
\usepackage[utf8]{inputenc}
\usepackage[spanish]{babel}

%\usepackage{cite} % para contraer referencias
%\usepackage[alf]{abntex2cite}

\title{First document}
\author{Hubert Farnsworth \thanks{funded by the Overleaf team}}
\date{February 2014}

\begin{document}

\begin{titlepage}
\maketitle
\end{titlepage}


\hrule

\hfill


libro: \cite{Devaney}

\hfill

y la mayor parte de la investigación relacionada con estos objetos en ese entonces se centró en la iteración de funciones cuadráticas. 
Hoy en día, esta investigación se ha expandido para incluir todo tipo de diferentes tipos de funciones, incluidos polinomios de grado 
superior, mapas racionales, funciones exponenciales y trigonométricas, y muchas otras.



\hfill

La belleza de esta materia es que el único requisito previo para los estudiantes es un curso de cálculo de un año (no se requieren ecuaciones diferenciales). En su mayor parte en este libro, nos concentramos en la iteración de funciones cuadráticas reales o complejas de la forma x2 + c. 

\hfill

\hrule

\hfill


libro: \cite{MELLODGE:2016}

 Las aplicaciones prevalecen en la ingeniería mecánica, eléctrica y biomédica y se pueden encontrar en sistemas robóticos, automotrices, aeroespaciales y humanos, entre otros.


\hfill

Cada concepto abstracto se discute en profundidad, se describe de manera legible y práctica, y se ilustra con ejemplos prácticos.

\hfill

Se supone que los lectores tienen una base matemática sólida en cálculo, ecuaciones diferenciales y teoría de matrices.

Al presentar el material, el énfasis está en la aplicación de la teoría, por lo que hay relativamente pocos teoremas y ninguna prueba. 

\hfill

El motivo de este nivel de detalle es ayudar a los lectores a comprender la aplicación completa en los sistemas del mundo real.


\hfill

Sin embargo, este libro intenta cubrir muchos temas relevantes que un ingeniero en el campo encontraría y proporciona una comprensión fundamental para un estudio posterior.

\hfill

El Capítulo 1 presenta los sistemas dinámicos, proporciona motivación sobre por qué es importante estudiarlos y analiza los diferentes tipos de sistemas.


\hfill

El Capítulo 2 analiza el modelado y cubre ecuaciones diferenciales y en diferencias, funciones de transferencia, modelos de espacio de estado, valores propios, vectores propios y descomposición de valores singulares.

\hfill

El Capítulo 3 se enfoca en soluciones de ecuaciones dinámicas, puntos de equilibrio y estabilidad.

\hfill

El Capítulo 4 analiza los sistemas no lineales y algunos comportamientos ricos que solo se encuentran en ellos, como los ciclos límite, las bifurcaciones, el caos y la linealización.

\hfill

Finalmente, el Capítulo 5 presenta los sistemas hamiltonianos, que normalmente caen en el ámbito de los físicos. 


\hfill

Sin embargo, los sistemas vibracionales no amortiguados y sus equivalentes son una clase importante de sistemas hamiltonianos, y existe una teoría muy rica en esta área.



\bibliographystyle{acm}
\bibliography{biblio}

\end{document}

