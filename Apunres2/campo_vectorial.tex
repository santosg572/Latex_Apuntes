\documentclass{beamer}
\usepackage[spanish]{babel}
\usepackage[utf8]{inputenc}
\usetheme{Warsaw}
\usecolortheme{crane}
\useoutertheme{shadow}
\useinnertheme{rectangles}

\title[Animales]{Animales de todo tipo}
\subtitle{Dando nombres a los animales}
\author[Adan, Eva, Serpiente]{
A. Adán$^{1}$ \and E. Eva$^{2}$ \and S. Serpiente$^{3}$}
\institute[EDEN \& HELL]{
  $^{1-2}$
  Universidad de Edén\\
  Al lado del manzano, Paraíso
  \and
  $^{3}$
  Universidad del Infierno\\
  Inframundo, 666, Tierra
  \and
  \texttt{\{$^{1}$eva, $^{2}$adan\}@paraiso.com, $^{3}$serpiente@infierno.com}
}
\date{\today}

\begin{document}

\frame{\titlepage}

\begin{frame}
Definición. Un campo vectorial sobre $D \subset \mathbb{R}^2$ es una función $\bar{F}$ que cada punto $(x,y) \in D$
le asigna un (único) vector de dos componentes $\bar{F} \in V_2$.

Entonces podemos escribir

\[
\bar{F} = \hat{i} P(x,y) + \hat{j} Q(x,y)
\]

Ejemplo.

\hfill

1) $\bar{F} = \hat{i} \sin y + \hat{j} e^x $

\hfill

\textbf{Dominio}. El dominio de un campo vectorial es el plano es un subconjunto de $\mathbb{R}^2$. El "dominio natural" del campo está dado por la intersección de los dominios naturales de las funciones componentes.


\end{frame}

\begin{frame}

\textbf{Ejemplo.} $\bar{F}(x,y) = \hat{i} \ln(xy) + \hat{j} \cos (x+y)$. Tiene como dominio natural todos los puntos del primer y tercer cuadrante del plano, excepto los ejes coordenados.

\hfill

\textbf{Representación Grafica}

\hfill

\textbf{Ejemplo 1.} Describa el campo $\bar{F} ? -\hat{i} y + \hat{j} x$ en el plano, trazando algunos de los vectores.

\hfill

x = 1, y = 0, etonces $\bar{F} = \hat{j}$

x = 0, y = 1, etonces $\bar{F} = -\hat{i}$

x = -1, y = 0, etonces $\bar{F} = -\hat{j}$

x = 0, y = -1, etonces $\bar{F} = \hat{i}$

\hfill

Observar.

\begin{enumerate}
\item $|\bar{F}| = \sqrt{(-y)^2 + x^2} = |\bar{r}|$
\item $\bar{r} \cdot \bar{F} = (x,y) \cdot (-y, x) =0$
\end{enumerate}


\end{frame}


\begin{frame}
\textbf{Campos de Gradientes}

\hfill

Si $f: D \subset \mathbb{R}^2 \rightarrow \mathbb{R}$. Se define el vector gradiente en cada punto $(x,y) \in D$ como 

\[
\bar{\bigtriangledown } f(x,y) = \hat{i} f_x (x,y) + \hat{j} f_y (x,y)
\]

\end{frame}

\begin{frame}

\end{frame}

\begin{frame}

\end{frame}

\begin{frame}

\end{frame}

\begin{frame}

\end{frame}

\begin{frame}

\end{frame}

\begin{frame}

\end{frame}

\begin{frame}

\end{frame}

\begin{frame}

\end{frame}

\begin{frame}

\end{frame}



\end{document}

