\documentclass{beamer}
\usepackage[spanish]{babel}
\usepackage[utf8]{inputenc}
\usetheme{Warsaw}
\usecolortheme{crane}
\useoutertheme{shadow}
\useinnertheme{rectangles}

\title[Animales]{Ecuaciones de primer orden}
%\subtitle{Dando nombres a los animales}
\author[Adan, Eva, Serpiente]{
L. Gonzalez-Santos$^{1}$}
\institute[EDEN \& HELL]{
  $^{1}$
  Universidad de Edén\\
  Al lado del manzano, Paraíso
  \and
  \texttt{\{$^{1}$eva, $^{2}$adan\}@paraiso.com, $^{3}$serpiente@infierno.com}
}
\date{\today}

\begin{document}

\frame{\titlepage}

\begin{frame}

El propósito de este tema es desarrollar algunos ejemplos elementales pero importantes de ecuaciones diferenciales de primer orden. Estos ejemplos ilustran algunas de las ideas básicas de la teoría de las ecuaciones diferenciales ordinarias de la forma más sencilla posible.

\hfill

Anticipamos que los primeros ejemplos de este tema serán familiares para los lectores que hayan tomado un curso de introducción a las ecuaciones diferenciales. Se incluyen ejemplos posteriores, como el modelo logístico con cosecha, para darle al lector una idea de ciertos temas (bifurcaciones, soluciones periódicas y mapas de Poincaré) a los que volveremos a menudo a lo largo del curso. 


\end{frame}

\begin{frame}

\textbf{El ejemplo más simple}

\hfill

La ecuación diferencial familiar para todos los estudiantes de cálculo

\[
\frac{dx}{dt} = ax
\]

es la ecuación diferencial más simple. También es uno de las más importantes. Primero, qué significa Aquí x = x(t) es una función desconocida de valor real de una variable real t y dx/dt es su derivada (también usaremos x o x(t) para la derivada). Además, $a$ es un parámetro; para cada valor de $a$ tenemos una ecuación diferencial diferente. La ecuación nos dice que para todo valor de t la relación
 
\end{frame}

\begin{frame}

Las soluciones de esta ecuación se obtienen del cálculo: si k es cualquier número real, entonces la función $x(t) = ke^{at}$ es una solución ya que

\hfill

Además, no hay otras soluciones. Para ver esto, sea u(t) cualquier solución y calcule la derivada de u(t)e-at:

\hfill

Por lo tanto u(t)e-at es una constante k, entonces u(t) = keat. Esto prueba nuestra afirmación. Por lo tanto, hemos encontrado todas las posibles soluciones de esta ecuación diferencial. A la colección de todas las soluciones de una ecuación diferencial la llamamos solución general de la ecuación.


\end{frame}

\begin{frame}

www

\end{frame}

\begin{frame}

www

\end{frame}

\begin{frame}

www

\end{frame}

\begin{frame}

www

\end{frame}

\begin{frame}

www

\end{frame}

\begin{frame}

www

\end{frame}

\begin{frame}

www

\end{frame}

\begin{frame}

www

\end{frame}

\begin{frame}

www

\end{frame}

\begin{frame}

www

\end{frame}



\end{document}