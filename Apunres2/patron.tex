\documentclass[a4paper,10pt]{article}

%%%%%%%%%%%% PREÁMBULO %%%%%%%%%%%%%%%%%%%%%

% Paquetes

\usepackage[utf8]{inputenc}
\usepackage[spanish,es-tabla]{babel}
\usepackage[T1]{fontenc}

\usepackage{listings}

% Comandos

\renewcommand{\lstlistingname}{Código}
\renewcommand{\lstlistlistingname}{Índice de fragmentos de código fuente}

% Opciones

\title{Python 2.*}
\author{Ondiz Zarraga}

%%%%%%%%%%%%%%%%%%%%%%%%%%%%%%%%%%%%%%%%%%%%%%

\begin{document}
\maketitle

\begin{abstract}
Este documento es una pequeña guía de Python 
\end{abstract}

\tableofcontents

\section{Sobre el lenguaje}

\begin{itemize}
    \item Interpretado
    \item Indentación obligatoria
    \item Distingue mayúsculas - minúsculas
    \item No hay declaración de variables (\textit{dynamic typing})
    \item Orientado a objetos  
    \item Garbage colector: quita los objetos a los que no haga referencia nada
\end{itemize}

\end{document}

