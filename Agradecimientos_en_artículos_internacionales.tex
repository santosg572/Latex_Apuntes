Agradecimientos en artículos internacionales

Olalde-Mathieu, V.E., Licea-Haquet, G.,L., Alcauter, S., Atilano-Barbosa, D., Dominquez-Frausto, C., A., Angulo-Perkins, A., Barrios, F.A., “Empathy-related 
differences in the anterior cingulate functional connectivity of regular cannabis users when compared to controls”, Neuroscience Research, 1-9, 2023,  
diciembre de 2023.
Atilano-Barbosa, D. and Barrios, F.A., “Brain morphological variability between Whites and African Americans: the importance of racial identity in brain 
imaging research”, frontiers, 1-15, 2023,  diciembre de 2023.
Rasgado-Toledo, J., Siddharth-Duvvadab, S., Shahc, A. , Ingalhalikarc, M. , Allurib, V., GarzaVillarreal, E.A., “Structural and functional pathology in 
cocaine use disorder with polysubstance use: A multimodal fusion approach structural-functional pathology in cocaine use disorder”, Progress in Neuro 
Psychopharmacology & Biological Psychiatry, 1-8, 2023,  noviembre de 2023.
Cruz-Carrillo, G., Trujillo-Villarreal, L.A., Angeles-Valdez, D., Concha, L.,Eduardo Garza-Villarreal, E.A., Camacho-Morales, A., “Prenatal Cafeteria Diet 
Primes Anxiety-like Behavior Associated to Defects in Volume and Di_usion in the Fimbria-fornix of Mice O_spring”, NEUROSCIENCE, 70-85, 2023,  noviembre de 
2023.
Román-López, T. V., García-Vilchis, B. , Murillo-Lechuga, V. , Chiu-Han, E., López-Camaño, X. , Aldana-Assad, O. , Diaz-Torres, S., 5 , Caballero-Sánchez, 
U., Ivett Ortega-Mora, Ramírez-González, D., Zenteno, D. , Espinosa-Valdés, Z., Tapia-Atilano, A. , Pradel-Jiménez, S., Rentería, M.E., MedinaRivera, A., 
Ruiz-Contreras, A.E., Alcauter, S., “Estimating the Genetic Contribution to Astigmatism and Myopia in the Mexican Population”, CAMBRIDGE, University Press, 
1-9, 2023,  noviembre de 2023.
Villaseñor, P. J., Cortés-Servín, D., Pérez-Moriel, A., Aquiles, A., Luna-Munguía, H., Ramírez Manzanares, A., Coronado-Leija, R., Larriva-Sahd, J., 1 and 
Concha, L., “Multi-tensor diffusion abnormalities of gray matter in an animal model of cortical dysplasia”, frontiers, 1-10, 2023,  noviembre de 2023.
Vázquez, P.G., Whitfield-Gabrieli, S., Bauer, C.C.C. & Barrios, F.A., “Brain functional connectivity of hypnosis without target suggestion. An intrinsic 
hypnosis rs-fMRI study”, The World Journal of Biological Psychiatry, 1-12, 2023,  noviembre de 2023.
Gracia-Tabuenca, Z., Díaz-Patiño, J.C., Arelio-Ríos,I., Moreno-García, M.B., Barrios, F.A. and Alcauter, S., “Development of the Functional Connectome 
Topology in Adolescence: Evidence from Topological Data Analysis”, eNeuro, 1-10, 2023,  noviembre de 2023.
Olalde-Mathieu, V.E., Sassi, F., Reyes-Aguilar, A., Mercadillo, R.E., Alcauter, S., Barrios., A.F., “Greater Empathic Abilities and Resting State Brain 
Connectivity Differences in Psychotherapists Compared to Non-psychotherapists”, Neuroscience, 1-10,  diciembre de 2022.
Aguilar-Moreno, A., Ortiz, J., Concha, L., Alcauter, S., Paredes, R.G., “Brain circuits activated by female sexual behavior evaluated by manganese enhanced 
magnetic resonance imaging”, PLOS ONE, 1-24,  agosto de 2022. 
Olalde-Mathieu, V.E., Licea-Haquet, G., Reyes-Aguilar, A. & Barrios, F.A., “Psychometric properties of the Emotion Regulation Questionnaire in a Mexican 
sample and their correlation with empathy and alexithymia”, Cogent Psychology, 1-11,  diciembre de 2022.
Rodríguez-Nieto, G., Mercadillo, R.E., Pasaye, E.H., and Fernando A. Barrios, F.A., “Affective and cognitive brain-networks are differently integrated in 
women and men while experiencing compassion”, Frontiers in Psychology, 1-12,  diciembre de 2022.
Domínguez-Arriola, M. E., Olalde-Mathieu, V. E., Garza-Villarreal, E.A., Barrios, F.A, “The Dorsolateral Prefrontal Cortex Presents Structural Variations 
Associated with Empathy and Emotion Regulation in Psychotherapists”, Brain Topography, 1-14, 21,  agosto de 2022.
Hevia-Orozco, J., Azalea Reyes-Aguilar, A., Pasaye, E.H., Barrios, F.A, “Participation of visual association areas in social processing emerges when rTPJ is 
inhibited”, eNeurologicalSci 27, 1-6,  diciembre de 2022.
Llamas-Alonso, L.A., Barrios, F.A., González-Garrido A., Ramos-Loyo J., “ Emotional faces interfere with saccadic inhibition and attention re-orientation: 
An fMRI study”, Neuropsychologia, 1-9,  diciembre de 2022.
Garcia-Saldivar, P., Garimella, A., Garza-Villarreal, E.A., Mendez, F.A., Concha, L., Merchant, H., “PREEMACS: Pipeline for preprocessing and extraction of 
the macaque brain surface”, NeuroImage, 227, p1-16.,  noviembre de 2021.
López-Gutiérrez, M.F., Gracia-Tabuenca, Z., Ortiz, J.J., Camacho, F.J., Young, L.J., Paredes, R.G., Díaz, N.F., Portillo, W., Alcauter, S., “Brain 
functional networks associated with social bonding in monogamous voles”, eLife, p1-25,  noviembre de 2021.
Gracia-Tabuenca, Z., Moreno, M.B., Barrios, F.A., Alcauter, S., “Development of the brain functional connectome follows puberty-dependent nonlinear 
trajectories”, NeuroImage, 229, p1-9,  noviembre de 2021.
Lizcano-Corte_s, F., Rasgado-Toledo, J., Giudicessi, A. and Giordano, M., “Theory of Mind and Its Elusive Structural Substrate”, frontiers in Human 
Neuroscience, p1-9, Vol. 15,  noviembre de 2021.
Rasgado-Toledo, J., Lizcano-Corte_s, F., Vi_ctor Olalde-Mathieu, V.E., Licea-Haquet, G., Zamora-Ursulo, M.A., Giordano, M. and Reyes-Aguilar, A., “A Dataset 
to Study Pragmatic Language and Its Underlying Cognitive Processes”, frontiers in Human Neuroscience, p1-7, Vol. 15,  noviembre de 2021.
Marti_nez, A.Y., Demertzi, A., Bauer, C.C.C., Gracia-Tabuenca, Z., Alcauter, S., and Barrios, F.A., “The Time Varying Networks of the Interoceptive 
Attention and Rest”, eNeuro, p1-13,  noviembre de 2021.
Domi_nguez-Arriola, M.E., Vi_ctor E. Olalde-Mathieu, V.E., Garza-Villareal, E.A., Barrios, F.A., “The Dorsolateral Prefrontal Cortex Presents Structural 
Variations Associated with Empathic Capacity in Psychotherapists“, bioRxiv, p1-19,  noviembre de 2021.
Lazcano, I., Cisneros_Mejorado, A., Concha, L., Jose_ Ortiz_Retana, J.J., Garza_Villarreal, E.A. & Orozco, A., “MRI_ and histologically derived neuro 
anatomical atlas of the Ambystoma mexicanum (axolotl)”, Scientific Reports, p1-13,  noviembre de 2021.
Luna-Munguia, H., Marquez-Bravo, L., Concha, L., “Longitudinal changes in gray and white matter microstructure during epileptogenesis in pilocarpine-induced 
epileptic rats”, Seizure: European Journal of Epilepsy, 90, p130-140.,  noviembre de 2021.
Hatton, S. N., Huynh, K. H., Bonilha, L., Abela, E., Alhusaini, S., Altmann, A., Alvim, M. K. M., Akshara R. Balachandra, A. R., Concha, L., “White matter 
abnormalities across different epilepsy syndromes in adults: an ENIGMA-Epilepsy study”, BRAIN, A Journal of Neurology, 1-20,  abril de 2020.
Rodríguez-Cruces, R., Bernhardt, B. C., Concha, L., “Multidimensional associations between cognition and connectome organization in temporal lobe epilepsy”, 
NeuroImage (213)1-11,  marzo de 2020.
Niño, S. A., Erika Chi-Ahumada, E., Ortíz, J., Zarazua, S., Concha, L., Jiménez-Capdeville, M. E., “Demyelination associated with chronic arsenic exposure 
in Wistar rats”, Toxicology and Applied Pharmacology, (393)1-9, 2020.,  marzo de 2020.
García-Gomar, M.G., Concha, L., Soto-Abraham, J., Tournier, J. D., Aguado-Carrillo, G., VelascoCampos, F., “Long-Term Improvement of Parkinson Disease Motor 
Symptoms Derived From Lesions of Prelemniscal Fiber Tract Components”, OPERATIVE NEUROSURGERY, Vol. 0, Num. 0, 1-12,  abril de 2020.
Gracia-Tabuenca, Z., Juan Carlos Díaz-Patiño, J.C., Arelio, I.,2 and Alcauter, S., “Topological Data Analysis Reveals Robust Alterations in the Whole-Brain 
and Frontal Lobe Functional Connectomes in Attention-Deficit/Hyperactivity Disorder”, March-April, 7(2) ENEURO, 1–10,  marzo de 2020.
Rodríguez-Cruces, R., Bernhardt, B.C., Luis Concha, L., “MULTIDIMENSIONAL ASSOCIATIONS BETWEEN COGNITION AND CONNECTOME ORGANIZATION IN TEMPORAL LOBE 
EPILEPSY”, COGNITIVE CONNECTOMICS IN TLE, bioRxiv, June, 1-25,  noviembre de 2019.
Rojas-Vite, G., Coronado-Leija, R., Narvaez-Delgado, O., Ramírez-Manzanares, A., Marroquín, J.L., Noguez-Imm, R., Aranda, M.L, Sherrer, B., Larriva-Sahd, 
J., Concha, L., “Histological validation of per-bundle water diffusion metrics within a region of fiber crossing following axonal degeneration”, NeuroImage, 
201,  noviembre de 2019.
CCC, Bauer, Whitfield-Gabrieli S, Díaz JL, Pasaye EH, Barrios FA, “From state-to-trait meditation: Reconfiguration of central executive and default mode 
networks”, eNeuro, 1-20,  noviembre de 2019.
Narvaez-Delgado, O., Rojas-Vite, G., Coronado-Leija, R., Ramírez-Manzanares, A., Marroquín, J.L., Noguez-Imm, R., Aranda, M.L, Sherrer, B., Larriva-Sahd, 
J., Concha, L., “Histological and diffusion-weighted magnetic resonance imaging data from normal and degenerated optic nerve and chiasm of the rat”, Data in 
brief, 26, 1-19,  noviembre de 2019.
Angulo-Perkins, A., Concha, L., “Discerning the functional networks behind processing of music and speech through human vocalizations”, PLOS-ONE,  octubre 
de 2019.
Manno, F.A.M, Raul R. Cruces,R.R., Lau, C. and Barrios, F.A., “Uncertain Emotion Discrimination Differences Between Musicians and Non-musicians Is 
Determined by Fine Structure Association: Hilbert Transform Psychophysics”, frontiers in Neuroscience,1-13,  noviembre de 2019.
Ortiz, J.J., Portillo, W., Paredes, R.G., Young, L.J, Alcauter, S., "Resting state brain networks in theprairie vole", SCiENTifiC REPOrTS, 8:1-11, 2018.,  
noviembre de 2018.
Mercadillo, R.E., Alcauterc, S., Barrios, F.A., "Effects of primatological training on anthropomorphic valuations of emotions", IBRO Reports 5, 54–59, 
2018.,  noviembre de 2018.
Rodríguez-Cruces, R, Velázquez-Pérez, L., Rodríguez-Leyva, I., Velasco, A.L.,Trejo-Martínez, D., Barragán-Campos, H.M., Camacho-Téllez, V., Concha, L., 
"Association of white matter diffusion characteristics and cognitivedeficits in temporal lobe epilepsy", Epilepsy & Behavior 79, 138–145, 2018.,  noviembre 
de 2018.
Gracia-Tabuenca, Z., Moreno, M.B., Barrios, F.A., Alcauter, S., "Hemispheric asymmetry and homotopy of resting state functional connectivity correlate with 
visuospatial abilities in school-age children", NeuroImage 174, 441–448, 2018. ,  noviembre de 2018.
Hernández-Abrego, A., Vázquez-Gómez, E., García-Colunga, J., "Effects of the antidepressant mirtazapine and zinc on nicotinic acetylcholine receptors", 
Neuroscience Letters 665, 1-6, 2018. ,  noviembre de 2018.
Reyes-Aguilar, A., Fernández-Ruiz, J., Pasaye, E.H., Barrios, F. A., “Executive Mechanisms forThinking about Negative Situations in both cooperative and 
non-cooperative contexts”, Frontiersin Human Neuroscience, 11(1-5),  mayo de 2017.
Sarael Alcauter, Liliana García-Mondragóna, Zeus Gracia-Tabuenca, Martha B. Moreno, Juan J.Ortiza, Fernando A. Barrios, “Resting state functional 
connectivity of the anterior striatum and prefrontal cortex predicts reading performance in school-age children”, Brain and Language, 174(94-102),  junio de 
2017.
Martínez-Soto J., Montero M.E., Córdova A., “Restauración psicológica y naturaleza urbana: algunas implicaciones para la salud mental”, Salud Mental, 37, 
217-224,  mayo de 2014.
Donnet S., Bartolo R., Fernández J.M., Silva Cunha J.P., Prado L., and Merchant H., “Monkeys time their pauses of movement and not their movement-kinematics 
during a synchronizationcontinuation rhythmic task¨, J Neurophysiol 111: 2138–2149.,  septiembre de 2014.
Angulo-Perkins A., William Aubé W., Peretz I., Barrios F.A. , Armony J.L. and Concha L., “Music listening engages specific cortical regions within the 
temporal lobes: Differences between musicians and non-musicians”, CORTEX, 59, 126-137.,  septiembre de 2014.
Bauer C.C.C., Díaz J.L, Concha L., Barrios F.A., “Sustained attention to spontaneous thumb sensations activates brain somatosensory and other proprioceptive 
areas”, Brain and Cognition 87, 86–96.,  septiembre de 2014.
Sánchez-Resendis O., Medina A.C., Serafín N., Prado-Alcalá R.A., Roozendaal B., Quirarte G.L., “Glucocorticoid-cholinergic interactions in the dorsal 
striatum in memory consolidation of inhibitory avoidance traning”, frontiers in Behavioral Neuroscience, Vol. 6, 1-6.,  noviembre de 2012.
Mercadillo R.E., Trujillo C., Sánchez-Cortazar J., Barrios F.A., “Brain activity in adhd patients performing the counting stroop task: A social neuroscience 
approach”, Phychological Reports: Disability and Trauma, 111,2,652-668.,  septiembre de 2012.
Mercadillo, R.E., Díaz, J. L., Pasaye, E. H., Barrios, F. A., "Perception of suffering and compassion experience: Brain gender disparities", Brain and 
Cognition 76, 5–14.,  noviembre de 2011.
Alcauter, S, Barrios, F. A., Díaz, R, Fernández-Ruiz, J., "Gray and white matter alterations in spinocerebellar ataxia type 7: An in vivo DTI andVBM study", 
NeuroImage 55 1–7,  2011.
ROMERO-ROMO J.I, BAUER C.C.C, PASAYE E.H., GUTIÉRREZ R.A., FAVILA R., BARRIOS F.A., “Abnormal Functioning of the Thalamocortical System Underlies the 
Conscious Awareness of the Phantom Limb Phenomenon”, The Neuroradiology Journal 23: 671-679.,  noviembre de 2010.
Mercadillo, RE, Díaz, JL, Pasaye, EH, Barrios, FA, "Gender differences in brain activation during induced compassion experiences", Neuropsychologia, 
manuscript number: NSY-D-10-00001,  enero de 2010.
PASAYE E.H, GUTIÉRREZ R.A., S. ALCAUTER S., MERCADILLO R.E., AGUILAR-CASTAÑEDA E, DE ITURBE M., ROMERO-ROMO J., BARRIOS F.A., “Event-Related Functional 
Magnetic Resonance Images during the Perception of Phantom Limb. A Brushing Task”, The Neuroradiology Journal 23: 665-670.,  noviembre de 2010.
Castillo C.G, Mendoza S., Saavedra J., Giordano M., “Lack of effect of intranigral transplants of a GABAergic cell line on absence seizures”, Epilepsy & 
Behavior.,  noviembre de 2010.
Morales, T., Ahuilar, L., Ramos, E., Mena, F., Morgan, C., "Fos Expression induced by milk ingestion in the caudal brainstem of neonatal rats", Brain 
Reseach, 1241, 76-83 ,  2008.
Medina, AC, Charles, JR, Espinoza-González, V, Sánchez-Resendis, O, Prado-Alcalá, RA, Roozendaal, B, and Quirarte, GL, "Glucocorticoid administration into 
the dorsal stratium facilitates memory consolidation of inhibitory avoidance training but not of the context or footshock components", Learning & Memory, 
673-677,  mayo de 2007.
Herrera-Alarcón, J, Villagómez-Amezcua, E, González-Padilla E, Jiménez-Severiano, H, "Stereological study of postnatal testicular development in Blackbelly 
sheep", Theriogenology, 582591,  2007.
Anguiano, B, García-Solís, P, Delgado, G, and, Aceves, CA, "Uptake and Gene Expression with Antitumoral Doses of Iodine in Thyroid and Mammary Gland: 
Evidence That Chronic Administration Has No Harmful Effects", THYROID, 17(9): 851-859,  2007.
Frias, C., Torrero, C., Regalado, M., and Salas., M., Organization of olfactory glomeruli in neonatally undernourished rats, Nutritional Neuroscience, 1-7,  
2006.
Zugasti-Cruz, A., Maillo, M., López-Vera, E., Falcón, A., Heimer de la Cotera, E.P., Olivera, B., M., Aguilar, M.B., Amino acid sequence and biological 
activity of a gama-conotoxin-like peptide from the worm-hunting snail, Peptides 27, 506-611,  2006.
Castillo, C.G., Mendoza, S., Freed., W.J., Giordano, M., "Intranigral transplants of immortalized GABAergic cells decrease the expression of kainic 
acid-induced seizures in the rat", Behavioural Brain Research, 171, 109-115,  2006.
Cepeda-Nieto, A. C., Pfaff, S. L., Varela-Echavarría, A., “Homeodomain transcription factors in the development of subsets of hindbrain reticulospinal 
neurons”, Mol. Cell Neurosci., 28:30-41,  mayo de 2005.
Luna, M., Barraza, N., Berumen, L., Carranza, M., Pedernera, E., Harvey, S., Arámburo, C., “Heterogeneity of growth hormone immunoreactivity in lymphoid 
tissues and changes during ontogeny in domestic fowl”, Gen. Comp. Endocrinol., 144:28-37,  mayo de 2005.
García-Solís, P, Alfaro, Y, Anguiano, B, Delgado G, Guzman, R. C. , Nandi, S., Díaz-Muñoz, M., Vazquez-Martinez, O., Aceves, C., “Inhibition of 
N-methyl-N-nitrosourea-induced mammary carcinogenesis by molecular iodine (I2) but not by iodide (I-) treatment Evidence that I2 prevents cancer promotion”, 
Mol. Cell Endocrinol., 236:49-57,  mayo de 2005.
Arroyo-Helguera, O., Mejia-Viggiano, C., Varela-Echavarria, A., Cajero-Juárez, M., Aceves, C., “Regulatory role of the 3I untranslated region (3’UTR) of rat 
5I deiodinase (D1). Effects on Messenger RNA translation and stability”, Endocrine, 27:219-227,  mayo de 2005.
Aguilar, M. B. , Lopez-Vera, E., Imperial, J. S., Falcon, A., Olivera, B. M., De la Cotera, E. P., “Putative gamma-conotoxins in vermivorous cone snails: 
the case of Conus delessertii”, Peptides. 26:23-7,  mayo de 2005.
Aguilar, M. B. , Lopez-Vera, E., Ortiz, E., Becerril, B., Possani, L., D., Olivera, B. M., De la Cotera, E. P., “A novel conotoxin from conos delessertii 
with posttranslationally modified lisien residues”, Biochemistry, 44:11130-11136,  mayo de 2005.
Anguiano, B., Rojas-Huidobro, R., Delgado, G., Aceves, C., “Has the mammary gland a protective mechanism against overexposure to triiodothyronine during the 
peripartum period? The prolactin pulse down-regulates mammary type I deiodinase responsiveness to norepinephrine”, J. Endocrinol., 183:267-77,  mayo de 
2004.
Berumen, L. C., Luna, M., Carranza, M., Martinez-Coria, H., Reyes, M., Carabez, A., Arámburo, C., “Chicken growth hormone: further characterization and 
ontogenic changes of an N-glycosylated isoform in the anterior pituitary gland”, Gen. Comp. Endocrinol., 139:113-23,  mayo de 2004.
López-Vera, E., de la Cotera, E. P., Maillo, M., Riesgo-Escovar, J., Olivera, B. M., Aguilar, M. B., “A novel structural class of toxins: the 
methionine-rich peptides from the venoms of turrid marine snails (Mollusca, Conoidea)”, Toxicon., 43:365-74,  mayo de 2004.
Mena-Segovia, J., Favila, R., Giordano, M., “Long-term effects of striatal lesions on c-Fos immunoreactivity in the pedunculopontine nucleus”, Eur. J. 
Neurosci., 20:2367-76,  mayo de 2004.
Peréz-León, J.A., Sarabia, G., Miledi, R, Garcia-Alcocer, G., “Distribution of 5- hydroxytriptamine2C receptor RNA in rat retina”, Brain Res Mol Brain Res. 
125:140-2,  mayo de 2004.
Granados-Rojas, L., Aguilar, A., Díaz-Cintra, S., “The mossy fiber system of the hippocampal formation is decreased by chronic and postnatal but not by 
prenatal protein malnutrition in rats”, Nutr. Neurosci., 7:301-8,  mayo de 2004.
Díaz-Cintra, S, Yong A, Aguilar A, Bi, X., Lynch, G., Ribak, C.E., “Ultrastructural analysis of hippocampal piramidal neurons from apolipoprotein 
E-deficient mice treated with a cathepsin inhibitor”, J. Neurocytol. 33:37-48,  mayo de 2004.
Luna, M., Huerta, L, Berumen, L, Martínez-Coria, H., Harvey, S., Arámburo, C., “Growth hormone in the male reproductive tract of the chicken: heterogeneity 
and changes during ontogeny and maturation”, Gen. Comp. Endocrinol. 137:37-49,  mayo de 2004.
Díaz del Guante, M.A., Rivas, M., Prado-Alcalá, R. A., Quirarte, G. L., Amnesia produced by pretraining infusion of serotonin into the substantia nigra, 
SYNAPTIC TRANSMISSION NEUROREPORT, Vol. 15, No. 15, 2527-2529,  2004.
Hernádez-Montiel, H. L., Meléndez-Herrera, E., Cepeda-Nieto, A. C., Mejía-Viggiano, C., LarrivaSahd, J., Guthrie, S., Varela-Echavarría, A.”Diffusible 
signals and fasciculated growth in reticulospinal axon pathfinding in the hindbrain”, Dev. Biol., 255:99-112,  mayo de 2003.
Mena-Segovia, J., Giordano, M., “Striatal dopaminergic stimulation produces c-Fos expression in the PPT and an increase in wakefulness”, Brain Res., 
986:30-8,  mayo de 2003.
Vázquez-Martínez, O., Cañedo-Merino, R., Díaz-Muñoz, M., Riesgo-Escovar, “Biochemical characterization, distribution and phylogenetic analysis of Drosophila 
melanogaster ryanodine and IP3 receptors, and thapsigargin-sensitive Ca2+ ATPase”, J. Cell Sci., 116:2483-94,  mayo de 2003.
Prado-Alcalá, R. A., Solana-Figueroa, R., Galindo, L. E., Medina, AC, Quitarte, G. L., “Blockade of striatal 5-HT2 receptors produces retrograde amnesia in 
rats”, Life Sci., 74:481-8,  mayo de 2003.
Prado-Alcalá, R. A., Ruiloba, M. I., Rubio, L., Solana-Figueroa, R., Medina, C., Salado-Castillo, R., Quitarte, G. L., “Regional infusions of serotonin into 
the striatum and memory consolidation”, Synapse, 47:169-75,  mayo de 2003.
Martinez, I., Quitarte, G. L., Díaz-Cintra, S., Quiroz, C., Prado-Alcala, R. A., “Effects of lesions of hippocampal fields CA1 and CA3 on acquisition of 
inhibitory avoidance”, Neuropsychobiology, 46:97103,  mayo de 2002.
Martínez-Coria, H., López-Rosales, L. J., Carranza, M., Berumen, L., Luna, M., Arámburo, C., “Differential secretion of chicken growth hormone variants 
after growth hormone-releasing hormone stimulation in Vitro”, Endocrine, 17:91-102,  mayo de 2002.
Mena-Segovia, J., Cintra, L., Prospéro-Garcia, O., Giordano, M. “Changes in sleep-waking cycle after striatal excitotoxic lesions”, Behav. Brain Res., 
136:475-81,  mayo de 2002.
Solana-Figueroa, R., Salado-Castillo, R., Galindo, L. E., Quitarte, G. L., Prado-Alcalá, R. A., “Effects of pretraining intrastriatal administration of 
p-chloroamphetamine on inhibitory avoidance”, Neurobiol Learn Mem., 78:178-85,  mayo de 2002.
Díaz, N., Huerta, I., Marina, N., Navarro, N., Mena, F., “Regional mechanisms within anterior pituitary of lactating rats may regulate prolactin secretion”, 
Endocrine, 18:41-6,  mayo de 2002.
Marina, N., Morales, T., Díaz, N., Mena, F., “Suckling-induced activation of neural c-fos expression at lower thoracic rat spinal cord segments”, Brain 
Res., 954:100-14,  mayo de 2002.
Larriva-Sahd, J., Condés Lara, M., Martínez-Cabrera, G., Varela-Echavarría, A., “Histological and ultrastructural characterization of interfascicular 
neurons in the rat anterior commissure”, Brain Res., 931:81-91,  mayo de 2002. 
Giordano, M., Mejía-Viggiano, M. C., “Gender differences in spontaneous and MK-801-induced activity alter striatal lesions”, Brain Res. Bull., 56:553-61,  
mayo de 2001.
Fernández-Bouzas, A., Harmony, T., Fernández, T., Silva-Pereyra, J., Valdés, P., Bosch, J., Aubert, E., Casian, G., Otero Ojeda, G., Ricardo, J., 
Hernandez-Ballesteros, A., Santiago, E., “Sources of abnormal EEG activity in brain infarctions”, Clin. Electroencephalogr., 31:165-9,  mayo de 2000.
Aguilar, M.B., Lezama-Monfil, L., Maillo, M., Pedraza-Lara, H., López-Vera E., Heimer de la Cotera, E.P., "A biologically active hydrophobic T-1-conotoxin 
from the venom of Conus spurius", Peptides, 27, 500-505,  2000.
Arámburo, C., Luna, M., Carranza, M., Reyes, M., Martínez-Coria, H., Scanes, C. G., “Growth hormone size variants: changes in the pituitary during 
development of the chicken”, Proc. Soc. Exp. Biol. Med., 223:67-74,  mayo de 2000.
Gutiérrez-Ospina, G., Jiménez-Trejo, F. J., Favila, R., Moreno-Mendoza, N. A., Granados Rojas, Barrios, F. A., Diaz-Cintra, S., Merchant-Larios, H., 
“Acetylcholinesterase-positive innervation is present at undifferentiated stages of the sea turtle Lepidochelis olivacea embryo gonads: implications for 
temperature-dependent sex determination”, J. Comp. Neurol., 410:90-8,  mayo de 1999.
Dueñas, Z., Torner, L., Corbacho, A. M., Ochoa, A., Gutiérrez-Ospina, G., López-Barrera, F., Barrios, F. A., Berger, P., Martínez de la Escalera, G., Clapp, 
C., “Inhibition of rat corneal angiogenesis by 16-kDa prolactin and by Endogenous prolactin-like molecules”, Invest. Ophthalmol. Vis. Sci., 40:2498-505,  
mayo de 1999.
García-Colunga, J., Valdiosera, R., García, U., “P-type Ca2+ current in crayfish peptidergic neurones”, J. Exp, Biol., 202:429-440,  mayo de 1999.
Giordano, M., Salado-Castillo, R., Sánchez-Alvarez, M., Prado-Alcalá, R. A., “Striatal transplants prevent AF64A-induced retention deficits”, Life Sci., 
63:1953-61,  mayo de 1998.
Gutiérrez-Ospina, G., Díaz-Cintra, S., Aguirre-Portilla, A., Aguilar-Vázquez, A., López, S. R., Barrios, F.A., “Comparable activity levels in 
developmentally deprived and non-deprived layer IV cortical columns of the adult rat primary somatosensory cortex”, Neurosci. Lett., 247:5-8,  mayo de 1998.


