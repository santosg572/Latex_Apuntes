\documentclass[11pt, letterpaper]{article}
 
\begin{document}

\textbf{{\Large  Cursos recibidos}}

\begin{enumerate}

Agradecimientos en tesis de doctorado
1. Atilano-Barbosa, D., “Morfometría y conectividad funcional cerebral en una muestra de sujetos con
diversas identidades étnicas y raciales: un estudio a partir del proyecto del Conectoma Humanos”,
Doctorado de Ciencias Biomédicas, Créditos en: Tesis Doctorado, mayo de 2025.
2. López-Guerrero, N., G., “Trayectorias de desarrollo y diferencias en las propiedades funcionales de
las redes cerebrales de neonatos prematuros y a término”, Doctorado de Ciencias Biomédicas,
Créditos en: Tesis Doctorado, enero de 2025.
Agradecimientos en tesis de licenciatura

1. Cureño-Mendoza, D.E., “Desarrollo de Modelos Tridimensionales del Sistema Nervioso Central:
Implementación en la enseñanza de neuroanatomía en la licenciatura en neurociencias”, tesis en
Licenciatura en Neurociencias, Créditos en: Tesis Licenciatura, agosto de 2025.
2. Ocampo-Luna, P., “Estudio de los núcleos de la vía auditiva en adultos jóvenes mediante
resonancia magnética y potenciales evocados auditivos”, tesis en Licenciatura en Neurociencias,
Créditos en: Tesis Licenciatura, agosto de 2025.
Agradecimientos en artículos internacionales
1. Angeles-Valdez, D., Lopez-Castro, A., Rasgado-Toledo, J., Naranjo-Albarran, L., Garza-Villarreal,
E.A., “Improved classification of alcohol intake groups in the Intermittent-Access Two-Bottle choice rat
model using a latent class linear mixed model”, Progress in Neuropsychopharmacology & Biological
Psychiatry 139, 1-7, 2025, Créditos en: Artículo internacional, noviembre de 2025.
Agradecimientos en congresos internacionales
1. Espinosa-Méndez, M., Ramírez-González, D., Díaz-Patiño, J.C., Román-López, T.V., PasayeAlcaraz, E.H., 
Domínguez-Frausto, I., Zaldívar-Morales, E., Piña-Hernández, A., Medina-Rivera, A.,
Ruiz-Contreras, A.E., , Rentería, M., Alcauter, S., “The topology of the brain functional connectome
and its association with cognitive performance”, Neuroscienc SfN, November 15–19, San Diego, E.U,
Créditos en: Congreso internacional, noviembre de 2025.
2. Piña Hernández, A., Espinosa Méndez, I., Atilano Barbosa, D., Barrios, F.A., “Socioemotional and
cognitive domains in family members of feminicide victims in Mexico”, Neuroscienc SfN, November
15–19, San Diego, E.U, Créditos en: Congreso internacional, noviembre de 2025.
Otros agradecimientos
1. “Q2 Summer School”, Staff and General Support, Institute of Neurobiology, August 4-15, Créditos
en: Otro, agosto de 2025.
2. “Semana del Cerebro: Moldea tu Cerebro”, Centro Académico Cultural, Campus Juriquilla, 20 hrs.,
marzo 11-15, Créditos en: Otro, marzo de 2025.
3. Participación como Juez, Créditos en: XXXII Jornadas Académicas del Instituto de Neurobiología,
septiembre de 2025.
4. López-Guerrero, N., Sarael Alcauter, S., “Developmental Trajectories and Differences in Functional
Brain Network Properties of Preterm and At-Term Neonates”, Human Brain Mapping, 1-17, 2025,
Créditos en: Agradecimiento en articulo internacional, noviembre de 2025.
5. Rasgado-Toledo, J., Angeles-Valdez, D., Carranza-Aguilar, C. J., Lopez-Castro, A., and Eduardo A.
Garza-Villarreal, E, “The effect of chronic stress and chronic alcohol intake on behaviour, brain
volume, and functional connectivity in a longitudinal rat model”, BRAIN COMMUNICATIONS, 1-16,
2025, Créditos en: Agradecimiento en articulo internacional, noviembre de 2025.
6. Coutiño, D., Guerrero, J., Ayala, M., Badillo, M, and Concha, L., “Detección de la generación de la
sustancia blanca de la vía visual mediante imágenes de resonancia magnética sensibles a difusión”,
Jornadas Académicas del Instituto de Neurobiología, UNAM, septiembre 24-26, Créditos en: XXXII
Jornadas Académicas del Instituto de Neurobiología., septiembre de 2025.
7. Guerrero Morales, J.R., García-Miranda, L., Sánchez-Yépez, J., Mendoza-Trejo, M.S., GarcíaGomar, M.G., Giordano M., 
Rodríguez Córdova, V.M., “Cambios en el sistema nervioso central
medidos por tensor de difusión en un modelo de exposición repetida al herbicida atrazina en
roedores”, Reunión conjunta de Procesamiento de Neuroimágenes y Visión Computacional, XXVII
NeuroVisión, Centro de Investigación en Matemáticas A.C. (Cimat), octubre 29-31, Créditos en:
XXVII NeuroVisión, octubre de 2025.
8. Ocampo Luna, P., Carbajal-Valenzuela, C.C., Fair, M., García-Gomar, M.G., “Estudio de la

microestructura de los núcleos de la vía auditiva en adultos jóvenes mediante Imágenes de
Resonancia Magnética”, Reunión conjunta de Procesamiento de Neuroimágenes y Visión
Computacional, XXVII NeuroVisión, Centro de Investigación en Matemáticas A.C. (Cimat), octubre
29-31, Créditos en: XXVII NeuroVisión, septiembre de 2025.
Estímulos académicos
1. PRIDE C Dirección General de Asuntos del Personal Académico, INB-UNAM, Fecha de inicio:
marzo de 2003.
Proyectos
Proyectos
1. “Propiedades Turnpike en Problemas de Control Óptimo Estocástico”, Responsable: Dr. Cutberto
Romero Melèndez, Tipo: Investigación, Status: Inicio, Monto $0.00, Fecha de inicio: abril de 2023,
Fecha de conclusión: abril de 2026.
Publicaciones
Artículos publicados
1. Romero-Meléndez, C., Castillo-Fernández, D., González-Santos, L, “Asymptotic Stability in a
Controlled Stochastic Lotka-Volterra Model with Lévy Noise”, International Journal of Applied Physics
and Mathematics, noviembre de 2025; 15(1), 1-12.
Docencia
Cursos regulares
1. “Introducción a la computación y programación con Phyton”, Modalidad: Presencial y a Distancia,
Semestre: 1, Maestría en Ciencias, Neurobiología, Maestría, Instituto de Neurobiología, Instituto de
Neurobiologia, Tipo de programa académico: Escolarizado, Año del programa: 2025, Periodos: 2026-
1.
2. “Introducción al lenguaje de programación R”, Modalidad: Presencial y a Distancia, Semestre: 1,
Maestría en Ciencias, Neurobiología, Maestría, Instituto de Neurobiología, Instituto de Neurobiologia,
Tipo de programa académico: Escolarizado, Año del programa: 2025, Periodos: 2025-2.
Cursos especiales
1. “Introducción a la Estadística con R”, Duración: Semanal, Modalidad: A Distancia, Horas por
semana: 6, Total de horas: 30, Instituto de Neurobiología, País: México, Fecha de inicio: junio de
2025, Fecha de conclusión: junio de 2025.

\end{document}

