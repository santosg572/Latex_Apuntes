\begin{frame}
\begin{enumerate}
\item Dados los puntos A(-3,2) y B(5,2). ¿cuál es la longitud de AB?
\item Calcular las coordenadas del punto medio de AB en cada caso:

\begin{enumerate}
\item A(-3, 2), B(5,2)
\item A(6,4), B(8.2)
\item A(-2, -1), B(3,4)
\end{enumerate}

\end{enumerate}
\end{frame}

\begin{frame}

\begin{enumerate}
\item A está a dos tercios de la distancia (1, 10) a (-8, 4) y B está en el punto medio del segmento que une (0, -7) con (6, -11). Valcular la distancia AB.
\item Demostrar que (-1, 4), (-3,-6) y (3, -2) son vértives de un triángulo isósceles.
\item Dos vértices de un triángulo son (-5, 3) y (-1, -1). La ordenada del tercer vértice es 5. Si el área del triángulo es de 16b unidades de superficie, ¿cuáles son las posibles abscisas del tercer vértice?
\item Calcular el área del pentágono cuyos vértices son consecutivamente, (7,6), (0,5), (-1,0), (5, -2) y (3,2).
\end{enumerate}
\end{frame}


\begin{frame}


\end{frame}

\begin{frame}


\end{frame}

\begin{frame}


\end{frame}

\begin{frame}


\end{frame}

\begin{frame}


\end{frame}

\begin{frame}


\end{frame}

\begin{frame}


\end{frame}

\begin{frame}


\end{frame}

\begin{frame}


\end{frame}

\begin{frame}


\end{frame}

\begin{frame}


\end{frame}

\begin{frame}


\end{frame}

\begin{frame}


\end{frame}

\begin{frame}


\end{frame}

\begin{frame}


\end{frame}

\begin{frame}


\end{frame}

\begin{frame}


\end{frame}


