\textbf{\color{red}Agradecimientos en tesis de doctorado}

\begin{enumerate}
\item Aguilar-Moreno, J. A., “Conducta sexual y actividad cerebral evaluada por resonancia magnética por incrementos de manganeso (MEMRI) en ratas hembra de 
la 
cepa Wistar”, Instituto de Neurobiología, 2023,  noviembre de 2023.

\item RODRÍGUEZ-VIDAL, LL., “ESTUDIO DE LA CONECTIVIDAD FUNCIONAL EN ESTADO DE REPOSO DEL CLAUSTRUM”, DOCTORADO EN CIENCIAS BIOMÉDICAS, UNAM, 2023,  
diciembre de 
2023.

\item Olalde Mathieu, V.E., "CONECTIVIDAD FUNCIONAL CEREBRAL RELACIONADA A COMPONENTES DE LA RESPUESTA EMPÁTICA EN PSICOTERAPEUTAS,  diciembre de 2022.

\item Gracia-Tabuenca, Z., “Desarrollo de la conectividad funcional cerebral en la adolescencia y su correlato con el desempeño cognitivo”, Instituto de 
Neurobiología, UNAM,  noviembre de 2020.

\item Ortiz Retana, J. J., “Redes cerebrales en estado de reposo de topillo de la pradera (microtus ochrogaster)”, Doctorado en Ciencias Biomédicas 
(Neurobiología), INB-UNAM,,  mayo de 2019.

\item Rodríguez Cruces, R., “Anormalidades de la conectividad cerebral y deterioro cognitivo en la epilepsia del lóbulo temporal”, Doctorado en Ciencias 
Biomédicas (Neurobiología), INB-UNAM,  mayo de 2019.

\item García Gomar, María G., “Tractografía del Subtálamo en Pacientes con Enfermedad deParkinson”,  julio de 2017.

\item Reyes-Aguilar, Azalea, "Correlatos Neuronales de la Inferencia de Estados Mentales de Otros Contextos de Cooperación VS No Cooperación",  septiembre 
de 
2016.

\item Alejandra A. Angulo Perkins, "Identificación mediante resonancia magnética funcional de regiones corticales que participan en la percepción de la 
música y 
el habla", Instituto de Neurobiología, UNAM,  diciembre de 2015.

\item Sofía González-Salinas, "Eventos Moleculares Asociados a la Evocación de la Memoria y a la Extinción", Instituto de Neurobiología, UNAM,  diciembre de 
2015.

\item Martha B. Moreno-García, "Correlación entre área y la microestructura del Cuerpo Calloso y Procesos Cognitivos en niños de 7 a 9 años", Instituto de 
Neurobiología, UNAM,  septiembre de 2015.

\item Mercadillo-Caballero R.E., “Correlatos Cerebrales de la Experiencia Emocional de Compasión y la Impulsividad”, Tesis Doctoral, Instituto de 
Neurobiología.,  
noviembre de 2012.

\item Pasaye Alcaraz, E. H., "Resonancia Magnética Funcional a 3.0 T en sujetos amputados durante la sensación del Miembro Fantasma", Instituto de 
Neurobiología,  
noviembre de 2011.

\item Sandoval Minero, M.T, "Regulación de la proyección axonal de neuronas decusantes en el rombencéfalo caudal",  2008.

\item Frías Castañeda, María del Carmen,"Efecto de la desnutrición postnatal en la morfología celular del bulbo olfatorio",  2007.

\item Martín González, Cecilia, "Regulación Gabaérgica de la actividad de las adenilato ciclasas en las células GT1",  2007.

\item Martínez-García, M. I., “Participación del hipocampo en la memoria de largo plazo”, Doctorado en ciencias biomédicas, UNAM-INB,  2003.

\end{enumerate}

