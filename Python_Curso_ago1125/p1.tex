\documentclass{beamer}
%Information to be included in the title page:
\title{Introducción a la Programación con "PYTHON"}
\author{L. González-Santos}
\institute{Instituto de Neurobiología, UNAM}
\date{\today}

\begin{document}

\frame{\titlepage}

\begin{frame}
\frametitle{Sistemas de Numeración}

* Números Naturales = $\mathbb{N} = \{1, 2, 3, 4, 5, .... \}$

* Números Enteros = $\mathbb{Z} = \{..., -4, -3, -2, -1, 0, 1, 2, 3, 4, 5, .... \}$

* Números Racionales = $\mathbb{Q} = \{\frac{p}{q} | p, q \in \mathbb{Z} \text{ y } q \neq 0 
\}$

* Números Irracionales = $\mathbb{I} = \{ -\pi, \pi, \sqrt{2}, e, .... \}$

* Números Reales = $\mathbb{R} = \mathbb{Q} \bigcup \mathbb{I} $

\end{frame}

\begin{frame}
\frametitle{Operadores Aritméticos y de Comparación}

\begin{itemize}
\item +, -, *, /, **
\item $ <, \leq, >, \geq, ==, != $
\item and, or, not
\item Valores Lógicos: TRUE, FALSE 
\end{itemize}

\begin{center}
\begin{tabular}{ |c|c|c| } 
 \hline
 \textcolor{red}{and} & TRUE & FALSE \\ 
 TRUE & TRUE & FALSE \\ 
 FALSE & FALSE & FALSE \\ 
 \hline
\end{tabular}
\end{center}

\hfill

\begin{center}
\begin{tabular}{ |c|c|c| }
 \hline
 \textcolor{red}{or} & TRUE & FALSE \\
 TRUE & TRUE & TRUE \\
 FALSE & TRUE & FALSE \\
 \hline
\end{tabular}

\end{center}


\end{frame}

\begin{frame}
\frametitle{Vectores}

Un vector en matemáticas se define como $\textbf{x} = ( x_1, x_2, x_3, ..., x_n )$, 
donde $x_i \in  \mathbb{R} $. Un número en los números reales es llamado \textbf{escalar}.

\hfill

Dados dos vectores \textbf{x} y \textbf{y} se define la suma de estos vectores como:

\[
\begin{matrix}
\textbf{x+y} = ( x_1, x_2, x_3, ..., x_n ) + ( y_1, y_2, y_3, ..., y_n ) = \\
( x_1+y_1, x_2+y_2, x_3+y_3, ..., x_n+y_n )
\end{matrix}
\]

La multiplicación de un escalar $\alpha$ y un vector se define como:

\[
\begin{matrix}
\mathbf{\alpha x} = \alpha ( x_1, x_2, x_3, ..., x_n ) = \\
( \alpha x_1+ \alpha x_2, \alpha x_3, ..., \alpha x_n )
\end{matrix}
\]

\end{frame}

\begin{frame}
\frametitle{Matrices}

Una matriz en matemáticas se define como

\[
A = \begin{pmatrix}
a_{11} & a_{12}  & ...  & a_{1n} \\
a_{21} & a_{22}  & ...  & a_{2n} \\
a_{31} & a_{32}  & ...  & a_{3n} \\
... & ... & ... & ... \\
a_{m1} & a_{m2}  & ...  & a_{mn}  \\
\end{pmatrix}
\]

Donde $a_{ij}$, con i=1,..., m y j=1..., n son números reales. Decimos que la matriz es de 
ramaño $m \times n$


\end{frame}

\begin{frame}
\frametitle{Matrices}

Suma de matrices:

\[
A + B = \begin{pmatrix}
a_{11} & a_{12}  & ...  & a_{1n} \\
a_{21} & a_{22}  & ...  & a_{2n} \\
a_{31} & a_{32}  & ...  & a_{3n} \\
... & ... & ... & ... \\
a_{m1} & a_{m2}  & ...  & a_{mn}  \\
\end{pmatrix} + \begin{pmatrix}
b_{11} & b_{12}  & ...  & b_{1n} \\
b_{21} & b_{22}  & ...  & b_{2n} \\
b_{31} & b_{32}  & ...  & b_{3n} \\
... & ... & ... & ... \\
b_{m1} & b_{m2}  & ...  & b_{mn}  \\
\end{pmatrix} =
\]

\[
A = \begin{pmatrix}  
a_{11}+b_{11} & a_{12}+b_{12}  & ...  & a_{1n}+b_{1n} \\
a_{21}+b_{21} & a_{22}+b_{22}  & ...  & a_{2n}+b_{2n} \\
a_{31}+b_{31} & a_{32}+b_{21}  & ...  & a_{3n} \\
... & ... & ... & ... \\
a_{m1} & a_{m2}  & ...  & a_{mn}  \\
\end{pmatrix}
\]



\end{frame}\begin{frame}
\frametitle{Sistemas de Numeración}


\end{frame}\begin{frame}
\frametitle{Sistemas de Numeración}


\end{frame}\begin{frame}
\frametitle{Sistemas de Numeración}


\end{frame}\begin{frame}
\frametitle{Sistemas de Numeración}


\end{frame}\begin{frame}
\frametitle{Sistemas de Numeración}


\end{frame}\begin{frame}
\frametitle{Sistemas de Numeración}


\end{frame}\begin{frame}
\frametitle{Sistemas de Numeración}


\end{frame}

\end{document}


