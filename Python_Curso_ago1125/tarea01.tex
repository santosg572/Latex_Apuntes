\documentclass[a4paper]{article}

\usepackage{xcolor}

\title{Tarea 01 \\ \textcolor{red}{\small entregar antes del próximo lunes }}

\author{\small santosg572@gmail.com }

\begin{document}
\maketitle

\hrule

\begin{enumerate}
\item Escriba los pasos que hay que seguir para resolver una ecuación de segundo grado: $ax^2+bx+c=0$, dode $a,b, c 
\in R$
\item Dada la función $f(x) = 2 - (x-2)^2$ definida en el intervalo [1,3]. 
Escriba los pasos necesarios para encontrar el área bajo la curva que define la función y el eje x.
\item Convertir $30^o$ a radianes.
\item Una aproximación para calcular el valor de $e^x$ es utilizar la siguiente expresión:

\[
1 + x + \frac{x^2}{2!} + \frac{x^3}{3!} + \frac{x^4}{4!} + \frac{x^5}{5!}
\]   

Calcule $e^{.3}$ y $e^{1}$

\item Simular el lanzamiento de una moneda, aguila o sol, \{ A, S \}.

\end{enumerate}

 \textcolor{red}{\small NOTA: } Para resolver los dos ejeercicios últimos utilice los modulos, 
 \textcolor{blue}{math} y  \textcolor{blue}{random}
  
\end{document}

