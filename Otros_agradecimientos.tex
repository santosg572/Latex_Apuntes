\textbf{Otros agradecimientos}

\begin{enumerate}

\item Vélez-Uriza, F., Cisneros-Mejorado, A., Garay, E. and Arellano, R. O., “LA ADMINISTRACIÓN SISTÉMICA DE LA CARBOLINA N-BUTIL-CARBOLINA3- CARBOXILATO 
(-CCB) 
PROMUEVE LA REMIELINIZACIÓN EN EL MODELO DE DESMIELINIZACIÓN POR CUPRIZONA”, Jornadas Académicas del Instituto de Neurobiología, UNAM, 2023, Créditos en: 30 
Jornadas Académicas del Instituto de Neurobiología UNAM, noviembre de 2023.

\item Moreno, J. A., González-Pérez, E. G., Rocha-García, M., Carranza-Aguilar, C. J. and AlcauterSolórzano, S., “CAMBIOS MORFOLÓGICOS EN LAS CORTEZAS CEREBRALES 
MOTORA PRIMARIA Y MOTORA SECUNDARIA Y SU RELACIÓN CON EL DESEMPEÑO MOTRIZ DE RATAS WISTAR EXPUESTAS A UN RÉGIMEN DE EJERCICIO AERÓBICO”, Jornadas Académicas del 
Instituto de Neurobiología, UNAM, 2023, Créditos en: 30 Jornadas Académicas del Instituto de Neurobiología UNAM, noviembre de 2023.

\item Espinosa-Méndez, I.M., Román-López, T. V., Ramírez-González, D., Sánchez-Moncada, C. I., Díaz-Téllez, X., Domínguez-Frausto, C. A., Murillo-Lechuga, V., 
López-Camaño, X. J., GuzmánTenorio, G. E., Robles-Rodríguez, G. D., Ortiz-Tapia, E. B., Piña-Hernández, A., Aldana-Assad, O.., Medina-Rivera, A., Ruiz-Contreras, 
A. E., Rentería, M.. and Alcauter, S., “HERITABILITY AND GENETIC RELATIONSHIPS OF CORTICAL SURFACE AREA AND THICKNESS IN THE MEXICAN POPULATION”, Jornadas 
Académicas del Instituto de Neurobiología, UNAM, 2023, Créditos en: 30 Jornadas Académicas del Instituto de Neurobiología UNAM, noviembre de 2023.

\item Rocha-García, M., González-Pérez, E. G., Alfaro-Moreno, J., Carranza-Aguilar, C. J. and Alacuter, S, “CAMBIOS EN LA PLASTICIDAD CEREBRAL EN EL HIPOCAMPO DE 
ROEDORES EXPUESTOS A EJERCICIO AERÓBICO EXPLORADOS A TRAVÉS DE INMUNOHISTOQUÍMICA”, Jornadas Académicas del Instituto de Neurobiología, UNAM, 2023, Créditos en: 
30 Jornadas Académicas del Instituto de Neurobiología UNAM, noviembre de 2023.

\item Serrano-Ramírez, M. S., Rasgado-Toledo, J., Medina-Sánchez, D., Elizarrarás-Herrera, A. D., Ángeles-Valdez, D., Carranza-Aguilar, C. J. and 
Garza-Villarreal, E. 
A, “ALTERACIONES CEREBELOSAS Y COGNITIVAS INDUCIDAS POR LA AUTOADMINISTRACIÓN CRÓNICA DE MORFINA EN RATAS WISTAR MACHO”, Jornadas Académicas del Instituto de 
Neurobiología, UNAM, 2023, Créditos en: 30 Jornadas Académicas del Instituto de Neurobiología UNAM, noviembre de 2023.

\item Vázquez, A., Garay, E., Arellano O., R. and Cisneros-Mejorado, A., “SYSTEMIC TREATMENT WITH THE STEROID GANAXOLONE INDUCES HETEROGENEOUS EFFECTS IN BRAIN 
FOLLOWING CUPRIZONE-INDUCED DEMYELINATION”, Jornadas Académicas del Instituto de Neurobiología, UNAM, 2023, Créditos en: 30 Jornadas Académicas del Instituto de 
Neurobiología UNAM, noviembre de 2023.

\item González-Pérez, E., Ortíz-Retana, J., Alfaro-Moreno, J., Rocha-García, M., Espinosa-Mendez, I., Gasca-Martinez, D., Carranza-Aguilar, C. J. and 
Alcauter-Solórzano, S., “EFECTOS DEL AMBIENTE ENRIQUECIDO EN EL COMPORTAMIENTO, LA CONECTIVIDAD FUNCIONAL Y LA MORFOLOGÍA CEREBRAL DE ROEDORES DESDE LA INFANCIA 
HASTA LA ADULTEZ TEMPRANA”, Jornadas Académicas del Instituto de Neurobiología, UNAM, 2023, Créditos en: 30 Jornadas Académicas del Instituto de Neurobiología 
UNAM, noviembre de 2023.

\item Ramírez-González, D., Román-López, T.V., Sánchez-Moncada, C.I., Espinosa-Méndez, I. M., García-Vilchis, B., Díaz-Téllez, X., Domínguez-Frausto, C. A., 
Murillo-Lechuga, V., López-Camaño, X. J., Guzmán-Tenorio, G. E., Robles-Rodríguez, G. D., Ortiz-Tapia, E. B., Piña-Hernández, A., AldanaAssad, O., Medina-Rivera, 
A., Ruiz-Contreras, A. E., Rentería, M. and Alcauter, S., “AVANCES DEL REGISTRO MEXICANO DE GEMELOS (TWINSMX) PARA LA CARACTERIZACIÓN DE LA HEREDABILIDAD DE 
FENOTIPOS CONDUCTUALES Y DE FUNCIONAMIENTO CEREBRAL”, Jornadas Académicas del Instituto de Neurobiología, UNAM, 2023, Créditos en: 30 Jornadas Académicas del 
Instituto de Neurobiología UNAM, noviembre de 2023.

\item Cisneros-Mejorado, A., Ordaz, R. P., Garay, E., and Arellano-Ostoa, R, “CARBOLINES THAT ENHANCE GABAAR RESPONSE EXPRESSED IN OLIGODENDROCYTES PROMOTE 
REMYELINATION IN AN IN VIVO RAT MODEL OF FOCAL DEMYELINATION”, Jornadas Académicas del Instituto de Neurobiología, UNAM, 2023, Créditos en: 30 Jornadas 
Académicas del Instituto de Neurobiología UNAM, noviembre de 2023.

\item Hernández, A., Cisneros-Mejorado, A., Sierra-Camacho, J. J., Arellano, R. O. and MartínezTorres, A., “NEURAL PROGENITOR CELL TRANSPLANTATION FOR THE 
REGENERATION 
THERAPY IN A HEMI-PARKINSON DISEASE MODEL”, Jornadas Académicas del Instituto de Neurobiología, UNAM, 2023, Créditos en: 30 Jornadas Académicas del Instituto de 
Neurobiología UNAM, noviembre de 2023.

\item Trujillo-Villarreal, L.A., Cruz-Carrillo, G., Angeles-Valdez, D., Garza-Villarreal, E. and Camacho-Morales, A., “EFECTO TRANSGENERACIONAL DE LA 
PROGRAMACIÓN 
FETAL POR DIETA MATERNA SOBRE LA ESTRUCTURA CEREBRAL Y SU ASOCIACIÓN A CONDUCTAS SIMILARES A LA ANSIEDAD EN RATAS MACHO”, Académicas del Instituto de 
Neurobiología, UNAM, 2023, Créditos en: 30 Jornadas Académicas del Instituto de Neurobiología UNAM, noviembre de 2023.

\item Rasgado-Toledo, J., Angeles-Valdez, D., Maya-Arteaga, J. P., Carranza-Aguilar, C., Lopez-Castro, A., Trujillo-Villarreal, L., Serrano, M. S., 
Medina-Sánchez, D., 
Elizarrarás-Herrera, A. D. and GarzaVillarreal, E., “STRESS INCREASES ETHANOL CONSUMPTION IN THE ACUTE PHASE BUT NOT IN THE CHRONIC OF AN IA2BC MODEL”, Jornadas 
Académicas del Instituto de Neurobiología, UNAM, 2023, Créditos en: 30 Jornadas Académicas del Instituto de Neurobiología UNAM, noviembre de 2023.
Rasgado Toledo, J., Ángeles Valdez, D., Carranza Aguilar, C., Maya Arteaga, J. P., Ortuzar, D., López Castro, A. and Garza Villarreal, E. A., "EL ESTRÉS AUMENTA 
EL CONSUMO DE ETANOL EN LA FASE AGUDA PERO NO EN LA FASE CRÓNICA DE UN MODELO DE ELECCIÓN DE 2 BOTELLAS DE ACCESO INTERMITENTE EN RATAS WISTAR", XXIX Jornadas 
Académicas del Instituto de Neurobiología, Créditos en: Congreso local, septiembre de 2022.

\item Vázquez, A., Vélez Uriza, F., Garay, E., Arellano, Ostoa, R. and Cisneros Mejorado, A., "IMPROVEMENT OF REMYELINATION IN DEMYELINATED CENTRAL NERVOUS 
SYSTEM 
USING GANAXOLONE IN THE CUPRIZONE MICE MODEL OF MULTIPLE SCLEROSIS", XXIX Jornadas Académicas del Instituto de Neurobiología, Créditos en: Congreso local, 
septiembre de 2022.

\item López Gutiérrez, M.F., Gracia Tabuenca, Z., Ortiz, J., Camacho, F., Paredes, R.G., Young, L. J., Diaz, N.F., Alcauter, S. and Portillo W., "CAMBIOS EN LA 
CONECTIVIDAD FUNCIONAL CEREBRAL POR LA CRIANZA MONOPARENTAL EN EL TOPILLO DE LA PRADERA", XXIX Jornadas Académicas del Instituto de Neurobiología, Créditos en: 
Congreso local, septiembre de 2022.

\item Villaseñor, P.J., Aquiles, A., Luna Munguia, H., Larriva Sahd, J. and Concha, L., "ANÁLISIS LONGITUDINAL DE LA CORTEZA CEREBRAL EN UN MODELO ANIMAL DE 
DISPLASIA 
CORTICAL", XXIX Jornadas Académicas del Instituto de Neurobiología, Créditos en: Congreso local, septiembre de 2022.

\item Coutiño, D., Hidalgo Flores, F.J., Gasca Martínez, D., Concha, L., Luna Munguia, H., "ANÁLISIS LONGITUDINAL DE LOS CAMBIOS MICROESTRUCTURALES EN FIMBRIA E 
HIPOCAMPO TRAS LA LESIÓN DEL SEPTUM MEDIAL MEDIANTE IMÁGENES DE DIFUSIÓN", XXIX Jornadas Académicas del Instituto de Neurobiología, Créditos en: Congreso local, 
septiembre de 2022.

\item Ocampo Ruiz, A. L., Dimas Rufino, M. A., Castillo, X., Vázquez Carrillo, D., Dena Beltrán, J. L., Garay, E., Martínez de la Escalera, G., Clapp, C., 
Arellano, 
R., Cisneros Mejorado, A. and Macotela, Yazmín", PROLACTIN RECEPTOR DEFICIENCY PROMOTES HYPOMYELINATION IN THE CORPUS CALLOSUM DURING POSTNATAL DEVELOPMENT IN 
MICE", XXIX Jornadas Académicas del Instituto de Neurobiología, Créditos en: Congreso local, septiembre de 2022.

\item Munguía-Villanueva Deyanira, Cisneros-Mejorado Abraham, Arellano O. Rogelio”, MOUSE CORPUS CALLOSUM DEMYELINATION EVALUATED BY DIFFUSION WEIGHTED MAGNETIC 
RESONANCE IMAGING”, Créditos en: Jornadas Académicas INB, noviembre de 2021.

\item Romero Santiago, S., Cisneros Mejorado, A. and Arellano, R, O, “CURSO TEMPORAL DE LA MIELINIZACIÓN EN LA ETAPA POSNATAL DE RATÓN EVALUADO CON RESONANCIA 
MAGNÉTICA E HISTOLOGÍA”, Créditos en: Jornadas Académicas INB, noviembre de 2021.

\item Ocampo Ruiz, A. L., Arellano O, R., Garay, E., Martínez de la Escalera, G., Clapp C., Cisneros Mejorado, A.. and Macotela, Y., “PROLACTIN RECEPTOR 
DEFICIENCY 
PROMOTES HYPOMYELINATION IN THE DEVELOPING CENTRAL NERVOUS SYSTEM OF MICE”, Créditos en: Jornadas Académicas INB, noviembre de 2021.

\item Cortes, G., B. A., Regalado, M., Concha, L. and Luna Munguía, H, “ANÁLISIS LONGITUDINAL DE LOS CAMBIOS DEL SISTEMA LÍMBICO TRAS LA MODULACIÓN DE LA 
VÍA 
SEPTO-HIPOCAMPAL EN UN MODELO DE EPILEPSIA”, , Créditos en: Jornadas Académicas INB, noviembre de 2021.

\item Lazcano, I., Cisneros Mejorado, A., Hernández, Y., Ortiz Retana, J., Arellano, R., Concha, L. and Orozco, A, “DIFERENCIAS EN EL SISTEMA NERVIOSO CENTRAL 
ENTRE 
UN AJOLOTE PRE- Y POST- METAMÓRFICO INDUCIDO POR HORMONAS TIROIDEAS”, Créditos en: Jornadas Académicas INB, noviembre de 2021.

\item García Saldivar, P., De León Andrez, C., Ayala, Y.A., Prado, L., Concha, L. and Merchant, H, “THE ROLE OF SUPERFICIAL WHITE MATTER IN THE SENSORIMOTOR 
SYNCHRONIZATION”, Créditos en: Jornadas Académicas INB, noviembre de 2021.

\item Vélez Uriza, F. Z., Cisneros Mejorado, A., Garay, E. and Arellano, R. O, “DEMYELINATION–REMYELINATION OF CEREBELLAR PEDUNCLES OF MICE EVALUATED WITH MRI 
AND 
BLACKGOLDII STAIN “, Créditos en: Jornadas Académicas INB, noviembre de 2021.

\item Valles Capetillo, E. and Giordano, M, “THE NEUROCOGNITION OF SOCIAL COMMUNICATION”, Créditos en: Jornadas Académicas INB, noviembre de 2021.

\item Cisneros Mejorado, A., Garay, E., Moctezuma, J. P. and Arellano, R, “B-CARBOLINES IMPROVE REPAIR OF WHITE MATTER INJURY IN A FOCAL DEMYELINATION MURINE 
MODEL”, 
Créditos en: Jornadas Académicas INB, noviembre de 2021.

\item Delgado-Herrera, M. and Giordano M., “How to study the brain while lying? A systematic review”, 23-27, Créditos en: Jornadas Académicas INB, septiembre de 
2019.

\item Atilano-Barbosa, D., Passaye Alcaraz, E.H. and Mercadillo-Caballero, R. E., “Cognitive function and brain morphometry in Mexican workers occupational 
exposed to 
solvents”, 23-27, Créditos en: Jornadas Académicas INB, septiembre de 2019.

\item Garcia-Saldivar, P., De León-Anfrez, C., Ayala Y., A., Prado, L., Concha L. and Merchant, H., “The role of the superficial white matter in the 
sensoriomotor 
synchronization”, 23-27, Créditos en: Jornadas Académicas INB, noviembre de 2019.

\item Olalde-Mathieu, V.E, Sassi, F., Reyes-Aguilar, A., Mercadillo, R.E., Alcauter, S., Barrios, F.A., “Psychotherapists present differences related to empathy 
sub-processes when compared to nonpsychotherapists”, Jornadas Académicas del INB-UNAM, 23-27 , Créditos en: Otro, septiembre de 2019.

\item García-Saldivar, P., De León, C., Prado, L., Concha, L., Merchant H., "LONGITUDINAL STRUCTURAL CHANGES IN GRAY MATTER OF THE RHESUSMONKEY ASSOCIATED TO 
TRAINING 
IN A SENSORIMOTOR SYNCHRONIZATION TASK", 25 Jornadas Académicas, Instituto de Neurobiología, UNAM, Septiembre 24-28, Créditos en: Jornadas Académicas INB, 
noviembre de 2018.

\item Barbosa-Luna, M., Ricardo-Garcell, J., Alcauter-Solorzano, S., Solis-Vivanco, R., GarcíaHernández, S. and Pasaye-Alcaraz, E.H., "MULTIDIMENSIONAL 
CHARACTERIZATION OF OBSESSIVE-COMPULSIVEDISORDER. A CLINICAL, NEUROPSYCHOLOGICAL AND NEUROIMAGEN PERSPECTIVE", 25 Jornadas Académicas, Instituto de 
Neurobiología, UNAM, Septiembre 24-28, Créditos en: Jornadas Académicas INB, septiembre de 2018.

\item Fajardo-Valdez, A., Rodríguez-Cruces, R., Rosas-Carrera, A.E., Concha L, Pasaye-Alcaraz, E., "ALTERATIONS IN RESTING STATE FMRI CONNECTIVITY AND 
COGNITIVEPERFORMANCE IN TEMPORAL LOBE EPILEPSY PATIENTS", 25 Jornadas Académicas, Instituto de Neurobiología, UNAM, Septiembre 24-28, Créditos en: Jornadas 
Académicas INB, noviembre de 2018.

\item Gallego, R. J., Corsi Cabrera, M., Garcell Ricardo, J., Alcauter Solorzano, S., Pasaye Alcaraz, E., "ADRESSING MRI DATA RELIABILITY DURING SIMULTANEOUS 
EEG-FMRIRECORDING", 25 Jornadas Académicas, Instituto de Neurobiología, UNAM, Septiembre 24-28, Créditos en: Jornadas Académicas INB, noviembre de 2018.

\item Gracia-Tabuenca, Z., Moreno, B., Barrios, F., Alcauter, S., "CHARACTERIZATION OF THE FUNCTIONAL BRAIN NETWORK IN ADOLESCENCE:A FUNCTIONAL SEGREGATION 
PROCESS 
THAT PREDICTS MEMORY PERFORMANCE", 25 Jornadas Académicas, Instituto de Neurobiología, UNAM, Septiembre 24-28, Créditos en: Jornadas Académicas INB, noviembre de 
2018.

\item Carrillo-Peña, AV, Valles-Capetillo, DE, Licea-Haquet, GL, and Giordano, M., "Effect of familiarity in the interpretation of pragmatic language in healty 
mexican 
subjects", Créditos en: Jornadas Académicas INB, septiembre de 2016.

\item Liliana García, "Association between functional connectivity and language abilities in school-age children", 21 Annual Meeting of the Organization for 
Human 
Brain Mapping, Honolulu Hawaii, Créditos en: Otro, diciembre de 2015.

\item Gracia, Z., "Functional Connectivity asymmetries in School-Age Children: Sex and Cognitive Performance Effects", 21 Annual Meeting of the Organization for 
Human 
Brain Mapping, Honolulu Hawaii, Créditos en: Otro, junio de 2015.

\item Gracia Z, "Asimetrías en conectividad funcional cerebral en niños de edad escolar", 22 Jornadas Académicas del Instituto de Neurobiología, UNAM, Créditos 
en: 
Apoyo técnico, octubre de 2015.

\item  Bauer, C.C.C., "The after effect of meditation: Increased resting-state functional connectivity after a single 20 min meditation epoch", 21 Annual Meeting 
of the 
Organization for Human Brain Mapping, Honolulu Hawaii, Créditos en: Otro, junio de 2015.

\item Hernandez-Rios, E.N, Barrios, A.F., Utilización de programas para el analisís de imágenes como apoyo a la investigación en el INB, Créditos en: Otro, 2009.

\item Aranda-López, N., Delgado, G., Aceves, C., Anguiano-Serrando, B. “Actividad desyodativa tipo I (dio 1) durante la ontogenia del epidídimo”, Jornadas del 
Instituto de Neurobiología, Sep. 19-23, Querétaro, México, Créditos en: Otro, 2005.

\item García-Solís, P., Anguiano, B., Delgado, G., Aceves, C., “Diferencias de señalización y captura del yodo molecular (I2) en la glándula mamaria lactante, 
virgen y 
neoplástica”, Jornadas del Instituto de Neurobiología, Sep. 19-23, Querétaro, México, Créditos en: Otro, 2005.

\item Castillo, C. G., Mendoza, M. S., Falcón, A., Aguilar, M. B., Giordano, M., “Cuantificación de gaba liberado por una línea celular estriatal inmortalizada y 
trasnfectada con el cdna del gad67h”, Jornadas del Instituto de Neurobiología, Sep. 19-23, Querétaro, México, Créditos en: Otro, 2005.

\item León-Jacinto, U., Quitarte, G., Serafín-López, N., Aguilar-Vázquez, A., Beltrán-Campos, V., Prado-Alcalá, Díaz-Miranda, S., Y., “Cambios en densidad y tipo 
de 
espinas dendríticas en el hipocampo asociados a una tarea de sobreentrenamiento”, Jornadas del Instituto de Neurobiología, Sep. 19-23, Querétaro, México, 
Créditos en: Otro, 2005.

\item Arroyo-Helguera, O., Aceves, C., “Captura y efecto atiproliferativo del yodo molecular (I2) en cultivos celulares de cancer mamario”, Jornadas del 
Instituto de 
Neurobiología, Sep. 19-23, Querétaro, México, Créditos en: Otro, 2005.

\item Sandoval-Minero, M. T., Varela-Echavarría, A., “Interacciones axonales entre neuronas decusantes como un mecanismo para el cruce de la línea media en 
vertebrados”, Jornadas del Instituto de Neurobiología, Sep. 19-23, Querétaro, México, Créditos en: Otro, 2005.

\item López-Juárez, A., Aceves, C., Delgado, G., Anguiano, B., “La actividad sexual incrementa la producción local triyodotironina (T3) en el lóbulo ventral de 
la 
próstata”, Jornadas del Instituto de Neurobiología, Sep. 19-23, Querétaro, México, Créditos en: Otro, 2005.

\item Beltrán-Campos, V., Rodríguez-Santillán, E., Quitarte, G., Serafín-López, N., Aguilar-Vázquez, A., Díaz’Miranda, S. Y., “Influencia estrogémica con la 
densidad 
de espinas dendríticas de las células piramidales del CA1 del hipocampo”, Jornadas del Instituto de Neurobiología, Sep. 19-23, Querétaro, México, Créditos en: 
Otro, 2005.

\item Frías, C., Torrero, C., Regalado, M., Salas, M., “Desarrrollo de las células mitrales en ratas desnutridas posnatalmente: posibles alteraciones 
funcionales”, 
Jornadas del Instituto de Neurobiología, Sep. 19-23, Querétaro, México, Créditos en: Otro, 2005.

\item Tinajero, A., Ramos., M., E., Morales, T., “Respuesta diferencial de los núcleos hipotalámicos, paraventricular y supraóptico, al estrés osmótico durantes 
el 
ciclo estral de la rata”, Jornadas del Instituto de Neurobiología, Sep. 19-23, Querétaro, México, Créditos en: Otro, 2005.

\item Aguilar-Vázquez, A., Martínez-Jaramillo, J., Beltrán-Campos, V., Díaz-Miranda, S. Y., “Influencia de los estrógenos en la plasticidad del hipocampo: 
espinogénesis”, Jornadas del Instituto de Neurobiología, Sep. 19-23, Querétaro, México, Créditos en: Otro, 2005.

\item Ortiz, J. J., harmony, T., Fernández-Bouzas, A., Barrios, F., A., “Espectroscopia por resonancia magnética en 1.0T en infantes”, Jornadas del Instituto de 
Neurobiología, Sep. 19-23, Querétaro, México, Créditos en: Otro, 2005.

\item Quiroz, C., Díaz del Guante, M. A., Garín-Aguilar, M. E., Quirarte, G. L., Prado-Alcalá R., “Inactivación del Hipocampo Dorsal: Efectos sobre la retención 
de la 
memoria evocada de dos diferentes niveles de reforzamiento”, Jornadas del Instituto de Neurobiología, Sep. 20-24, Querétaro, México, Créditos en: Otro, 2004.

\item Delgado, G., Anguiano, B., Aceves, C. “La interacción hormonas tiroideas-acido retinoico es necesaria para antener la diferenciación del epitelio en 
carcinomas 
mamarios”, Jornadas del Instituto de Neurobiología, Sep. 20-24, Querétaro, México, Créditos en: Otro, 2004.

\item Sandoval-Minero, M.T., Varela-Echavarría, A., “Regulación de la proyección axonal de neuronas decusantes en el rombencéfalo caudal”, Jornadas del Instituto 
de 
Neurobiología, 20 al 24 de Sep, Créditos en: Otro, 2004.

\item Martínez, Y., Díaz-Cintra, S., Díaz del Guante, M., Aguilar A., Cárabez-Trejo, A., Quirarte G. L., Prado-Alcalá R., “Hipocampo Senil: Un estudio conductal 
y 
molfológico”, Jornadas del Instituto de Neurobiología, Sep. 20-24, Querétaro, México., Créditos en: Otro, 2004.

\item Cepeda Nieto, A. C., Pfaff, S. , Varela-Echavarría, A., “Diferenciación y proyección de las neuronas ticuloespinales romboencefálicas”, Jornadas del 
Instituto de 
Neurobiología, Sep. 20-24, Querétaro, México, Créditos en: Otro, 2004.

\item Díaz del Guante, M.A., Quiroz, C., Garín-Aguilar, M.E., Quirarte, G.L. y Prado-Alcalá R. “Bloqueo temporal de la amígdala: ¿Reconsolidación o evocación?”, 
Jornadas del Instituto de Neurobiología, Sep. 20-24, Querétaro, México, Créditos en: Otro, 2004.

\item Soriano, O., Anguiano, B., Aceves, C., “Inhibición del efecto protector del lugol por progesterona en el cáncer mamario inducido por 
7,12-dimetilbenzeno[a]antraceno (DMBA)”, Jornadas del Instituto de Neurobiología, Sep. 20-24, Querétaro, México, Créditos en: Otro, 2004.

\item Rojas-Huidobro, R., Aceves C., “Efecto Protector de T4 y KI en la inducción del cáncer mamario por 7, 12-dimetilbenzo[a] antraceno (DMBA) en ratas 
púberes”, 
Jornadas del Instituto de Neurobiología, Sep. 22-26, Querétaro, México, Créditos en: Otro, 2003.

\item Aguilar Vázquez, A., Granados-Rojas, L. y Díaz-Cintra S., “Estudio inmunocitoquimico de las células gabaergicas en el hipocampo de la rata con malnutrición 
hipoproteínica crónica de 30 y 60 días de edad”, Jornadas del Instituto de Neurobiología, Sep. 22-26, Querétaro, México, Créditos en: Otro, 2003.

\item Castillo, R. A., Villalobos, M. S., Galindo, L. E., Quirarte G. L., Prado-Alcalá R. “Efectos de la inactivación reversible del estriado sobre la evocación 
de la 
memoria de un aprendizaje incrementado”, Jornadas del Instituto de Neurobiología, Sep. 22-26, Querétaro, México, Créditos en: Otro, 2003.

\item Quiroz C., Quirarte G. L., Prado-Alcalá R., “El sobrerreforzamiento impide la deficiencia en la retención inducida por la aplicación pre-entrenamiento de 
TTX en 
el hipocampo”, Jornadas del Instituto de Neurobiología, Sep. 22-26, Querétaro, México, Créditos en: Otro, 2003.

\item Quiroz, C., Garín-Aguilar, M. E., Morales T., Quirarte G. L., Prado-Alcalá R. “Efecto de la tetrodotoxina (TTX) en el hipocampo de ratas: expresión de 
C-FOS 
inducida por ácido kaínico (AK) y pentilenetetrazol (PTZ) como indicador de inactivación”, Jornadas del Instituto de Neurobiología, Sep. 22-26, Querétaro, 
México, Créditos en: Otro, 2003.

\item Teja, I. S., Prado-Alcalá R. Quirarte, G. L., “Participación de los receptores a corticosterona estriatales en la memoria de una tarea de evitación 
inhibitoria”, 
Jornadas del Instituto de Neurobiología, Sep. 22-26, Querétaro, México, Créditos en: Otro, 2003.

\item Granados-Rojas, L., Sánchez, A., Aguilar, A., Quirarte G. L., Prado-Alcalá R., Díaz-Cintra, S., “Crecimiento de fibras musgosas después del 
sobreentrenamiento en 
el laberinto acuático de morris en ratas con malnutrición hipoproteínica prenatal, crónica y postnatal”, Jornadas del Instituto de Neurobiología, Sep. 22-26, 
Querétaro, México, Créditos en: Otro, 2003.
\item Sánchez, A., Granados-Rojas, L., Pineda, T., Aguilar, A., Prado-Alcalá, R. Quirarte, G. L., DíazCintra, S., “Efecto del sobre entrenamiento en el laberinto 
acuático de morris sobre la plasticidad sináptica de las fibras musgosas en ratas malnutridas”, Jornadas del Instituto de Neurobiología, Sep. 23-27, Querétaro, 
México, Créditos en: Otro, 2002.

\item Aguilar-Vázquez, A., Granados-Rojas, L., Díaz-Cintra S., “Marcaje in vitro de vesículas sinápticas activas con FMI-43 en el hipocampo de ratas de 220 días 
con 
malnutrición prenatal y crónica”, Jornadas del Instituto de Neurobiología, Sep. 23-27, Querétaro, México, Créditos en: Otro, 2002.
Moreno-Ocaña, G., Martínez-Cabrera, G., Varela-Echavarría, A., A., Condés-Lara, M., LarrivaSahd, J., “Neuronas interfasciculares de la comisura anterior: 
Caracterización citológica e interacciones locales”, Jornadas del Instituto de Neurobiología, Sep. 23-27, Querétaro, México, Créditos en: Otro, 2002.

\item Reyes-Haro, D., Miledi, R., García-Colunga, J., “Transmisión serotonérgica en el cuerpo calloso de la rata”, Jornadas del Instituto de Neurobiología, Sep. 
23-27, 
Querétaro, México, Créditos en: Otro, 2002.

\end{enumerate}


