\textbf{Agradecimientos en Tesis de Maestría}

\hfill

\begin{enumerate}

\item Gaspar Martinez, E., “Diferencias volumétricas en estructuras cerebrales de pacientes con enfermedad de Parkinson, 
prodrómicos y controles, y su asociación con perfiles cognitivos”, Maestría en Ciencias (Neurobiología), agosto de 2024.

\item Mares Román, C.P., “Caracterización de la sincronía auditivo-motora para distintos estímulos y efectores”, Maestría 
en 
Ciencias (Neurobiología), UNAM-Juriquilla, diciembre de 2022.

\item Reyes González, I.D., “Estudio de los correlates neutrales de la percepción emocional por análisis de patrones en 
multitud de voxels”, noviembre de 2021.

\item Brzezinski Rittner, A., “Estudio Longitudinal de la conectividad funcional en respuesta a una tarea de memoria de 
trabajo, n-back, por interacciones psico-fisiológicas”, Maestria en Ciencias (Neurobiología), INB-UNAM, agosto de 2019.

\item Catillo López, G., “Relación de anormalidades en sustancia blanca y acumulaciones de alfasinucleína en biopsias de 
piel 
de pacientes con parkinson”, Maestria en Ciencias (Neurobiología), mayo de 2019.

\item Rasgado Toledo, J., “La modulación de la intención comunicativa mediante la expresión facial y su correlato neural”, 
Maestria en Ciencias (Neurobiología), INB-UNAM, octubre de 2019.

\item López Gutiérrez, M.F., “Efectos de la formación de vínculos de pareja sobre la conectividad funcional cerebral en 
microtus ochrogaster”, Maestria en Ciencias Físicas (Neurobiología), INBUNAM, junio de 2019.

\item Camacho Herrera, C., “Caracterización de la variabilidad en la localización de la corteza motora suplementaria usando 
referencias anatómicas extracraneales”, Maestria en Ciencias Físicas (Neurobiología), INB-UNAM, junio de 2019.

\item Rojas Vite, G., "Evaluación de la microestructura de la substancia blanca en regiones de cruce de fibras", Maestría 
en 
Ciencias (Neurobiología), UNAM, noviembre de 2018.

\item Zamora Ursulo, M.A., "El correlato neural de la comprensión de la metáfora en adultos hispanohablantes", Maestría en 
Ciencias (Neurobiología), UNAM, noviembre de 2018.

\item Martínez López, A. Y., "Correlatos Neuronales de la atención sostenida evaluada por estados de conectividad funcional 
dinámica", Maestría en Ciencias (Neurobiología), UNAM, noviembre de 2018.

\item Carrillo-Peña, A.V., “Estudio de los Refranes Mexicanos y sus Correlatos Conductuales y Neurales”, 
Ciencias(Neurobiología), UNAM-Juriquilla, diciembre de 2017.

\item Rueda Zarazúa, Bertha G., “Análisis de la Carga Mutacional en el Genoma Mitocondrial enFibroblastos de Pulpa Dental 
Humana”, julio de 2017.

\item Navarrete-Acevedo, Norma E., "Maduración de las Propiedades de Conectividad Funcional Cerebral en la Adolescencia", julio de 2016.

\item Olalde-Mathieu, Víctor E., "Caracterización de la Conectividad Funcional Cerebral relacionada a Componentes de la 
Respuesta Empática", agosto de 2016.

\item Circe A. Wilke-Quintero, "Participación de la Vía Dorsal auditiva en la Percepción del ritmo en la música, evaluada 
mediante IRMf",Instituto de Neurobiología, UNAM, agosto de 2015.

\item Jiménez Valverde L.O. “Descripción de la Actividad Cortical asociada a la memoria de trabajo en pacientes con 
epilepsia 
del lóbulo temporal Un estudio de resonancia magnética funcional”, Tesis, Maestría en Ciencias (Neurobiología), UNAM, 
septiembre de 2014.

\item Manni Zepeda M., “Correlatos Neurales de la Observación Natural de un Video con Carga Emocional”, Tesis, Maestría en 
Ciencias (Neurobiología), UNAM, septiembre de 2014.

\item Cuaya Retana, L. V., "Correlatos Cerebrales de una Decisión Causal evaluados mediante Resonancia Magnética Funcional, 
septiembre de 2013.

\item Rodríguez Nieto G., “Estudio de la Respuesta Hemodinámica en la Ínsula y Corteza Prefrontal en un Modelo de Compasión 
por Resonancia Magnética Funcional”, Tesis, Maestría en Ciencias (Neurobiología), septiembre de 2013.

\item Cruz García, M. G., "Efectos de la exposición al arsénico sobre los transportadores de glucosa y el receptor de 
insulina en el hipocampo de ratones macho de la cepa C57BL-6J", Instituto de Neurobiología, UNAM, agosto de 2012.

\item Domínguez Vargas, A. U., "Análisis del neurodesarrollo de los movimientos oculares en niños con leucomalacia 
periventricular", Instituto de Neurobiología, octubre de 2011.

\item Cruz García M.G., “Efectos de la exposición al arsénico sobre los transportadores de glucosa y el receptor de 
insulina 
en el hipocampo de ratones macho de la cepa C57BL/6J”, Maestro en Ciencias (Neurobiología), noviembre de 2010.

\item Rodríguez-Vidal, Lluviana, Diferencias en el metabolismo cerebral de acuerdo al temperamento en sujetos 
control, UNAM-INB, 2009.

\item Cano Sotomayor, Yoana Daniela, "Efecto del estradiol sobre la neurogénesis en el estriado lesionado con ácido kaínico 
en la rata hembra adulta", noviembre de 2007.

\item Cipriano, Margarita de Jesús, Diferencias entre imágenes anisotrópicas por difusión de sujetos control y pacientes 
con 
enfermedad de parkinson idiopática, UNAM-INB, 2007.

\item Mercadillo Caballero, Roberto Emmanuele, "Correlatos Cerebrales de la Percepción del Sufrimiento en otro: Un estudio 
por Resonancia Magnética Funcional", mayo de 2007.

\item Díaz Trujillo, A., “Identificación de genes asociados con la consolidación de la memoria en el neoestriado mediante 
microarreglos de cDNA”, Maestría en ciencias (neurobiología), UNAM-INB, 2005.

\item Herrera-Alarcón, J., “Evaluación del desarrollo de las células de sertoli en ovinos durante el período postnatal”, 
Maestría en ciencias (neurobiología), UNAM-INB, 2005.

\item Arroyo-Helguera, O. E., “Papel regulatorio de la región 3 no traducible de los RNA mensajeros de la enzima desyodasa 
tipo 1: Efecto sobre su traducción y estabilidad”, Maestría en ciencias (neurobiología), UNAM-INB, 2004.

\item Galindo-Martínez, L. E., “Efecto del agotamiento de serotonina cerebral sobre los procesos de adquisición y retención 
en una tarea de evitación activa”, Maestría en ciencias (neurobiología), UNAM-INB, Créditos en: Tesis Maestría, 2003.

\item Morales-Vega, D. A., “Estudio de la activación medular por medio de la resonancia magnética funcional”, Maestría en 
ciencias (neurobiología), UNAM-INB, 2003.

\item González-Torres, M., A., “Localización de los receptores s estriadol en el hipotálamo de la oveja y sus variaciones 
en 
relación con el parto y la conducta maternal”, Maestría en ciencias (neurobiología), UNAM-INB, 2002.

\item Castillo-Martín del Campo, C. G., “Transplante de una línea celular inmortalizada productora de gaba en amígdala de 
la 
rata albina”, Maestría en ciencias (neurobiología), UNAM-INB, 2002.

\item Hurtazo-Oliva, H. A., “Evaluación del funcionamiento del sistema vomeronasal por fos en ratas macho con lesiones del 
área preóptica media del hipotálamo anterior”, Maestría en ciencias (neurobiología), UNAM-INB, 2002.

\item Huerta-Ocampo, I., “Dimorfismo sexual en la comisura anterior de la rata: Estudio morfológico y morfométrico”, 
Maestría 
en ciencias (neurobiología), UNAM-INB, 2001.

\item Mena-Segovia, J., “Efectos de la lesión excitotoxica del estriado sobre la actividad electroencefalografica durante 
el 
ciclo vigilia sueño en la rata albina”, Maestría en ciencias (neurobiología), UNAM-INB, 2001.

\item Hernández-Montiel, H. L., “Control del crecimiento axonal longitudinal en el sistema nervioso central de 
vertebrados”, 
Maestría en ciencias (neurobiología),UNAM-INB, 2001.

\item Geovannini-Acuña, H., “Re-evaluación del papel de la actividad neuronal asociada con el uso y de la densidad de 
inervación periférica en la plasticidad sensoriomodal de la neocorteza de la rata”, Maestría en ciencias (neurobiología), 
UNAM-INB, 2001.

\item Gutiérrez de la Barrera, A., “Efectos del factor de crecimiento I tipo insulina en la muerte celular normal durante 
el 
desarrollo retiniano en roedores”, Maestría en ciencias (neurobiología), UNAMINB, 2001.

\item Uribe-Querol, E., “Efecto directo de esteroides ováricos sobre la secreción de GnRH”, Maestría en ciencias 
(neurobiología), UNAM-INB, Créditos en: Tesis Maestría, 2000.

\item Alonso-Onofre, F., “Diferenciación Funcional entre la región dorsal y ventral del estriado en el aprendizaje 
visuoespecial”, UNAM-INB, Maestría en ciencias (neurobiología), 1999.

\end{enumerate}
