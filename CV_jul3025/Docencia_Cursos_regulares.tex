\textbf{Docencia Cursos Regulares}

\hfill

\begin{enumerate}

\item “Introducción al lenguaje de Programación R”, Modalidad: Presencial, Semestre: 1, Maestría en Ciencias 
Neurobiología, Maestría, Instituto de Neurobiología, Instituto de Neurobiologia, Tipo de programa académico: Escolarizado, 
Año del programa: 2023, Periodos: 2024-2, 2024-1.

\item “Introducción al lenguaje de Programación R”, Modalidad: Presencial, Semestre: 2, Maestía en Ciencias neurobiología, 
Maestría, Instituto de Neurobilogía, Instituto de Neurobiologia, Tipo de programa académico: Escolarizado, Año del 
programa: 2023, Periodos: 2024-1.

\item “Introducción a la programación y computación con Phyton”, Modalidad: Presencial, Semestre: 2, Maestria w¡en Ciencias 
Neurobiología, Maestría, Instituto Neurobiología, Instituto de Neurobiologia, Tipo de programa académico: Escolarizado, Año 
del programa: 2023, Periodos: 2023-2.

\item “Introducción al lenguaje de Programación R”, Modalidad: Presencial, Semestre: 2, Maestría en Ciencias, Neurobiología, 
Maestría, Instituto de Neurobiologia, Instituto de Neurobiologia, Tipo de programa académico: Escolarizado, Año del 
programa: 2023, Periodos: 2023-2.

\item Introducción a la Computación y Lenguaje de Programación R, Modalidad: A Distancia, Semestre: 1, Maestría en Ciencias 
(Neurobiología), Maestría, Instituto de Neurobiología, Instituo de Neurobiologia, Tipo de programa académico: Escolarizado, 
Año del programa: 2022, Periodos: 20222, 2022-1.

\item “Introducción a la Computación y Lenguaje de Programación PYTHON”, Modalidad: A Distancia, Semestre: 1, Maestría en 
Neurobiología, Maestría, Neurobiología, Instituo de Neurobiologia, Tipo de programa académico: Escolarizado, Año del 
programa: 2021, Periodos: 2021-2.

\item “Introducción a la Computación y Lenguaje de Programación R”, Modalidad: A Distancia, Semestre: 1, Maestría en 
Neurobiología, Maestría, Neurobiología, Instituo de Neurobiologia, Tipo de programa académico: Escolarizado, Año del 
programa: 2021, Periodos: 2021-2.

\item ‘Introducción a la Computación y Lenguaje de Programación “R”’, Modalidad: Presencial, Semestre: 2, Maestría en Ciencias 
(Neurobiología), Maestría, Instituto de Neurobiología UNAMJuriquilla, Instituto de Neurobiología, Tipo de programa 
académico: Escolarizado, Año del programa: 2019, Periodos: 2021-1, 2020-2, 2020-1, 2019-2.

\item Introducción al Lenguaje de Programación "R", Modalidad: Presencial, Semestre: 2, Programa Maestría en Ciencias 
(Neurobiología), Maestría, Instituto de Neurobiología UNAM-Juriquilla, INBUNAM, Instituto de Neurobiología UNAM-Juriquilla, 
Tipo de programa académico: Escolarizado, Año del programa: 2018, Periodos: 2019-1.

\item Introducción al Lenguaje de Programación "R", Modalidad: Presencial, Semestre: 1, Programa Maestría en Ciencias 
(Neurobiología), Maestría, Instituto de Neurobiología UNAM-Juriquilla, INBUNAM, Instituto de Neurobiología UNAM-Juriquilla, 
Tipo de programa académico: Escolarizado, Año del programa: 2018, Periodos: 2018-2.

\item “Introducción al Lenguaje de Programación Python”, Modalidad: Presencial, Semestre: 1, Maestría en Ciencias 
(Neurobiología), UNAM, Doctorado, Instituto de Neurobiología UNAMJuriquilla, INB-UNAM, Instituto de Neurobiología 
UNAM-Juriquilla, Tipo de programa académico: Escolarizado, Año del programa: 2018, Periodos: 2018-1.

\item “Introducción al Lenguaje de Programación R”, Modalidad: Presencial, Semestre: 2, Maestría en Ciencias (Neurobiología), 
Doctorado, Instituto de Neurobiología UNAM-Juriquilla, INB-UNAM, Instituto de Neurobiología UNAM-Juriquilla, Tipo de 
programa académico: Escolarizado, Año del programa: 2017, Periodos: 2017-2.

\item “Métodos Numéricos”, Modalidad: Presencial, Semestre: 2, Maestría en Ciencias (Neurobiología), Doctorado, Instituto de 
Neurobiología UNAM-Juriquilla, INB-UNAM, Instituto de Neurobiología UNAM-Juriquilla, Tipo de programa académico: 
Escolarizado, Año del programa: 2017, Periodos: 2017-1.

\item Métodos Numéricos, Modalidad: Presencial, Semestre: 1, Créditos: 4, Maestría en Ciencias (Neurobiología), Maestría, 
Instituto de Neurobiología UNAM-Juriquilla, INB-UNAM, Instituto de Neurobiología UNAM-Juriquilla, Tipo de programa 
académico: Escolarizado, Año del programa: 2016, Periodos: 2017-1.

\item Estadística, Modalidad: Presencial, Semestre: 2, Créditos: 8, Maestría en Ciencias (Neurobiología), Maestría, Instituto de Neurobiología 
UNAM-Juriquilla, INB-UNAM, Instituto de Neurobiología 
UNAM-Juriquilla, Tipo de programa académico: Escolarizado, Año del programa: 2016, Periodos: 2016-2.

\item Estadística, Modalidad: Presencial, Semestre: 2, Maestría en Ciencias (Neurobiología), Maestría, Instituto de 
Neurobiología UNAM-Juriquilla, INB-UNAM, Instituto de Neurobiología UNAM-Juriquilla, Tipo de programa académico: 
Escolarizado, Año del programa: 2015, Periodos: 2015-2.

\item Métodos Numéricos, Modalidad: Presencial, Semestre: 2, Maestría en Ciencias (Neurobiología), Maestría, Instituto de 
Neurobiología UNAM-Juriquilla, INB-UNAM, Instituto de Neurobiología UNAMJuriquilla, Tipo de programa académico: 
Escolarizado, Año del programa: 2015, Periodos: 2015-2.

\item Estadística, Modalidad: Presencial, Semestre: 2, Programa de Maestría en Ciencias (Neurobiología), Maestría, Instituto 
de Neurobiología UNAM-Juriquilla, INB-UNAM, Instituto de Neurobiología UNAM-Juriquilla, Tipo de programa académico: 
Escolarizado, Año del programa: 2014, Periodos: 2015-1.

\item Lenguaje de Programación MatLab, Modalidad: Presencial, Semestre: 1, Maestría en Ciencias (Neurobiología), Maestría, 
Instituto de Neurobiología UNAM-Juriquilla, INB-UNAM, Instituto de Neurobiología UNAM-Juriquilla, Tipo de programa 
académico: Escolarizado, Año del programa: 2012, Periodos: 2012-1.

\item Lenguaje de Programación Matlab, Modalidad: Presencial, Semestre: , Maestría en Ciencias (Neurobiología), Maestría, 
Instituto de Neurobiología UNAM-Juriquilla, INB-UNAM, Instituto de Neurobiología UNAM-Juriquilla, Tipo de programa 
académico: Escolarizado, Año del programa: 2009, Periodos: 2008-1.

\item Lenguaje Pascal y “C”, Modalidad: Presencial, Semestre: 2, Ingeniero en Computación, Licenciatura, Tecnológico de 
Estudios Superiores de Ecatepec, Tipo de programa académico: Escolarizado, Año del programa: 1996, Periodos: Indefinido.

\item Mecánica, Modalidad: Presencial, Semestre: 2, Ingeniero Químico, Licenciatura, Facultad de Química de la UAQ. 
Querétaro, México, Tipo de programa académico: Escolarizado, Año del programa: 2001, Periodos: Indefinido.

\item Mecánica, Modalidad: Presencial, Semestre: 3, Fisíco, Licenciatura, Facultad de Química de la UAQ. Querétaro, México, 
Tipo de programa académico: Escolarizado, Año del programa: 2001, Periodos: Indefinido.

\item Taller de Matemáticas, Modalidad: Presencial, Semestre: , Maestría en Ciencias (Neurobiología), Maestría, Instituto de 
Neurobiología UNAM-Juriquilla, INB-UNAM, Instituto de Neurobiología UNAMJuriquilla, Tipo de programa académico: 
Escolarizado, Año del programa: 1997, Periodos: Indefinido.

\item Lenguaje de Programación MatLab, Modalidad: Presencial, Semestre: 1, Maestría en Ciencias (Neurobiología), UNAM, 
Maestría, Instituto de Neurobiología UNAM-Juriquilla, INB-UNAM, Instituto de Neurobiología UNAM-Juriquilla, Tipo de 
programa académico: Escolarizado, Año del programa: 2010, Periodos: 2010-1.


\end{enumerate}
