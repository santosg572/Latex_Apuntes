\textbf{Agradecimientos en Congresos Nacionales}

\hfill

\begin{enumerate}

\item Cisneros-Mejorado, A., Aguilera, P., Peralta-Arrieta, I., Gega-Miranda, A., Hernández-Cruz, A., Alquisiras-Burgos, I., 
“Klotho Anti-Aging Protein Deficiency Impact on Cerebral Cytoarchitecture”, V Congreso Nacional de Neurobiología y de la 
Sociedad Mexicana de Bioquímica, Morelia, Michoacán, del 13 al 17, 2024.

\item Cisneros-Mejorado, A., Vétez Uriza, F., Pablo Ordaz, R., Cárdenas-Pérez, G., Garay, E., Arellano, R. O., “B-carboline B-CCB, a potent and 
specific potentiator of GABA, receptor expressed in oligodendrocytes, promote 
remyelination in an in vivo demyelination model”, V Congreso Nacional de Neurobiología y de la Sociedad Mexicana de 
Bioquímica, Morelia, Michoacán, del 13 al 17 de abril de 2024.

\item Cisneros-Mejorado, A., Garay, E., Arellano, R.O., "beta-carbolines in focal and systemic demyelination-remyelination 
models", 4TH Symposium on Physiology \& Pathology of Neuroglia, octubre de 2022.

\item Angulo Perkins A. A., López Carrera E., Barrios Álvarez F. A., Concha Loyola L., “EL LADO OCULTO DE LA IMAGEN POR 
RESONANCIA MAGNÉTICA: TÉCNICAS DE ANÁLISIS EN UN ESTUDIO DE MÚSICA Y MÚSICOS”, XIX Jornadas Académicas del Instituto de 
Neurobiología, UNAM, Septiembre 17-21, 2012.

\item Concha L., Angulo-Perkins A., Peretz A., Armony J., Barrios F.A., "Estímulos musicales provocan activaciones preferentes 
en el lóbulo temporal: Identificación mediante Resonancia Magnética Funcional", XVIII Jornadas Académicas del Instituto de 
Neurobiología,UNAM, Septiembre 19-23, 2011.

\item Léon-Jacinto U., Quirarte, G.L., Aguilar-Vázquez A.R., Serafín N., Beltrán-Campos, V., PradoAlcalá, R.A., Díaz-Miranda, 
S.Y.,"Cambios en el tipo de espinas dendrítis basales en el CA3 del hipocampo asociados al aprendizaje espacial 
incrementado",XLIX Congreso Nacional de Ciencias Fisiologicas; Querétaro, Qro., 2006.

\item Beltrán-Campos, V., Quirarte, G.L., Ramírez-Amaya, V.,Aguilar-Vázquez, A.R., Serafín, N., PradoAlcalá, R.A., 
Díaz-Miranda, S.Y., Densidad de las Espinas dendríticas en las células piramidales del CA1 del hipocampo después del 
aprendizaje espacial en ratas ovariectomizadas, XLIX Congreso Nacional de Ciencias Fisiológicas; Querétaro, Qro., 2006.

\item Aguilar Vazquez, A.R., Granados Rojas, L., Díaz Cintra, S.Y., Efecto de la malnutrición crónica sobre la densidad de las 
celulas gabaérgicas del giro dentado, XLVII Congreso Nacional de Ciencias Fisiológicas, Boca del Río, Veracruz, 2004.

\item C.R., Díaz del Guante, M.A., Garín-Aguilar, M.E., Quirare, G.L., Prado-Alcalá, E.A., Efecto de la inactivación del 
hipocampo dorsal (HD) con TTX sobre la retención de la memoria evocada de dos diferentes niveles de reforzamiento, XLVII 
Congreso Nacional de Ciencias Fisiológicas, Boca del Rio, Veracruz, 2004.

\item Martínez, M.I., Quirarte, G.L., y Prado-Alcala, R.A.,Efecto de la lesión bilateral del hipocampo ventral sobre el 
aprendizaje, XLVI Congreso Nacional de Ciencias Fisiólogicas, Universdad Autónoma de Aguascalentes, 2003.

\end{enumerate}
