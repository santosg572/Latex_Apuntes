\documentclass[12pt]{article}
\usepackage{lingmacros}
\usepackage{tree-dvips}
\begin{document}

CURRICULUM

V I TA E

González Santos Leopoldo

Género:

Fecha de nacimiento: Domicilio profesional:

Teléfono:

Fax:

Correo electrónico:

RFC o CURP:

Categoría:

PRIDE SNI Adscripción:

Número de plaza:

Total de citas: Responsable académico:

Masculino 1957-10-28

Laboratorio C-12, Instituto de Neurobiología - UNAM, Boulevard Juriquilla 3001, Juriquilla, Querétaro, México, C.P. 76230, 
Querétaro, México

52 (55) 5623 40 53, (442) 238 10 53

lgs@servidor.unam.mx GOSL571028H48 Técnico académico Tit. B T.C.

C Neurobiología Conductual y Cognitiva 803564 7 Esquivel Quiroz Bertha

Perfil

Categorías

1. Técnico académico Tit. B T.C., Instituto de Neurobiología UNAM-Juriquilla, 1996-02-01.

Escolaridad

1. Maestría en Ciencias de la Computación, Maestría, Centro de Investigación en Computación, IPN, 1999.

2. Licenciado en Física y Matemáticas, Licenciatura, Esc. Superior de Física y Matemáticas del I.P.N, 1980.



Asistencia a eventos académicos

1. Romero-Melendez, C., Castillo-Fernandez, D., and Gonzalez-Santos, L., “Asymptotic stability in a controlled stochastic 
Lotka-Volterra model with Lévy noise”, 13th International Conference on Pure and Applied Mathematics, ICPAM, País: Croacia, 
Tipo: Conferencia, Ámbito: Internacional, julio de 2024.

2. Romero-Melendez, C., Castillo-Fernandez, D., and Gonzalez-Santos, L., “On Stability Properties in a Stochastic 
Controlled Lotka-Volterra Model”, 7th International Conference on Mathematics and Statistics, ICoMS, País: Portugal, Tipo: 
Conferencia, Ámbito: Internacional, junio de 2024.

3. Vázquez, D., Martínez-Soto, J., Barrios, F., González-Santos, Pasaye, E., Alcauter, S., "Impact of a stress stimulus and 
images with restorative potential in the amygdala's resting state functional connectivity", , País: Estados Unidos de 
América, Tipo: Congreso, Ámbito: Internacional, noviembre de 2016.

4. Vazquez, D., Martinez-Soto, J., Barrios, F., Gonzalez-Santos, L., Pasaye, E., Alcauter, S., "Amygdala's functional 
connectivity in stress and after stimuli with low/high restorative potential", 22nd Annual Meeting of the OHBM. 
Organization for Human Brain Mapping, País: Suiza, Tipo: Congreso, Ámbito: Internacional, junio de 2016.

Participación institucional

1. Asistencia técnica de análisis de imagen, Instituto de Neurobiología UNAM-Juriquilla, INB-UNAM, INB-UNAM, Fecha de 
inicio: agosto de 2005, Fecha de conclusión: septiembre de 2005.

2. Asistencia técnica de análisis de imagen, Instituto de Neurobiología UNAM-Juriquilla, INB-UNAM, INB-UNAM, Fecha de 
inicio: julio de 2004, Fecha de conclusión: agosto de 2004.

3. Asistencia técnica de análisis de imagen, Instituto de Neurobiología UNAM-Juriquilla, INB-UNAM, INB-UNAM, Fecha de 
inicio: agosto de 2003, Fecha de conclusión: agosto de 2003.

4. Asistencia técnica de análisis de imagen, Instituto de Neurobiología UNAM-Juriquilla, INB-UNAM, INB-UNAM, Fecha de 
inicio: julio de 2003, Fecha de conclusión: julio de 2003.


Agradecimientos en capítulos en libros

1. Merchant, H, Zarco, W, Prado, L, Pérez, O, "Behavioral and Neurophysiological Aspects of Target Interception", Editor: 
Dagmar Sternad, Springer, Créditos en: Capítulo de libro nacional, 2009.


Premios y distinciones

1. Premio: Primer Lugar, "Segmentación semi-automática por crecimiento de regiones con estimación robusta: aplicación a 
imágenes por resonancia magnetica", Otorgado por: Federación Mexicana de Radiologia e Imagen, A. C, septiembre de 1998.

Estímulos académicos

1. PRIDE C Dirección General de Asuntos del Personal Académico, INB-UNAM, Fecha de inicio: marzo de 2003.

Proyectos

Proyectos

1. "Propiedades Turnpike en Problemas de Control Óptimo Estocástico", Responsable: Dr. Cutberto Romero Melèndez, Tipo: 
Investigación, Status: Inicio, Monto $0.00, Fecha de inicio: abril de 2023, Fecha de conclusión abril de 2026.


Artículos enviados o aceptados

1. Martínez-Soto, J., González-Santos, L., Pasaye, E.H., “Adityavarman, R., Barrios, F.A, “Restorative Influences of Nature 
vs. Urban Settings on Resting State Networks of FunctionalBrain Connectivity”,

SALVA - Plat. Inf. Curric. a 30 de julio de 2025 08:52, lgs@servidor.unam.mx desde la dirección 132.248.219.80

Página 25 de 35 Environmental Psychology, diciembre de 2017, Status: Enviado.

2. Romero-Meléndez, C., González-Santos, L., An Iterative Algorithm for Optimal Control of TwoLevel Quantum Systems, 
CYBERNETICS AND PHYSICS, diciembre de 2017; 6(4), 231-238, Status: Aceptado.

Artículos en memorias arbitradas

1. Romero-Meléndez, C, González-Santos, L.

Numerical Approximation in Optimal Control ofTwo-Level Quantum System"; PHYSCON, Florence, T, 2017.




Comités de congresos

1. V Mexican Symposium on Medical Physics, País: México, Tipo: Simposio, Ámbito: Internacional, marzo de 2001.























Actividades de difusión y extensión

1. Caravana del Cerebro: 25 de abril, Tipo: Semana del cerebro, abril de 2024.

2. Neuro-Emociónate", CAC-UNAM, del 19 al 23: , Tipo: Semana del Cerebro, marzo de 2024.

3. Semana del Cerebro: Centro Educativo y Cultural ‘Manuel Gómez Morin’, Querétaro, Qro., 19-23 marzo, Tipo: Semana del 
cerebro, marzo de 2019.

4. ¡Muévete ... Mantén activo tu Cerebro!: Semana del Cerebro, Tipo: En la semana del cerebro, marzo de 2016.

5. Feria de las Ciencias y las Humanidades: El cerebro del buen comer, Tipo: Conferencia, octubre de 2013.

6. “¿Qué tengo en mi cabeza?: Participación de Carteles, Tipo: Semana del cerebro, marzo de 2012.

7. Semana del Cerebro, CerebrAte: , Tipo: En la semana del cerebro, marzo de 2011.

8. Juego de Rompecabezas del Cerebro": Participación en las actividades de la Semana del Cerebro, Marzo, 16-20, Tipo: En la 
semana del cerebro, 2010.

9. Puedes ver los que pasa en tu cerebro...Sí!: La Sociedad Mexicana de Ciencias Fisiológicas y El Capítulo de la Ciudad de 
México de la "Society for Neuroscience",Marzo, 15-21, Tipo: Semana del

SALVA - Plat. Inf. Curric. a 30 de julio de 2025 08:52, lgs@servidor.unam.mx desde la dirección 132.248.219.80

Página 34 de 35 Cerebro, 2010.

10. Organización de la "Celebración Internacional del Cerebro": , Tipo: Semana del Cerebro, abril de 2006.

11. Participación en la "Celebración Internacional del Cerebro": , Tipo: Semana del Cerebro, marzo de 2005.

SALVA - Plat. Inf. Curric. a 30 de julio de 2025 08:52, lgs@servidor.unam.mx desde la dirección 132.248.219.80

Página 35 de 35

\end{document}

