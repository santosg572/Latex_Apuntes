\documentclass{beamer}
\usepackage[utf8]{inputenc}
\usepackage[spanish]{babel}
\usetheme{Warsaw}
\usecolortheme{crane}
\useoutertheme{shadow}
\useinnertheme{rectangles}

\title[santosg572@gmail.com]{Sistema de Numeración}
%\subtitle{Dando nombres a los animales}
\author[L. González-Santos]{
L. González-Santos$^{1}$}
\institute[EDEN \& HELL]{
  $^{1}$
  Instituto de Neurobiología, UNAM\\
  Campus Juriquilla, Qro.
  \and
  \texttt{lgs@unam.mx}
}
\date{\today}

\begin{document}

\frame{\titlepage}

\begin{frame}
\frametitle{Tipos de Números}

Números Naturales =  $\mathbb{N} = \{ 1, 2, 3, 4, ... \}$

\hfill

Números Enteros =  $\mathbb{Z} = \{..., -3, -2, -1, 0, 1, 2, 3, 4, ... \}$

\hfill

Números Racionales =  $\mathbb{Q} = \{ ..., -3,  0, 1, 2, \frac{4}{3}, -\frac{5}{6},  \}$

\hfill

Números Irracionales =  $\mathbb{I} = \{ e, -e , \sqrt{2}, \pi,... \}$ 

\hfill

Números Reales =  $\mathbb{R} = $\mathbb{Q} \bigcup \mathbb{I}$ 

\hfill

Se cumple: $\mathbb{Q} \bigcap \mathbb{I} = \O$ y  $\mathbb{N} \subset \mathbb{Z} \subset \mathbb{R} $ 

\end{frame}

\begin{frame}
\frametitle{Operadores Aritmético, Comparación, Funciones, Operadores Lógicos, Algunas Constantes}

\begin{itemize}
\item $+, -, /, *, \^, *$
\item $<, \leq, >, \geq, ==$
\item $\sqrt{x}, \sin{(x)}, \log{(x)}$
\item $\&,  ||$
\item $TRUE, FALSE, \infty$
\end{itemize}
\end{frame}

\begin{frame}
\frametitle{Vectores, Matrices y Arregos}

Un vector es definido por $x = \{x_1, x_2, x_3, ..., x_n\}$ donde $x_i \in \mathbb{R}$

\hfill

Una matrix es definida como:

\hfill

\begin{pmatrix}
a_{11} & a_{12}  & a_{13} & ... & a_{1n}  \\
a_{21} & a_{22}  & a_{23} & ... & a_{2n}  \\
... & ...  & ... & ... & ... \\
a_{m1} & a_{m2}  & a_{m3} & ... & a_{2mn}  \\
\end{pmatrix}

\hfill

Un arreglos es definido como

\begin{tiny}
\begin{pmatrix}
a_{111} & a_{121}  & a_{131} & ... & a_{1n1}  \\
a_{211} & a_{221}  & a_{231} & ... & a_{2n1}  \\
... & ...  & ... & ... & ... \\
a_{m11} & a_{m21}  & a_{m31} & ... & a_{mn1}  \\
\end{pmatrix}\begin{pmatrix}
a_{112} & a_{122}  & a_{132} & ... & a_{1n2}  \\
a_{212} & a_{222}  & a_{232} & ... & a_{2n2}  \\
... & ...  & ... & ... & ... \\
a_{m12} & a_{m22}  & a_{m32} & ... & a_{mn2}  \\
\end{pmatrix} ...

\vdots 

... \begin{pmatrix}
a_{11p} & a_{12p}  & a_{13p} & ... & a_{1np}  \\
a_{21p} & a_{22p}  & a_{23p} & ... & a_{2np}  \\
... & ...  & ... & ... & ... \\
a_{m1p} & a_{m2p}  & a_{m3p} & ... & a_{mnp}  \\
\end{pmatrix} 
\end{tiny} es te tamaño $m\times n \times p$
\end{frame}

\begin{frame}
\frametitle{Operadores Aritmético, Comparación}

\end{frame}

\begin{frame}
\frametitle{Operadores Aritmético, Comparación}

\end{frame}

\begin{frame}
\frametitle{Operadores Aritmético, Comparación}

\end{frame}

\begin{frame}
\frametitle{Operadores Aritmético, Comparación}

\end{frame}

\begin{frame}
\frametitle{Operadores Aritmético, Comparación}

\end{frame}

\begin{frame}
\frametitle{Operadores Aritmético, Comparación}

\end{frame}

\begin{frame}
\frametitle{Operadores Aritmético, Comparación}

\end{frame}

\begin{frame}
\frametitle{Operadores Aritmético, Comparación}

\end{frame}

\begin{frame}
\frametitle{Operadores Aritmético, Comparación}

\end{frame}

\begin{frame}
\frametitle{Operadores Aritmético, Comparación}

\end{frame}

\begin{frame}
\frametitle{Operadores Aritmético, Comparación}

\end{frame}

\begin{frame}
\frametitle{Operadores Aritmético, Comparación}

\end{frame}

\begin{frame}
\frametitle{Operadores Aritmético, Comparación}

\end{frame}

\begin{frame}
\frametitle{Operadores Aritmético, Comparación}

\end{frame}

\begin{frame}
\frametitle{Operadores Aritmético, Comparación}

\end{frame}

\begin{frame}
\frametitle{Operadores Aritmético, Comparación}

\end{frame}




\end{document}


