\documentclass{beamer}
\usetheme{Madrid} % Optional theme

\title{Teoría de Conjuntos}
\author{L. González-Santos}
\date{\today}

\begin{document}

% Title Page
\begin{frame}
    \titlepage
\end{frame}

% Table of Contents
\begin{frame}{Outline}
    \tableofcontents
\end{frame}

% Example Content Frame
\section{Introduction}

\begin{frame}{Frame Title}
    \frametitle{Conjuntos}

Intuitivamente, un \textbf{conjunto} es una colección de objetos, reales o imaginarios, llamados
\textbf{elementos} del conjunto. Esta definición tiene algunos problemas cuando los conjuntos son muy
grandes, pero lo importante es que cada conjunto está determinado por sus elementos:

\hfill

Dos conjuntos son iguales si tienen los mismos elementos.

\hfill

Para decir que x es elemento de A escribimos $x \in A$, y al conjunto que contiene a x,y,z se le
denota por {x,y,z} donde el orden no importa y no hay repeticiones

\end{frame}

\begin{frame}{Frame Title}
    \frametitle{Conjuntos}

\textbf{Ejemplos.}

\begin{itemize}
\item El conjunto de todos los seres vivos.
\item El conjunto de las especies de seres vivos.
\item El alfabeto ingles = \{a,b,c,d,e,f,g,h,i,j,k,l,m,n,o,p,q,r,s,t,u,v,w,x,y,z \}
\item El conjunto de los números naturales $\mathbb{N} = \{1,2,3,4,5,...\}$.
\item El conjunto $\mathbb{R}$ de todos los números reales.
\end{itemize}

Aunque los conjuntos pueden ser muy heterogéneos, los conjuntos mas útiles están formados por
elementos con alguna propiedad en común, algo como

\hfill

$\{ x | P(x) \}$ = El conjunto de los x que tienen la propiedad P(x)

\end{frame}

\begin{frame}{Frame Title}
    \frametitle{Conjuntos}

\textbf{Ejemplos:}

\hfill

\begin{itemize}
\item El conjunto de las canciones de los Beatles = $\{x | \text{x es canción y x fue escrita por los Beatles} \}$
\item El conjunto de los números primos = $\{ n \in Ν | n \text{ no es divisible por ningún m con } 1<m<n \}$
\item El conjunto de los números racionales $\mathbb{Q} = \{ r \in R | \exists  m,n \in Z, 
r=m/n \}$
\item El conjunto de los números irracionales $\mathbb{I} = \{ r \in R | \text{ NO } \exists 
m,n \in N, r=m/n \}$
\end{itemize}

\end{frame}

\begin{frame}{Frame Title}
    \frametitle{Conjuntos}

Un \textbf{conjunto vacío} es un conjunto que no tiene elementos, así que todos los conjuntos vacíos
son iguales, lo denotamos por $\{ \}$ o por $\O$.

\hfill

\textbf{Ejemplos.}

\hfill

\begin{itemize}
\item El conjunto de todos los perros voladores = $\O$ = El conjunto de los triángulos con 4 lados.
\item $\O = \{ \} \neq \{\{ \}\} = \{ \O \}$
\end{itemize}

\hfill

Si A y B son conjuntos, decimos que A esta \textbf{contenido} en B, o que A es \textbf{subconjunto} de B, si
todos los elementos de A son elementos de B, y escribimos $A \subset B$.
$$
A \subset B \ \forall \ x, x \in A \Rightarrow  x \in B
$$

Por el contrario, cuando A no esta contenido en B escribimos $A \nsubseteq B$

$$
A \nsubseteq B \ \forall \ \exists \ x, x \in A \notin B
$$

\end{frame} 

\begin{frame}{Frame Title}
    \frametitle{Conjuntos}

\textbf{Ejemplos.}

\hfill

1. Si A = El conjunto de todas las aves 

   B = El conjunto de todos los animales con alas 

   C = El conjunto de todos los animales que vuelan

Entonces $A \subset B , B \nsubseteq A , B \nsubseteq C , C \subset B , A \nsubseteq C , C 
\nsubseteq A$ .

\hfill

2. Las rectas y planos son conjuntos de puntos. Los puntos son elementos del plano mientras 
que las rectas son subconjuntos del plano.

\end{frame} 

\begin{frame}{Frame Title}
    \frametitle{Conjuntos}

\textbf{Afirmación.} Si A, B y C son conjuntos y $A \subset B y B \subset C entonces A \subset 
C$.

\hfill

\textbf{Afirmación.} A = B si y solo si $A \subset B$ y $B \subset A$


\end{frame} 

\begin{frame}{Frame Title}
    \frametitle{Conjuntos}

\textbf{Operaciones con conjuntos.}

\hfill

La \textbf{unión} de dos conjuntos A y B es el conjunto AUB formado por los elementos de A y 
los 
elementos de B, es decir
$$
A \cup B = { x | x \in A & x\in∈ B}
$$

La \textbf{intersección} de dos conjuntos A y B es el conjunto $A \cap B$ formado por los 
elementos 
de A y los 
elementos de B, es decir
$$
A \cap B = \{ x | x \in A \ & \ x \in B \}
$$

La \textbf{diferencia} entre dos conjuntos A y B es el conjunto A-B formado por los elementos 
de A que 
no son elementos de B.
$$
A - B = \{ x | x \in A \ & \ x \notin B \}
$$

\end{frame} 

\begin{frame}{Frame Title}
    \frametitle{Conjuntos}


Lema. Si A, B y C son conjuntos entonces se cumplen: 

\begin{enumerate}
\item $A \cup B = B \cup A$ La unión es conmutativa


A

U

B

=

B

U

A

2. A ∩ B = B ∩ A

3.

4.

5.

6.

La intersección es conmutativa La unión es asociativa La intersección es asociativa A ∩ (BUC) 
= (A∩B) U (A∩C) La intersección se distribuye sobre la unión A U (B∩C) = (AUB) ∩ (AUC) La 
unión se distribuye sobre la intersección

\end{frame} 

\begin{frame}{Frame Title}
    \frametitle{Conjuntos}


\end{frame} 

\begin{frame}{Frame Title}
    \frametitle{Conjuntos}


\end{frame} 

\begin{frame}{Frame Title}
    \frametitle{Conjuntos}


\end{frame} 

\begin{frame}{Frame Title}
    \frametitle{Conjuntos}


\end{frame} 




\end{document}

