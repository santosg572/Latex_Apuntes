Docencia

Cursos regulares

“Introducción al lenguaje de Programación R”, Modalidad: Presencial, Semestre: 1, Maestría en Ciencias Neurobiología, Maestría, Instituto de Neurobiología, Instituto de Neurobiologia, Tipo de programa académico: Escolarizado, Año del programa: 2023, Periodos: 2024-2, 2024-1.
“Introducción al lenguaje de Programación R”, Modalidad: Presencial, Semestre: 2, Maestía en Ciencias neurobiología, Maestría, Instituto de Neurobilogía, Instituto de Neurobiologia, Tipo de programa académico: Escolarizado, Año del programa: 2023, Periodos: 2024-1.
“Introducción a la programación y computación con Phyton”, Modalidad: Presencial, Semestre: 2, Maestria w¡en Ciencias Neurobiología, Maestría, Instituto Neurobiología, Instituto de Neurobiologia, Tipo de programa académico: Escolarizado, Año del programa: 2023, Periodos: 2023-2.
“Introducción al lenguaje de Programación R”, Modalidad: Presencial, Semestre: 2, Maestría en Ciencias, Neurobiología, Maestría, Instituto de Neurobiologia, Instituto de Neurobiologia, Tipo de programa académico: Escolarizado, Año del programa: 2023, Periodos: 2023-2.
Introducción a la Computación y Lenguaje de Programación R, Modalidad: A Distancia, Semestre: 1, Maestría en Ciencias (Neurobiología), Maestría, Instituto de Neurobiología, Instituo de Neurobiologia, Tipo de programa académico: Escolarizado, Año del programa: 2022, Periodos: 20222, 2022-1.
“Introducción a la Computación y Lenguaje de Programación PYTHON”, Modalidad: A Distancia, Semestre: 1, Maestría en Neurobiología, Maestría, Neurobiología, Instituo de Neurobiologia, Tipo de programa académico: Escolarizado, Año del programa: 2021, Periodos: 2021-2.
“Introducción a la Computación y Lenguaje de Programación R”, Modalidad: A Distancia, Semestre: 1, Maestría en Neurobiología, Maestría, Neurobiología, Instituo de Neurobiologia, Tipo de programa académico: Escolarizado, Año del programa: 2021, Periodos: 2021-2.
‘Introducción a la Computación y Lenguaje de Programación “R”’, Modalidad: Presencial, Semestre: 2, Maestría en Ciencias (Neurobiología), Maestría, Instituto de Neurobiología UNAMJuriquilla, Instituto de Neurobiología, Tipo de programa académico: Escolarizado, Año del programa: 2019, Periodos: 2021-1, 2020-2, 2020-1, 2019-2.
Introducción al Lenguaje de Programación "R", Modalidad: Presencial, Semestre: 2, Programa Maestría en Ciencias (Neurobiología), Maestría, Instituto de Neurobiología UNAM-Juriquilla, INBUNAM, Instituto de Neurobiología UNAM-Juriquilla, Tipo de programa académico: Escolarizado, Año del programa: 2018, Periodos: 2019-1.
Introducción al Lenguaje de Programación "R", Modalidad: Presencial, Semestre: 1, Programa Maestría en Ciencias (Neurobiología), Maestría, Instituto de Neurobiología UNAM-Juriquilla, INBUNAM, Instituto de Neurobiología UNAM-Juriquilla, Tipo de programa académico: Escolarizado, Año del programa: 2018, Periodos: 2018-2.
“Introducción al Lenguaje de Programación Python”, Modalidad: Presencial, Semestre: 1, Maestría en Ciencias (Neurobiología), UNAM, Doctorado, Instituto de Neurobiología UNAMJuriquilla, INB-UNAM, Instituto de Neurobiología UNAM-Juriquilla, Tipo de programa académico: Escolarizado, Año del programa: 2018, Periodos: 2018-1.
“Introducción al Lenguaje de Programación R”, Modalidad: Presencial, Semestre: 2, Maestría en Ciencias (Neurobiología), Doctorado, Instituto de Neurobiología UNAM-Juriquilla, INB-UNAM, Instituto de Neurobiología UNAM-Juriquilla, Tipo de programa académico: Escolarizado, Año del programa: 2017, Periodos: 2017-2.
“Métodos Numéricos”, Modalidad: Presencial, Semestre: 2, Maestría en Ciencias (Neurobiología), Doctorado, Instituto de Neurobiología UNAM-Juriquilla, INB-UNAM, Instituto de Neurobiología UNAM-Juriquilla, Tipo de programa académico: Escolarizado, Año del programa: 2017, Periodos: 2017-1.
Métodos Numéricos, Modalidad: Presencial, Semestre: 1, Créditos: 4, Maestría en Ciencias (Neurobiología), Maestría, Instituto de Neurobiología UNAM-Juriquilla, INB-UNAM, Instituto de Neurobiología UNAM-Juriquilla, Tipo de programa académico: Escolarizado, Año del programa: 2016, Periodos: 2017-1.
Estadística, Modalidad: Presencial, Semestre: 2, Créditos: 8, Maestría en Ciencias (Neurobiología), Maestría, Instituto de Neurobiología UNAM-Juriquilla, INB-UNAM, Instituto de Neurobiología UNAM-Juriquilla, Tipo de programa académico: Escolarizado, Año del programa: 2016, Periodos: 2016-2.
Estadística, Modalidad: Presencial, Semestre: 2, Maestría en Ciencias (Neurobiología), Maestría, Instituto de Neurobiología UNAM-Juriquilla, INB-UNAM, Instituto de Neurobiología UNAM-Juriquilla, Tipo de programa académico: Escolarizado, Año del programa: 2015, Periodos: 2015-2.
Métodos Numéricos, Modalidad: Presencial, Semestre: 2, Maestría en Ciencias (Neurobiología), Maestría, Instituto de Neurobiología UNAM-Juriquilla, INB-UNAM, Instituto de Neurobiología UNAMJuriquilla, Tipo de programa académico: Escolarizado, Año del programa: 2015, Periodos: 2015-2.
Estadística, Modalidad: Presencial, Semestre: 2, Programa de Maestría en Ciencias (Neurobiología), Maestría, Instituto de Neurobiología UNAM-Juriquilla, INB-UNAM, Instituto de Neurobiología UNAM-Juriquilla, Tipo de programa académico: Escolarizado, Año del programa: 2014, Periodos: 2015-1.
Lenguaje de Programación MatLab, Modalidad: Presencial, Semestre: 1, Maestría en Ciencias (Neurobiología), Maestría, Instituto de Neurobiología UNAM-Juriquilla, INB-UNAM, Instituto de Neurobiología UNAM-Juriquilla, Tipo de programa académico: Escolarizado, Año del programa: 2012, Periodos: 2012-1.
Lenguaje de Programación Matlab, Modalidad: Presencial, Semestre: , Maestría en Ciencias (Neurobiología), Maestría, Instituto de Neurobiología UNAM-Juriquilla, INB-UNAM, Instituto de Neurobiología UNAM-Juriquilla, Tipo de programa académico: Escolarizado, Año del programa: 2009, Periodos: 2008-1.
Lenguaje Pascal y “C”, Modalidad: Presencial, Semestre: 2, Ingeniero en Computación, Licenciatura, Tecnológico de Estudios Superiores de Ecatepec, Tipo de programa académico: Escolarizado, Año del programa: 1996, Periodos: Indefinido.
Mecánica, Modalidad: Presencial, Semestre: 2, Ingeniero Químico, Licenciatura, Facultad de Química de la UAQ. Querétaro, México, Tipo de programa académico: Escolarizado, Año del programa: 2001, Periodos: Indefinido.
Mecánica, Modalidad: Presencial, Semestre: 3, Fisíco, Licenciatura, Facultad de Química de la UAQ. Querétaro, México, Tipo de programa académico: Escolarizado, Año del programa: 2001, Periodos: Indefinido.
Taller de Matemáticas, Modalidad: Presencial, Semestre: , Maestría en Ciencias (Neurobiología), Maestría, Instituto de Neurobiología UNAM-Juriquilla, INB-UNAM, Instituto de Neurobiología UNAMJuriquilla, Tipo de programa académico: Escolarizado, Año del programa: 1997, Periodos: Indefinido.
Lenguaje de Programación MatLab, Modalidad: Presencial, Semestre: 1, Maestría en Ciencias (Neurobiología), UNAM, Maestría, Instituto de Neurobiología UNAM-Juriquilla, INB-UNAM, Instituto de Neurobiología UNAM-Juriquilla, Tipo de programa académico: Escolarizado, Año del programa: 2010, Periodos: 2010-1.

