\documentclass{beamer}
\usepackage[spanish]{babel}
\usepackage[latin1]{inputenc}
\usepackage{graphicx}
\usetheme{Warsaw}
\usecolortheme{crane}
\useoutertheme{shadow}
\useinnertheme{rectangles}

\title[lgs@unam.mx]{Problemas que se pueden resolver}
%\subtitle{Dando nombres a los animales}

\date{\today}

\begin{document}

\frame{\titlepage}

\begin{frame}
    \frametitle{Calcular:}
    
    \begin{enumerate}
    \item $1 + 2 + 3 + ... + 1000$
    \vspace{5mm}
    \item $2 + 4 + 6 + ... + 1000$
    \vspace{5mm}
    \item $100 + 101 + 102 + ... + 1100$
    
    
    \end{enumerate}
    
\end{frame}

\begin{frame}
    \frametitle{Encontrar:}
    
    \begin{enumerate}
    \item Un polinomio que pase por los puntos \{ (1,1), (2,4), (3, 1), (6, 3) \}
    \vspace{5mm}
    \item Una recta que sea representativa de los suntos \{ (1,1.3), (2,1.6), (3, 3.5), (4, 4), (5, 4.8), (6, 5.5), (7, 7.5), 
(8, 7.6), (9, 8.2) \}
    \vspace{5mm}
    \item Dada la función $f(x)=x^2$ definida en el intervalo [0,4]. 
    
    \renewcommand{\labelenumii}{\Roman{enumii}}
    \begin{itemize}
    \item Trazar una recta tangente en el punto (1,1).
    \vspace{5mm}
    \item Encontrar el área bajo la curva en el dintervalo [0,4].   
    \end{itemize}
    
    
    \end{enumerate}
    
\end{frame}

\begin{frame}
    \frametitle{Encontrar patrones:}
    
\includegraphics[scale=0.5]{pat01.png}


\end{frame}
    


\end{document}


